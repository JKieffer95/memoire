\documentclass[11pt,a4paper]{article}

%Encoding and language
\usepackage[T1]{fontenc}
\usepackage[utf8]{inputenc}
\usepackage[french]{babel}

%AMS packages
\usepackage{amsmath, amsthm, amssymb}

%Fonts and graphics
\usepackage{graphicx, textcomp, lmodern, fullpage, url, tikz}

%Diagrams
\usepackage[all]{xy}

%Algorithms
\usepackage[ruled, vlined, linesnumbered, french]{algorithm2e}

%Additional commands
\newcommand{\Z}{\mathbb{Z}}
\newcommand{\N}{\mathbb{N}}
\newcommand{\C}{\mathbb{C}}
\newcommand{\A}{\mathbb{A}}
\newcommand{\F}{\mathbb{F}}
\newcommand{\Q}{\mathbb{Q}}
\newcommand{\R}{\mathbb{R}}
\newcommand{\E}{\mathcal{E}}
\renewcommand{\Pr}{\mathcal{P}}
\newcommand{\Qr}{\mathcal{Q}}
\renewcommand{\H}{\mathbb{H}}
\renewcommand{\P}{\mathbb{P}}
\newcommand{\M}{\mathcal{M}}
\renewcommand{\O}{\mathcal{O}}
\newcommand{\Cl}{\mathcal{C}}
\renewcommand{\b}{\backslash}
\newcommand{\vers}{\longrightarrow}
\newcommand{\End}{\mathrm{End}}
\newcommand{\Hom}{\mathrm{Hom}}
\newcommand{\Ell}{\mathrm{Ell}}
\newcommand{\Jac}{\mathrm{Jac}}
\newcommand{\Gal}{\mathrm{Gal}}
\newcommand{\Spec}{\mathrm{Spec}\,}
\newcommand{\Ker}{\mathrm{Ker}\,}
\renewcommand{\frak}{\mathfrak}
\newcommand{\de}{\,:\,}
\newcommand{\id}{\mathrm{id}}
\newcommand{\pr}{\mathrm{pr}}
\renewcommand{\mod}{\ \mathrm{mod}\ }
\renewcommand{\v}{\vspace{5mm}}

%Theorem environments
\newtheorem*{thm}{Théorème}
\newtheorem*{lem}{Lemme}
\newtheorem*{prop}{Proposition}
\newtheorem*{cor}{Corollaire}
\newtheorem*{hyp}{Hypothèse}
\theoremstyle{definition}
\newtheorem*{rem}{Remarque}
\newtheorem*{defi}{Définition}
\newtheorem*{ex}{Exemples}

%Titlepage
\definecolor{darkgreen}{RGB}{0, 150, 0}

\begin{document}

\begin{titlepage}

\centering

{\Huge Un cryptosystème à base d'isogénies}
\vspace{1cm}

{\large Mémoire du Master 2 Mathématiques fondamentales de l'Université Paris VI}
\vspace{1cm}

{\Large Jean Kieffer}
\v

\today

\vfill

\vfill

\vfill

\begin{tikzpicture}[x=6cm, y=6cm]

\draw[]
 (1, 0) node(7) {2}
 (0.84, 0.54) node(443) {443}
 (0.42, 0.91) node(362) {362}
 (-0.14,0.99) node(104) {104}
 (-0.65,0.76) node(200) {200}
 (-0.96,0.28) node(468) {468}
 (-0.96, -0.28) node(84) {84}
 (-0.65, -0.76) node(385) {385}
 (-0.14, -0.99) node(242) {242}
 (0.42, -0.91) node(402) {402}
 (0.84, -0.54) node(341) {341};
\draw[]
 (7) edge[thick, blue, bend right = 10] (443)
 (443) edge[thick, blue, bend right = 10] (362)
 (362) edge[thick, blue, bend right = 10] (104)
 (104) edge[thick, blue, bend right = 10] (200)
 (200) edge[thick, blue, bend right = 10] (468)
 (468) edge[thick, blue, bend right = 10] (84)
 (84) edge[thick, blue, bend right = 10] (385)
 (385) edge[thick, blue, bend right = 10] (242)
 (242) edge[thick, blue, bend right = 10] (402)
 (402) edge[thick, blue, bend right = 10] (341)
 (341) edge[thick, blue, bend right = 10] (7);

\draw[]
 (7) edge[thick, red, bend left=45] (468)
 (468) edge[thick, red, bend left=45] (341)
 (341) edge[thick, red, bend left=45] (200)
 (200) edge[thick, red, bend left=45] (402)
 (402) edge[thick, red, bend left=45] (104)
 (104) edge[thick, red, bend left=45] (242)
 (242) edge[thick, red, bend left=45] (362)
 (362) edge[thick, red, bend left=45] (385)
 (385) edge[thick, red, bend left=45] (443)
 (443) edge[thick, red, bend left=45] (84)
 (84) edge[thick, red, bend left=45] (7);
 
\draw[]
 (7) edge[thick, darkgreen, bend left=10] (242)
 (242) edge[thick, darkgreen, bend left=10] (468)
 (468) edge[thick, darkgreen, bend left=10] (362)
 (362) edge[thick, darkgreen, bend left=10] (341)
 (341) edge[thick, darkgreen, bend left=10] (385)
 (385) edge[thick, darkgreen, bend left=10] (200)
 (200) edge[thick, darkgreen, bend left=10] (443)
 (443) edge[thick, darkgreen, bend left=10] (402)
 (402) edge[thick, darkgreen, bend left=10] (84)
 (84) edge[thick, darkgreen, bend left=10] (104)
 (104) edge[thick, darkgreen, bend left=10] (7);

\end{tikzpicture}

\vfill

\end{titlepage}

Le sujet de ce mémoire rentre dans le cadre de la \emph{cryptographie à base d'isogénies} entre courbes elliptiques. Les courbes elliptiques sont des groupes algébriques, et sont donc utilisables comme brique de base pour de nombreux protocoles. L'utilisation de courbes elliptiques en cryptographie est relativement jeune ; la plupart du temps, on utilise une courbe elliptique en cryptographie parce que le problème du logarithme discret pour les points de cette courbe est réputé difficile (plus difficile que dans le groupe multiplicatif d'un corps fini, par exemple).

Dans ce cadre, faire intervenir des isogénies entre courbes elliptiques est une idée récente, qui date du début des années 2000. À l'origine, ces morphismes entre courbes sont plutôt utilisés de manière destructrice, pour ramener une courbe elliptique donnée vers une courbe elliptique plus faible, c'est à dire plus facile à casser. Encore plus récemment, on a commencé à utiliser des isogénies de manière constructive, en proposant des protocoles cryptographiques les utilisant. La tête de pont de ce mouvement est le protocole SIDH, où interviennent des isogénies de petit degré entre courbes elliptiques sur un corps fini.

Le point de départ de ce document est une proposition de Jean-Marc Couveignes en 2006 d'utiliser une certaine action d'un groupe de classes sur un ensemble de courbes elliptiques pour construire un protocole d'échange de clés. En quelque sorte, il s'agit de l'analogue du système SIDH pour des courbes elliptiques ordinaires. Cette idée a été proposée la même année par Rostovtsev et Stolbunov, qui décrivent ce protocole plus en détail et en donnent quelques exemples jouets. L'objectif original de ce document est d'étudier ce protocole cryptographique, et d'être capable de comprendre précisément quels doivent être ses paramètres et quelle est son efficacité réelle.

\v
Avant de pouvoir décrire ce cryptosystème, il faut comprendre les propriétés de structure des isogénies entre courbes elliptiques sur un corps fini, données par la théorie de la \emph{multiplication complexe}. C'est l'objet de la première partie de ce document, qui introduit les propriétés et les concepts fondamentaux pour la suite. Lorsque l'on parle d'isogénies entre des courbes elliptiques, on est également rapidement amené à parler de \emph{courbes modulaires}, qui sont des espaces naturels paramétrisant les courbes elliptiques munies d'une certaine structure. Leur description fait l'objet de la deuxième partie, d'abord sur les nombres complexes où elles sont étroitement reliées aux formes modulaires, puis de façon plus générale. Pour cela, on prendra le point de vue de la géométrie algébrique moderne, qui décrit ces courbes en tant que schémas.

Après ces préparations, on parle du protocole de Couveignes--Rostovtsev--Stolbunov lui-même et du calcul d'isogénies entre courbes elliptiques \og en pratique\fg. Le but de la troisième partie est d'introduire différents algorithmes utilisés pour ce calcul, et de discuter leur coût; on y apprend également ce que représente ce mystérieux graphe de couverture. Comme tous les protocoles cryptographiques, certains paramètres restent à déterminer, et ceux-ci dépendent
étroitement de la qualité des attaques disponibles contre le protocole. Cette étude est faite en quatrième partie, où l'on montre également que la sécurité du cryptosystème est étroitement liée à des questions fines de structure de certains groupe de classes. De nombreuses réponses à ces questions d'arithmétique restent aujourd'hui inconnues. On discute également de la recherche d'une bonne courbe initiale, dont dépendent directement les performances du protocole.

Enfin, ce stage a été l'occasion d'\emph{implémenter} ce protocole cryptographique à l'aide de logiciels de calcul formel. Ce travail, assez chronophage lorsque l'on débute en programmation sérieuse, est décrit dans la cinquième et dernière partie du document. On obtient ainsi une mesure précise du coût de différentes étapes critiques des algorithmes évoqués ci-dessus, ce qui permet de se faire une idée de leur coût total, et permet également de désigner les meilleurs paramètres initiaux que l'on a pu trouver. À terme, ce code est destiné à devenir un module open-source pour le logiciel de calcul formel Nemo contenant différentes primitives utiles en cryptographie sur courbes elliptiques.



\v

Ce mémoire a été réalisé au cours du stage de second semestre du Master 2 de Mathématiques fondamentales, à l'université Paris VI. Ce stage s'est déroulé entre mars et août 2017, au sein de l'équipe GRACE (Geometry, Arithmetic, Codes and Encryption) de l'organisme de recherche publique français en mathématiques et informatique, Inria. Une partie du contenu de ce mémoire a fait l'objet d'un poster présenté à la conférence ISSAC (International Symposium on Symbolic and Algebraic Computation) qui a eu lieu au mois de juillet 2017 à Kaiserslautern. Celui-ci a été récompensé du Best Poster Award à cette occasion.

Un grand merci à Daniel Augot et aux membres de l'équipe Grace pour leur excellent accueil, à Jessica Gameiro pour le soutien logistique, à Benoît Stroh, Andrew Sutherland et Jean-Pierre Flori pour des échanges de mails éclairants. Enfin, bien sûr, merci à Luca De Feo et Benjamin Smith pour leur encadrement bienveillant et leurs réponses à mes (nombreuses) questions.

\vspace{1cm}

\tableofcontents

\newpage

\section{Courbes elliptiques}


\subsection{Définitions et propriétés basiques}


Une \emph{courbe elliptique} $E$ sur un corps $k$ est une courbe algébrique propre et lisse définie sur $k$ et de genre 1, munie d'un $k$-point fixé $\O_E$. Un exemple typique est donné par les \emph{équations de Weierstrass}:
$$y^2 + a_1xy = x^3 + a_2x^2 + a_4 x + a_6.$$
Cette courbe affine plane, lorsqu'elle est lisse (ce qui est équivalent à la non-nullité d'un certain discriminant, qui est une fonction polynomiale des $a_i$), peut être complétée en une courbe projective en ajoutant un unique point à l'infini. On obtient alors une courbe elliptique en prenant ce dernier point comme \og origine\fg\ de la courbe.

Soit $E$ une courbe elliptique sur $k$ dans le sens abstrait donné ci-dessus. Comme à toute courbe algébrique lisse, on peut lui attacher un schéma en groupes abélien défini sur $k$, sa \emph{jacobienne}. Un schéma en groupes (qui sera toujours abélien dans ce document) est simplement un schéma dont les points forment toujours un groupe abélien, c'est à dire un ensemble d'équations algébriques (ici à coefficients dans $k$) dont les solutions forment toujours un groupe. $GL_n$, le groupe des matrices carrées de déterminant inversible, ou $\mu_n$, le groupe des racines $n$-ièmes de l'unité, en sont des exemples qui sont définis sur $\Z$, donc aussi sur $k$. On peut définir les $\bar{k}$-points de $\Jac(E)$ de la manière suivante:
\begin{itemize}
\item[•] Un \emph{diviseur} de $E$ est une somme formelle de points de $E(\bar{k})$ à coefficients dans $\Z$.
\item[•] Le \emph{degré} d'un diviseur est la somme de ses coefficients.
\item[•] Le diviseur d'une fonction sur $E$ à valeurs dans $k$ (non nulle) est la somme de ses zéros et pôles avec multiplicité, et est systématiquement de degré zéro. Ces diviseurs sont dits \emph{principaux}.
\item[•] Le quotient du groupe des diviseurs de degré zéro par celui des diviseurs principaux forme le groupe des $\bar{k}$-points de la jacobienne de $E$.
\end{itemize}
Le théorème de Riemann--Roch, un outil puissant dans l'étude des courbes algébriques, permet de contrôler le nombre de fonctions sur $E$ dont le diviseur satisfait une certaine condition, qui se traduit en des contraintes de zéros et de pôles sur ces fonctions. Il implique que l'on dispose d'un unique isomorphisme entre la courbe $E$ et sa jacobienne envoyant $\O_E$ sur zéro, que l'on peut donner sur les points de la manière suivante:
$$\begin{aligned}
E(\bar{k}) &\overset{\sim}{\vers} \Jac(E)(\bar{k}) \\
P &\longmapsto \text{Classe de } (P) - (\O_E).
\end{aligned}$$
L'idée est la suivante: soit $D$ un diviseur de degré zéro. Alors $-D - (\O_E)$ est de degré $-1$, et il existe essentiellement une unique fonction $f$ dont les pôles et zéros sont contrôlés par ce diviseur, par le théorème de Riemann-Roch. Par la condition de degré, $f$ a un unique zéro supplémentaire, $P$. Le diviseur de $f$ est donc $- D - (\O_E) + (P)$, ce qui montre que $D = (P) - (\O_E)$ dans la jacobienne. On a donc montré la surjectivité de l'application ci-dessus.

Cet isomorphisme permet de munir $E$ d'une structure de $k$-schéma en groupes, en particulier de munir l'ensemble $E(L)$ des $L$-points de $E$ d'une structure de groupe pour toute extension $L/k$, dont l'élément neutre est $\O_E$. Le théorème de Riemann--Roch implique également que toute courbe elliptique sur $K$ admet une équation plane sous forme de Weierstrass, en regardant cette fois les fonctions ayant un pôle uniquement en $\O_E$: la fonction $x$ de l'équation plane précédente y admet un pôle double, $y$ un pôle triple, et l'équation polynomiale provient du fait qu'il y a trop de fonctions ayant un pôle d'ordre au plus 6 en ce point. Bien sûr, ce modèle n'a rien de canonique et on peut donner une courbe elliptique par d'autres équations. Si la caractéristique de $k$ est différente de 2 et 3, on peut de plus effectuer quelques changements de coordonées et choisir $a_1 = a_2 = a_3 = 0$, et l'on obtient alors une forme de Weierstrass dite \emph{réduite} ou \emph{courte}.

\v

Une \emph{isogénie} de $E$ vers $E'$ est une application rationnelle surjective envoyant $\O_E$ sur $\O_{E'}$, et est automatiquement un morphisme de groupes. Chez les courbes, une application rationnelle est surjective si et seulement si elle est non nulle, mais cette définition est la bonne si l'on souhaite l'étendre à des variétés abéliennes de dimension supérieure. En ajoutant l'application nulle $x\mapsto \O_{E'}$, on obtient les \emph{morphismes} entre $E$ et $E'$. Notons qu'un morphisme entre deux courbes elliptiques sur $k$ n'est pas nécessairement lui-même défini sur $k$. Par exemples, deux courbes elliptiques peuvent être isomorphes sur une extension quadratique de $k$ mais pas sur $k$ lui-même. Elles sont alors dites \emph{tordues} l'une de l'autre.

Notons $\End(E)$ l'anneau d'endomorphismes de $E$ : la structure de groupe donne une flèche
$$\begin{aligned}
&\Z &\longrightarrow&\ &\End(E) &\\
&n &\longmapsto& &[n]_E .\ \ &
\end{aligned}$$
Pour en dire plus sur la structure de $\End(E)$, on peut montrer que pour toute isogénie $\phi\de E\vers E'$ de degré $m$, il existe une unique isogénie notée $\widehat{\phi}\de E'\vers E$ de degré $m$ telle que $\phi\widehat{\phi}=[m]_{E'}$ et $\widehat{\phi}\phi=[m]_{E}$, que l'on appelle \emph{duale} de $\phi$. L'isogénie duale d'une somme est la somme des isogénies duales. On en déduit que le degré est une forme quadratique sur $\End(E)$ et $[m]_E$ est de degré $m^2$, donc la flèche précédente est injective.

On a alors trois possibilités: $\End(E)$ est soit
\begin{enumerate} 
\item simplement $\Z$, 
\item un ordre dans un corps quadratique imaginaire,
\item un ordre dans une algèbre de quaternions sur $\Q$. 
\end{enumerate}
Un \emph{ordre} est un sous-anneau qui est de type fini sur $\Z$ et engendre tout l'espace comme $\Q$-espace vectoriel. Par exemple, l'anneau des entiers $\O_K$ d'un corps quadratique imaginaire $K$ (on peut penser à $\Z[i]$ dans $\Q[i]$) est un ordre de $K$. C'est d'ailleurs l'ordre maximal, et tous les ordres de $K$ sont des sous-ordres de $\O_K$. Cette alternative à trois choix est le mieux que l'on puisse faire dans le cas général, mais les choses s'arrangent un peu lorsqu'on regarde deux cas particuliers: les corps finis d'une part, et la caractéristique nulle d'autre part. Dans l'optique d'utiliser des courbes elliptiques en cryptographie, ce sont les premières qui nous intéressent.

\v

\textbf{Courbes elliptiques sur un corps fini.} Reprenons les trois cas ci-dessus. Si $k = \F_p$, où $p$ est un nombre premier, on dit que $E$ est \emph{ordinaire} dans le second cas et que $E$ est \emph{supersingulière} dans le troisième, le premier cas étant exclu par la présence du morphisme de Frobenius
$$\pi_E\de (x,y)\mapsto (x^p, y^p)$$
qui est de degré $p$, donc n'est pas un élément de $\Z$. On vient d'écrire le Frobenius pour une équation plane de $E$, mais il existe une définition plus intrinsèque, l'idée générale restant de tout mettre à la puissance $p$.

La théorie de Hasse nous dit que $\pi_E$ vérifie une équation de la forme
$$\pi_E^2 - t \pi_E + p = 0.$$
L'autre élément de $\End(E)$ vérifiant cette équation est le dual du Frobenius.
Le nombre $t$, appelé \emph{trace} de $\pi_E$ (ou de la courbe $E$), est un entier vérifiant $|t|\leq 2\sqrt{p}$. On peut montrer que le cardinal de l'ensemble $E(\F_p)$ est donné par
$$\# E(\F_p) = p + 1 - t,$$
d'où les \emph{bornes de Hasse} :
$$p + 1 - 2\sqrt{p} \leq \#E(\F_p) \leq p + 1 + 2\sqrt{p}$$
qui jouent un rôle important en pratique dans le calcul de $\# E(\F_p)$. On peut facilement lire sur le cardinal de $E(\F_p)$ le caractère supersingulier ou ordinaire de $E$: la courbe $E$ est supersingulière si et seulement si on a l'égalité $t = 0 \mod p$, ce qui est équivalent à $t=0$ dès que $p > 2\sqrt{p}$, c'est à dire $p \geq 5$. On peut montrer que ces résultats sont en fait valables sur tout corps fini, en remplaçant $p$ par $q$, une puissance de nombre premier.

\v

\textbf{Courbes elliptiques en caractéristique zéro}. Tout d'abord, le principe de plongement de Lefschetz permet de ramener les corps de caractéristique zéro au cas complexe. Si $E/\C$ est une courbe elliptique, c'est en particulier une surface de Riemann. On peut montrer qu'il existe $\tau\in \H$, le demi-plan de Poincaré, tel que l'on ait un isomorphisme
$$\C/\Lambda_\tau \overset{\sim}{\vers} E.$$
On a noté ici $\Lambda_\tau = \Z\oplus \Z\tau$, et cet isomorphisme est donné par la fonction $\wp$ de Weierstrass et sa dérivée. Le nombre $\tau$ est unique modulo l'action de $\Gamma(1) = \mathrm{SL}_2(\Z)$ sur $\H$ par homographies. Dans cette description, une fonction sur $E$ à valeurs complexes est simplement une fonction méromorphe sur le plan complexe admettant deux périodes, données par les deux vecteurs d'une base de $\Lambda_\tau$. D'un point de vue historique, l'étude de ces \emph{fonctions elliptiques} (dont un membre éminent est la longueur d'arc d'une ellipse) est le point de départ de celle des courbes elliptiques elles-mêmes. C'est là l'origine d'un nom peut-être malheureux, car une courbe elliptique et une ellipse n'ont pas grand-chose en commun.

Dans ce contexte, une isogénie $E\vers E'$ est simplement la multiplication par un $\alpha\in \C^*$ tel que $\alpha \Lambda_\tau \subset \Lambda_\tau'$. Avec cette description explicite, il est facile de vérifier les différentes propriétés exposées plus haut: par exemple, le dual d'une isogénie est simplement son conjugué complexe. On voit également que $\End(E)$ ne peut jamais être un ordre d'une algèbre de quaternions. Tout cela motive une terminologie différente du cas des corps finis: on dira que $E$ est \emph{à multiplication complexe} si $\End(E)$ est un ordre quadratique imaginaire.
% Sinon, on dira que... $E$ est \emph{sans multiplication complexe}, le terme \og ordinaire\fg\ étant déjà pris. 
On utilise parfois également le terme de multiplication complexe pour parler de courbes elliptiques ordinaires sur un corps fini, leurs propriétés étant globalement similaires.

Cette description agréable en termes de réseaux ne tient plus en caractéristique positive, et le passage à celle-ci semble toujours faire intervenir une preuve non triviale. Dans ce mémoire, on se servira du cas complexe comme une motivation pour la démonstration des résultats généraux.

\v

Le matériel aperçu dans cette introduction est étudié plus en profondeur (et plus proprement) dans plusieurs cours sur les courbes elliptiques, comme \cite{Nekovar} et \cite{Stroh}.
Le but de ce document étant de travailler avec des courbes elliptiques ordinaires sur un corps fini, on commence par étudier la théorie de la multiplication complexe sur $\C$.


\subsection{Multiplication complexe sur $\C$}


La référence principale de cette section est \cite{Sil2}. On fixe ici un ordre $\O$ dans un corps quadratique imaginaire $K\subset \C$, une extension de $\Q$ de degré 2 vérifiant $K \not\subset\R$: concrètement, on a $K=\Q(\sqrt{-d})$ pour un certain entier $d\geq 2$ sans facteur carré. On note $\Ell_\C(\O)$ l'ensemble des courbes elliptiques sur $\C$ (à isomorphisme près) ayant multiplication complexe par $\O$, c'est à dire les courbes $E_\tau = \C/\Lambda_\tau$ pour lesquelles on a:
$$\End(E_\tau) = \{\alpha\in \C\ |\ \alpha\Lambda_\tau \subset \Lambda_\tau\} = \O.$$
On identifiera un élément de cet ensemble avec ses représentants. Dans \cite{Sil2}, on suppose que $\O$ est l'anneau d'entiers de $K$, ce qui permet de simplifier certaines démonstrations; cette supposition n'est pas essentielle en réalité, et on énonce ici les résultats pour un ordre quelconque. %Dans le cas général, $\O$ n'est pas un anneau de Dedekind (n'étant pas intégralement clos) mais reste un anneau n\oe thérien de dimension 1.
Si $\O_K$ désigne l'anneau d'entiers de $K$, on appelle $f=[\O:\O_K]$ le \emph{conducteur} de $\O$, et l'on a $\O = \Z  + f \O_K$. Le discriminant de $\O$ est alors $D = f^2 D_K$. Les idéaux de $\O$ premiers à $f$ sont les idéaux inversibles de $\O$ et engendrent un groupe d'idéaux fractionnaires, que l'on quotiente par les idéaux principaux pour obtenir le \emph{groupe de classes d'idéaux} $\Cl(\O)$ de $\O$.
\v

La première étape est de fixer un isomorphisme $[\,\cdot\,]\de \O\vers \End(E)$ pour toute courbe $E\in \Ell_\C(\O)$ afin de pouvoir identifier ces anneaux. On fait cela en regardant l'action sur le $\C$-espace vectoriel de dimension 1 des formes différentielles sur $E$, en imposant
$$\forall \omega\in H^0(E,\Omega^1),\ \forall\,\alpha\in\O,\ [\alpha]^*\omega = \alpha \omega.$$
Cela coïncide bien avec la description précédente: sur $E_\tau = \C/\Lambda_\tau$, la différentielle $dz$ est invariante, et l'on a bien $\alpha^* dz = \alpha dz$. Cette condition est suffisante pour fixer un unique isomorphisme, car l'action sur la différentielle invariante n'est nulle que si l'endomorphisme en question est nul. Cela provient du fait que tous les morphismes sont \emph{séparables} en caractéristique zéro.

Le but est maintenant de définir une action de $\Cl(\O)$ sur ces courbes elliptiques. Au niveau des idéaux, on procède de la manière suivante: soit $E\in \Ell_\C(\O)$. En raisonnant en termes de réseaux, on voit que pour tout sous-groupe fini $S$ de $E(\C)$, il existe une courbe elliptique $E'$ et une isogénie $E\vers E'$ de noyau $S$, et celle-ci est unique à isomorphisme près (on déclare que deux isogénies sont isomorphes si elles diffèrent d'un isomorphisme à l'arrivée). Soit $\frak a$ un idéal inversible de $\O$. On définit
$$E[\frak a]=\bigcap_{\phi\in \frak a} \mathrm{Ker}\,\phi$$
(rappelons l'identificiation $\End(E)\simeq\O$ ci-dessus). On note $\phi_{\frak a}$ l'isogénie de noyau $E[\frak a]$, et $\frak a\cdot E$ sa courbe image, qui est bien définie comme élément de $\Ell_\C(\O)$.

\v

Pour montrer que cette action par isogénies en est bien une, on peut faire appel une fois de plus aux réseaux et montrer la propriété suivante : pour tout idéal inversible $\frak a$ de $\O$ et toute courbe $E=\C/\Lambda\in \Ell_\C(\O)$, on a
$$\frak a\cdot E = \C/\frak a^{-1} \Lambda.$$
On en déduit que le degré de $\phi_{\frak a}$ est la norme de l'idéal $\frak a$, et que la courbe $\frak a\cdot E$ a également multiplication complexe par $\O$. On en déduit aussi la transitivité, et il est immédiat que les idéaux principaux agissent trivialement, l'isogénie en question étant alors l'endomorphisme qui engendre l'idéal. On obtient donc l'action annoncée
$$\Cl(\O) \circlearrowright \Ell_\C(\O).$$

De plus, cette dernière action est simplement transitive. Les preuves données dans \cite{Sil2} utilisent de manière essentielle la représentation des courbes elliptiques sur $\C$ sous la forme $\C/\Lambda$, et tout le jeu consiste à dire que $\Lambda$ est déjà essentiellement un idéal fractionnaire de $\O$. Cette méthode n'est donc pas transposable telle quelle en caractéristique positive. La question est donc d'étendre ces définitions et résultats pour un corps quelconque, ou au moins un corps fini.

\subsection{Théorèmes de relèvement}

Une première idée pour atteindre les corps finis est de quotienter un anneau dans un corps de nombres par un idéal. Partant d'une courbe elliptique définie sur $\F_p$, on se demande par exemple si l'on peut considérer un relèvement à $\Z$, sur lesquels les résultats sont connus vu que $\Z\subset\C$, et redescendre tout cela ensuite. 

Le problème inverse est de partir d'une courbe elliptique définie sur $\Q$ et étudier ses réductions modulo $p$. Si l'on se donne par exemple une courbe elliptique $E$ définie sur $\Q$, alors $E$ admet un modèle à coefficients dans $\Z$. Pour réduire $E$ en caractéristique $p$, on cherche d'abord à éliminer toutes les puissances possibles de $p$ pour trouver un \emph{modèle minimal} en $p$. Deux cas se présentent alors : soit la courbe réduite modulo $p$ est lisse (c'est alors une courbe elliptique et on dit que $E$ a \emph{bonne réduction} en $p$), soit elle ne l'est pas (elle dégénère en un n\oe ud ou une pointe et on dit que $E$ a \emph{mauvaise réduction}). Étudier la réduction de courbes définies sur $\Q$, ou plus généralement sur un corps de nombres est une question d'arithmétique vaste.

\v

Cette question de relèvement est ancienne, et on dispose des théorèmes suivants, dus à Deuring et dont une preuve figure dans \cite{Lang}. On se donne un corps fini $k$ de caractéristique $p$, et on note $x\mapsto\bar{x}$ les réductions modulo $p$. On fixe également un ordre quadratique $\O$, que l'on voit dans $\C$. Si $\O$ est l'anneau d'endomorphismes d'une certaine courbe définie sur $k$, alors on a une égalité d'idéaux
$$(p) = \frak{p}_1 \frak{p}_2$$
dans $\O$. Si $E$ a multiplication complexe par $\O$, il existe exactement deux isomorphismes entre $\End(E)$ et $\O$, l'un envoyant le Frobenius de $E$ dans $\frak p_1$, l'autre dans $\frak p_2$. On convient que l'on considère le Frobenius comme un élément de $\frak p_1$: c'est d'une certaine façon l'analogue de l'isomorphisme canonique de la section précédente. Le point important est que cette convention est compatible avec celle de la section précédente, dans le sens où le diagramme
$$
\shorthandoff{;:!?}
\xymatrix @!=8mm {
\O \ar[rr]^\sim \ar[rd]^\sim & & \End(E^0) \ar[ld]^{\mathrm{mod}\,\frak P} \\
 & \End(E) & 
}
$$
est commutatif, dès que la place $\frak P$ vit au-dessus de $\frak p_1$. On peut toujours être dans cette situation, car $\frak p_1$ et $\frak p_2$ sont simplement conjugués l'un de l'autre.

%Noter la réduction autrement qu'avec des barres ?

\begin{thm}[Relèvement d'un endomorphisme en caractéristique zéro]

Soit $E/k$ une courbe elliptique et $\alpha$ un endomorphisme de $E$. Alors il existe un corps de nombres galoisien $L$, une courbe elliptique $E^0/L$, un endomorphisme $\alpha^0$ de $E^0$ et un premier $\frak p$ de $L$ au-dessus de $p$, tels que $E^0$ a bonne réduction en $\frak p$, $E$ est isomorphe à $\bar{E^0}$ et $\alpha$ correspond à $\bar{\alpha^0}$ sous cet isomorphisme.

\end{thm}

En particulier, si $\End(E)$ est isomorphe à un ordre $\O$, on peut choisir pour $\alpha$ un générateur de $\End(E)$ et on obtient une surjection $\End(E^0)\vers \End(E).$ Le résultat suivant montre en fait que dans ce cas, on obtient une bijection entre les anneaux d'endomorphismes.

\begin{thm}[Bonne réduction]

Soit $L$ un corps de nombres, $E/L$ une courbe elliptique telle que $\End(E)\simeq \O$ est un ordre dans un corps quadratique imaginaire $K$. Soit $p$ un nombre premier et $\frak p$ un premier de $L$ au-dessus de $p$, en lequel $E$ a bonne réduction. Alors $\bar{E}$ est ordinaire si et seulement si $p$ est totalement scindé dans $K$. Dans ce cas, si $c=p^r c_0$ est le conducteur de $\O$, avec $c_0 \wedge p = 1$, on a :

\begin{itemize}
\item[(i)] $\End(\bar{E})=\Z+c_0 \O_K$ est l'ordre de $K$ de conducteur $c_0$.
\item[(ii)] Si $c=c_0$, alors la réduction donne un isomorphisme $\End(E)\vers\End(\bar{E})$.
\end{itemize}
De plus, $\End(E)\vers\End(\bar{E})$ préserve le degré.

\end{thm}

Utilisons maintenant ce résultat pour adapter la section précédente au cas des courbes elliptiques ordinaires sur un corps fini. On peut définir comme précédemment $\frak a\cdot E$ pour $\frak a$ un idéal inversible de $\O$ et $E\in \Ell_k(\O)$, comme l'image de l'isogénie de noyau $E[\frak a]$, à l'aide de la propriété suivante.

\begin{lem}[Quotient, cas étale] Soit $S$ un sous-groupe fini de $E(\bar{k})$. Il existe une courbe elliptique $E'$ définie sur $\bar{k}$ et une isogénie séparable $\phi_S\de E\vers E'$ de noyau $S$. Cette isogénie est unique à isomorphisme près. Si $S$ est défini sur $k$, c'est à dire globalement stable par l'action du groupe de Galois, alors $\phi_S$ et $E'$ peuvent également être définies sur $k$.
\end{lem} 

Ce lemme est en fait vrai dans un cadre plus général qui permet de prendre en compte le cas des isogénies inséparables, où $S$ est un sous-schéma en groupes fini et plat de $E$. Ici, la courbe $\frak a\cdot E$ reste donc définie sur $k$, puisque $E[\frak a]$ l'est. On peut donner deux stratégies pour obtenir ce lemme: l'une très concrète, en partant d'une équation de la courbe et du noyau et en construisant l'équation de la courbe image, et l'autre plus abstraite, en construisant la courbe image comme un quotient dans le monde des schémas. L'une et l'autre sont détaillées plus loin dans ce document.


\begin{prop}

Cette opération définit une action de $\Cl(\O)$ sur $\Ell_k(\O)$ qui est simplement transitive.

\end{prop}

\begin{proof}

Soit $E$ une courbe elliptique définie sur $k$, et $E^0$ un relèvement de $E$ en caractéristique zéro donné par les théorèmes de Deuring. Soit également $\frak a$ un idéal de $\O$. Comme $E^0$ a multiplication complexe par $\O$, l'idéal $\frak a$ agit sur $E^0$:
$$E^0 \overset{\phi_{\frak a}^0}{\vers} E'^0.$$
On montre alors que $E'^0$ et $\phi_{\frak a}^0$ sont définis sur une extension finie de $\Q$. En effet, si $\sigma$ est un automorphisme du corps $\C$, alors $\sigma(E'^0)$ ont également multiplication complexe par $\O$, et ces courbes sont en nombre fini à isomorphisme près par la section précédente (il y en a exactement le nombre de classes, $h_\O$). Ainsi $j(E'^0)$ vit dans une extension finie de $\Q$, ce qui montre que $E'^0$ peut être définie sur une extension finie de $\Q$. Un argument similaire fonctionne pour $\phi_{\frak a}^0$: ses images sous $\Gal(\C/\Q)$ sont des morphismes entre $E^0$ et $E'^0$ de même degré, donc il y a seulement un nombre fini de telles images et $\phi_{\frak a}^0$ est aussi définie sur une extension finie de $\Q$.

Soit $L$ l'extension de $\Q$ et $\frak p$ la place de $L$ donnés par le théorème de relèvement. Notons $M$ une extension de $L$ sur laquelle $E^0$, $E'^0$ et $\phi_{\frak a}^0$ sont définis, et soit $\frak P$ une place de $M$ au-dessus de $\frak p$. Le quotient $M/\frak P$ est donc une extension finie de $k = L/\frak p$. Comme on sait que $E^0$ a bonne réduction en $\frak p$, on peut tout réduire modulo $\frak P$, et on obtient
$$E \overset{\phi}{\vers} E'$$
où $\phi$ est définie sur une extension finie de $k$. Le degré de $\phi$ est égal à celui de $\phi_{\frak a}^0$, dont le noyau est $E^0[\frak a]$, et le noyau de $\phi$ contient les réductions à $k$ des éléments de $E^0[\frak a]$. 

Nous admettons maintenant un ingrédient essentiel, à savoir le fait suivant. Pour tout idéal $\frak a$ inversible de $\O$ de norme première à $p$, et toute courbe $E\in \Ell_k(\O)$, on peut choisir un relèvement $E^0$ vérifiant les conditions suivantes:
\begin{itemize}
\item[•] Les éléments de $E[\alpha]$ sont entiers en $\frak p$
\item[•] La réduction modulo $\frak p$ donne un isomorphisme $E^0[\frak a] \overset{\sim}{\to} E[\frak a]$.
\end{itemize}

Avec ce fait admis, on montre donc que le noyau de $\phi$ est précisément $E[\frak a]$. Autrement dit, $\phi$ est, à isomorphisme près, l'isogénie $\phi_{\frak a}$, et on a $E' = \frak a\cdot E$. La courbe $\frak a\cdot E$ est donc la réduction de $\frak a\cdot E^0$. D'autre part, on peut utiliser le théorème de bonne réduction pour voir que $\frak a\cdot E$ a, comme $E'^0$, multiplication complexe par l'ordre $\O$. On en déduit immédiatement que l'on a une action
$$\Cl(\O) \circlearrowright \Ell_k(\O).$$

Montrons qu'elle est transitive : si $E$ et $E'$ sont données en caractéristique $p$, choisissons deux relèvements $E^0, E'^0$ en caractéristique 0. Il existe alors un idéal $\frak a$ envoyant $E^0$ sur $E'^0$, par transitivité dans $\C$. En réduisant on voit que $\phi_{\frak a}$ a bien pour image $E'$.

Enfin, étant donné $E$, montrons que son stabilisateur est l'ensemble des idéaux principaux. Si $\frak a$ est principal, il laisse un relèvement $E^0$ invariant, et donc $E$ également. Réciproquement, si $\frak a$ laisse $E$ invariante, alors $\phi_{\frak a}$ admet un relèvement $\psi\in \End(E^0)$. Comme $\psi$ et $\phi_{\frak a}^0$ ont même degré, on voit que ces isogénies ont même noyau, ce qui montre $\psi=\phi_{\frak a}^0$ (modulo un isomorphisme) et $\frak a\cdot E^0 = E^0$. Ainsi $\frak a$ est principal selon la section précédente.
\end{proof}
\v

Cette preuve \og trouée\fg\ semble raisonnablement directe, mais nous ne sommes pas parvenus à la compléter, ni à en trouver une complétion dans la littérature. On peut simplement remarquer que le fait admis est vrai lorsque $\frak a$ est l'idéal engendré par un entier $n$ premier à $p$, ce qui est plutôt inutile pour notre preuve, car il s'agit en particulier d'un idéal principal. Une autre façon d'adapter aux corps finis les résultats obtenus sur $\C$ est d'utiliser les modules de Tate, qui sont des espaces naturels attachés aux courbes elliptiques sur lesquels $\End(E)$ agit de manière agréable.


\subsection{Modules de Tate}


La référence principale pour cette section est \cite{Waterhouse}. Soit $k$ un corps quelconque. Donnons la définition des modules de Tate pour une courbe elliptique $E/k$. Pour tout entier $n$, on désigne par $E[n]$ le noyau de l'endomorphisme $[n]$. Si $\ell$ est un nombre premier, on a des applications
$$ [\ell] \de E[\ell^{m+1}](\bar{k})\vers E[\ell^{m}](\bar{k})$$
et on définit le \emph{module de Tate}
$$T_\ell(E) = \lim_{\leftarrow} E[\ell^m](\bar{k}).$$
On a donc une identification canonique $E[\ell^m](\bar{k})\simeq T_\ell(E)/ \ell^{m} T_\ell(E)$. Cette définition a un sens dans un cadre plus général, celui des variétés abéliennes.
\v

Si $\ell$ est différent de la caractéristique de $k$, on sait que $E[\ell]$ est un schéma en groupes fini \emph{étale} de rang $\ell^2$. Autrement dit, les $\bar{k}$-points de ce noyau munis de l'action de Galois contiennent toute l'information, et sont au nombre de $\ell^2$, le degré du morphisme correspondant. Dans ce cas, $T_\ell(E)$ est un $\Z_\ell$-module libre de rang 2. On peut donc le voir comme un réseau dans le $\Q_\ell$-espace vectoriel de dimension 2
$$V_\ell(E) = T_\ell(E) \otimes_{\Z_\ell} \Q_\ell.$$

Le diagramme suivant est alors commutatif :
$$
\shorthandoff{;:!?}
\xymatrix {
\ell^{-m} T_\ell(E)/ T_\ell(E) \ar[r]_{\ell^m}^{\sim} \ar@{^{(}->}[dd]&
 T_\ell(E)/ l^m T_\ell(E) \ar[r]^{\sim}  & 
 E[\ell^m](\bar{k}) \ar@{^{(}->}[dd] \\ 
 \\
 \ell^{-m-1} T_\ell(E)/ T_\ell(E) \ar[r]_{\ell^{m+1}}^{\sim} &
 T_\ell(E)/ \ell^{m+1} T_\ell(E) \ar[r]^{\sim}  & 
 E[\ell^{m+1}](\bar{k})
}
$$
ce qui donne une identification canonique avec les points $\ell$-primaires de la courbe,
$$ V_\ell(E)/ T_\ell(E) \simeq E(\ell)(\bar{k}).$$

Avec cette identification, les sous-groupes finis de $\ell^m$-torsion de $E(\bar{k})$ pour un certain $m$ correspondent aux réseaux de $V_\ell(E)$ contenant $T_\ell(E)$. Tous ces objets sont munis de plus d'une action naturelle du groupe de Galois $G= \mathrm{Gal}(\bar{k}/k)$.

Si $\phi\de E\vers E'$ est une isogénie définie sur $k$, ses restrictions $E[\ell^m](\bar{k})\vers E'[\ell^m](\bar{k})$ sont compatibles, et $\phi$ induit donc naturellement une application
$$ \phi_\ell\de T_\ell(E)\vers T_\ell(E').$$

Cette application est un morphisme de $\Z_\ell[G]$-modules, que l'on peut également considérer de $V_\ell(E)$ dans $V_\ell(E')$. Avec ces notations, on a le théorème suivant, dû à Tate :

\begin{thm}[Équivalence de Tate] Si $k$ est un corps fini, cette application
$$\Hom_k(E, E') \vers \Hom_{\Z_\ell[G]} (T_\ell(E), T_\ell(E'))$$
est un isomorphisme pour tout $\ell$ distinct de la caractéristique de $k$.
\end{thm}

Un résultat similaire pour les corps de nombres est dû à Faltings.
\v

Dans le cas $\ell=p$, tout s'effondre. L'endomorphisme $[p]$ de $E$ n'est pas séparable: l'égalité
$[p] = \pi \hat{\pi}$
montre que le Frobenius de $E$ est un facteur de $[p]$. Par conséquent, les $\bar{k}$-points munis de l'action de Galois ne contiennent plus toute l'information, et ne sont plus au nombre de $p^2$. On peut toujours construire le module de Tate $T_p(E)$, et celui-ci vérifie toujours des propriétés fonctorielles. En revanche l'équivalence de Tate ne tient plus: par exemple, si $E$ est supersingulière, le module de Tate $T_p(E)$ est trivial.

Fort heureusement, on peut construire un autre $\Z_p[G]$-module fonctoriel qui vérifie l'équivalence de Tate, appelé \emph{module de Dieudonné}. Sa construction est plus complexe, et l'on renvoie à \cite{Waterhouse} pour une discussion sur ces modules. On se contentera ici du cas $\ell\neq p$.
\v

On suppose maintenant que $k$ est un corps fini. L'équivalence de Tate permet de faire le lien entre les isogénies et les réseaux dans les modules de Tate, et par là à l'action de l'anneau d'endomorphismes. On peut alors démontrer la simple transitivité de l'action de $\Cl(\O)$ sur $\Ell_k(\O)$. On garde la même convention pour les isomorphismes $\O\vers \End(E)$ pour $E\in\Ell_k(\O)$.

\begin{prop}
Si $\ell\neq p$ est un nombre premier, $\frak a$ un idéal de $\O$, et $\phi\de E\vers E/E[\frak a] = E'$ l'isogénie quotient, alors on a l'égalité
$$\phi_\ell^{-1} T_\ell(E') = \bigcap_{\rho\in\frak a}\ \rho_\ell^{-1} T_\ell(E) .$$
\end{prop}

Cela permet de décrire la \emph{localisation} $\frak a\otimes \Z_\ell$ de l'idéal $\frak a$ en $\ell$ en termes de modules de Tate. On invoque alors le résultat classique qu'un réseau est déterminé par ses localisations : la lettre $\mathcal{R}$ désignant l'ensemble des réseaux, le lemme de Weil--Eischle déclare que l'application
$$\begin{aligned}
\mathcal{R}(\Q^n) &\longrightarrow \prod_{\ell\in \mathcal{P}} \mathcal{R}(\Q_\ell^n) \\
\Lambda\ \ &\longmapsto (\Lambda_\ell = \Lambda\otimes_\Z \Z_\ell)_{\ell\in\mathcal{P}}
\end{aligned}$$
est une injection d'image l'ensemble des collections $(\Lambda_\ell)$ telles que $\Lambda_\ell = \Z_\ell$ pour presque tout $\ell$. Bien sûr, pour utiliser ce résultat, il faut regarder toutes les localisations (y compris en $p$), car la conclusion ne tient pas même si l'on n'ôte qu'un seul premier.

Nommons \emph{idéal de noyau} un idéal $\frak a$ de $\O$ tel que les endomorphismes s'annulant sur $\bigcap_{\rho\in \frak a} \Ker\rho$ sont des éléments de $\frak a$.

\begin{prop}
Soit $k$ un corps fini, et $\O$ un ordre quadratique qui est l'anneau d'endomorphismes d'une certaine courbe ordinaire.
\begin{itemize}
\item Tout idéal de $\O$ est un idéal de noyau pour toute $E\in \Ell_k(\O)$.
\item L'action précédente est bien définie et simplement transitive.
\end{itemize}
\end{prop}
Une preuve complète de ce résultat figure dans \cite{Waterhouse}.





\newpage

\section{Courbes modulaires}

Une fois que l'on sait que l'action du groupe de classes est simplement transitive, on cherche naturellement à la calculer. Par exemple, pour calculer l'action d'un idéal de norme $\ell$, un nombre premier, il s'agit de déterminer une isogénie de degré $\ell$ au départ d'une courbe donnée.

On en arrive alors rapidement à la notion de courbe modulaire. Ces courbes paramétrisent la donnée d'une courbe elliptique munie d'une certaine structure: on dit qu'elles représentent un \emph{problème modulaire}. L'objectif de cette section est de montrer l'existence de ces courbes (ou au moins certaines d'entre elles) en toute généralité, c'est à dire en tant que schéma, et d'en calculer des équations. La référence principale est ici \cite{KaMa}. Comme précédemment, l'étude du cas complexe permet de motiver et d'introduire l'étude générale.


\subsection{Courbes modulaires sur $\C$}

Cherchons par exemple un espace classifiant les courbes elliptiques sur $\C$ munies d'un point de $N$-torsion ($N\geq 2$) à isomorphisme près. L'usage est de désigner ces objets par la lettre $Y$ complétée de quelques indices. Comme on l'a déjà mentionné, l'espace des courbes elliptiques complexe à isomorphisme près est le demi-plan de Poincaré quotienté par l'action du groupe $\Gamma(1) = \mathrm{SL}_2(\Z)$, ce que l'on note:
$$Y(1)_\C = \Gamma(1) \b \H.$$
La fonction modulaire $j$ donne un isomorphisme de cette surface de Riemann vers $\C$.

Soit $\tau\in \H$. La courbe elliptique $E_\tau = \C/(\Z+\Z\tau)$ a un point de $N$-torsion primitif naturel qui est simplement $\frac{1}{N}$. Cette donnée supplémentaire n'est pas invariante par tous les éléments de $\Gamma(1)$, seulement par ceux du sous-groupe
$$\Gamma_1(N) = \left\{\left(
\begin{matrix}
a & b \\
c & d
\end{matrix}
\right) \in \mathrm{SL}_2(\Z)\ :\ a = d = 1 \mod{N},\ c = 0\mod{N}\right\}.$$
On montre alors qu'en effet, une courbe elliptique complexe munie d'un point de $N$-torsion est isomorphe à un unique couple $\left(E_\tau, \frac{1}{N}\right)$ avec $\tau\in \Gamma_1(N) \b \H$. On note cela
$$Y_1(N)_\C = \Gamma_1(N) \b \H.$$

Qu'en est-t-il si l'on relâche la condition d'un point primitif de $N$-torsion, pour simplement demander un sous-groupe cyclique d'ordre $N$ ? Là encore, si $\tau\in \H$, $E_\tau$ a un groupe cyclique de cardinal $N$ naturel qui est simplement $\left\{\frac{a}{N},\ 0\leq a\leq N-1\right\}$. Cette donnée supplémentaire est invariante par un groupe plus grand que $\Gamma_1(N)$, à savoir
$$\Gamma_0(N) = \left\{\left(
\begin{matrix}
a & b \\
c & d
\end{matrix}
\right) \in \mathrm{SL}_2(\Z)\ :\ c = 0\mod{N}\right\}.$$
De même que précédemment, étant donné une courbe elliptique et un sous-groupe cyclique de cardinal $N$, il existe une unique $E_\tau$ munie de cette donnée supplémentaire qui lui soit isomorphe modulo $\Gamma_0(N)$, autrement dit
$$Y_0(N)_\C = \Gamma_0(N) \b \H.$$
Remarquons que ce problème modulaire est aussi celui de l'isogénie cyclique de degré $N$: la donnée d'un sous-groupe cyclique de $E$ de cardinal $N$ est équivalente à celle d'une isogénie cyclique partant de $E$ de degré $N$. Le terme \emph{cyclique} signifie ici que le noyau est cyclique.
\v

Prenons un dernier exemple : la donnée d'un couple de points de $E$ qui engendrent $E[N] \simeq (\Z/N\Z)^2$. Ici la présence d'un invariant discret, l'accouplement de Weil de ces deux points, interdit que l'on trouve une courbe connexe qui classifie une telle structure à isomorphisme près. L'accouplement de Weil est défini par exemple dans \cite{Sil1}: c'est celui que l'on utilise pour montrer l'autodualité des schémas en groupes $E[N]$ pour la dualité de Poincaré.

Pout tout $\tau\in \H$, $E_\tau$ a deux points distingués qui engendrent $E_\tau[N]$, à savoir $\frac{\tau}{N}$ et $\frac{1}{N}$. Cette donnée est laissée invariante par le groupe
$$\begin{aligned}
\Gamma(N) &= \left\{\left(
\begin{matrix}
a & b \\
c & d
\end{matrix}
\right) \in \mathrm{SL}_2(\Z)\ :\ a = d = 1 \mod{N},\ b = c = 0\mod{N}\right\} \\
 &= \Ker(\mathrm{SL}_2(\Z) \to \mathrm{SL}_2(\Z/N\Z)).
\end{aligned}$$
On peut vérifier que l'accouplement de Weil prend la valeur $\zeta_N = e^\frac{2 i \pi}{N}$ sur ces deux points, et que l'on obtient bien la courbe de classification cherchée (en ajoutant la condition sur l'accouplement):
$$Y(N)_\C = \Gamma(N) \b \H.$$
%On pourrait s'amuser à trouver d'autres sous-groupes pour d'autres structures (par exemple la donnée simultanée d'un point de $N$-torsion et d'un sous-groupe cyclique d'ordre $M$), mais ces trois exemples suffiront pour nous guider.

Comme dans le cas de $Y(1)$, on peut compactifier les courbes $\Gamma \b\H$ pour tout sous-groupe de congruence $\Gamma$ de $\Gamma(1)$ en y ajoutant un nombre fini de points de $\P^1(\Q)$, appelés \emph{pointes}. On obtient alors des surfaces de Riemann compactes, qui sont donc des courbes algébriques, et que l'on notera avec la lettre $X$. Pour étudier ces courbes, une direction naturelle est de se demander quelles sont leurs fonctions et formes différentielles, ne serait-ce que pour en trouver une équation. Bien sûr, ce sont simplement des fonctions modulaires de différents poids pour $\Gamma$, et l'on voit un premier lien entre les courbes elliptiques et les formes modulaires.

\subsection{Calcul d'équations pour les courbes modulaires complexes}


Le but étant d'utiliser une courbe modulaire pour calculer des isogénies entre courbes elliptiques sur un corps fini, il est important de disposer d'équations pour ce genre de courbes. Mis à part le cas de $X(1)$, qui est isomorphe à la droite projective par le $j$-invariant, il est difficile de déterminer de telles équations à la main.

On aimerait travailler avec des équations planes, quitte à accepter des équations définissant des courbes singulières lorsque l'on est capable de contrôler les points singuliers. Une application birationnelle vers une courbe plane (projective) étant simplement la donnée de deux fonctions sur la courbe qui engendrent son corps de fonctions, on recherche tout d'abord ces dernières (par exemple sous forme de $q$-expansion) puis une équation polynomiale les reliant. La variable $q$ est définie par $q = e^{2i\pi\tau}$ lorsque $\tau$ parcourt $\H$, et la plupart des fonctions modulaires admettent une expression sous forme de série de Laurent en $q$.
\v

\textbf{Le cas $Y_0(N)$.} Ce cas est particulier, d'une part car il s'agit de la principale courbe modulaire utilisée dans ce document, et d'autre part car on peut en trouver une équation \og sans calcul \fg. On a vu que l'on peut reformuler le problème modulaire associé à $Y_0$ sous la forme du problème de $N$-isogénie cyclique $E\to E'$. On peut alors définir une application informelle
$$(E\to E') \longmapsto (E, E'),$$
ce qui suggère que l'application
$$\begin{aligned}
Y_0(N)_\C &\vers \C^2 \\
 \tau &\longmapsto (j(\tau), j(N\tau))
\end{aligned}$$
est bien définie, ce que l'on peut vérifier à la main.

Les fonctions modulaires $j(\tau)$ et $j(N\tau)$ sont simples, et on peut trouver une équation avec un papier et un crayon de la manière suivante. L'idée est que $\tau\mapsto N\tau$ est l'action d'une certaine matrice, et que l'on veut symétriser cela pour obtenir un polynôme à coefficients dans $\Z$. On définit
$$C(N)=\left\{ 
\left(
\begin{matrix}
a & b \\
0 & d 
\end{matrix}
\right)
\in \M_2(\Z)\ :\ ad=N,\ a>0,\ 0\leq b<d,\ \mathrm{pgcd}(a,b,d)=1\right\}.$$
et les $(\sigma^{-1}\Gamma(1)\sigma)\cap \Gamma(1)$ pour $\sigma\in C(N)$ sont exactement les classes à droite de $\Gamma_0(N)$ dans $\Gamma(1)$.

On montre alors qu'il existe un unique  polynôme $\Phi_N \in \Z[X,Y]$, appelé $N$-ième \emph{polynôme modulaire}, tel que
$$\forall \tau\in\H,\ \Phi_N(X,j(\tau))=\prod_{\sigma\in C(N)} (X-j(\sigma\tau)).$$
Ce polynôme est de plus symétrique en $X, Y$ et irréductible dans $\Z[X, Y]$. Ainsi l'image de l'application $Y_0(N)_\C\vers \C^2$ est exactement le lieu des zéros de ce polynôme $\Phi_N$. Les propriétés de $\Phi_N$ sont démontrées par exemple dans \cite{Sil2}.

Cependant, on ne peut pas vraiment dire que $\Phi_N=0$ définit une équation de la courbe modulaire $Y_0(N)_\C$, car le morphisme $Y_0(N)\vers \C^2$ n'est pas injectif. En effet, il existe des couples de réseaux $(\Lambda, \Lambda')$ tels qu'il existe plusieurs inclusions $\Lambda\vers\Lambda'$ de conoyau $\Z/N\Z$. Ces points singuliers comprennent notamment des points de $Y(1)$ ayant multiplication complexe par $\Z[i]$ ou $\Z[e^{\frac{2i\pi}{3}}]$, ce qui implique l'existence d'automorphismes non triviaux, mais ce ne sont pas les seuls. Pour contrôler ces points singuliers, on peut faire la remarque suivante. Si $\Lambda, \Lambda'$ sont comme ci-dessus, alors on dispose de deux isogénies cycliques de degré $N$:
$$\phi_1, \phi_2\de \C/\Lambda \vers \C/\Lambda'.$$
La composée $\widehat{\phi_1} \phi_2$ est un endomorphisme de la courbe elliptique $\C/\Lambda$ qui n'est pas scalaire, et est de degré $N^2$. Cela montre que $\C/\Lambda$ (et également $\C/\Lambda'$) sont des courbes à multiplication complexe, et dont le discriminant est borné par $N^2$. Déterminer plus précisément les points doubles de l'équation $\Phi_N = 0$ semble moins évident, mais cette borne grossière suffira dans ce mémoire.

Ainsi la courbe $\Phi_N=0$ détermine une courbe plane singulière qui, sur $\C$, est birationnelle à $Y_0(N)$. En dehors des points doubles, on dispose donc tout de même d'une équation qui permet, étant donnée une courbe, de déterminer les courbes que l'on peut atteindre par une isogénie cyclique de degré $N$.

Pour se débarrasser des points singuliers, on pourrait \emph{rigidifier} la situation pour tuer ces automorphismes, en rajoutant une structure de niveau $M$, avec $M$ grand; on aurait alors une application
$$Y_0(N)(M)_\C\vers Y(M)_\C^2$$
qui est une immersion fermée. Cela ne donnera cependant pas d'équation plane de cette nouvelle courbe, car $Y(M)_\C^2$ n'est pas $\C^2$.


\textbf{Calcul du polynôme modulaire.} D'une certaine manière, la définition précédente du polynôme modulaire est inutile, car elle ne donne pas de moyen direct de le calculer (on apprend tout de même son degré, qui est $N+1$, et qu'il est symétrique à coefficients dans $\Z$). Une méthode consiste à utiliser les $q$-expansions, avec $q = e^{2i\pi\tau}$, des fonctions modulaires $j(\tau)$ et $j(N\tau)$, que l'on obtient à partir de celles des fonctions de Weierstrass:
$$\begin{aligned}
j(\tau) &= \frac{1}{q} + 744 + 196884 q + 21493760 q^2 + 864299970 q^3 + \ldots\\
j(N\tau) &= \frac{1}{q^N} + 744 + 196884 q^N + 21493760 q^{2N} + 864299970 q^{3N} + \ldots
\end{aligned}$$
On peut calculer les premières puissances de ces fonctions modulaires, puis rechercher une équation polynomiale (symétrique) les reliant en faisant de l'algèbre linéaire. On trouve
$$\begin{aligned}
\Phi_2(X, Y) &=  X^3 + Y^3 - X^2 Y^2 + 1488(X^2Y + X Y^2)  - 162000(X^2 + Y^2) \\
&\quad + 40773375 XY  + 8748000000(X + Y) - 157464000000000\\
\Phi_3(X, Y) &= X^4 + Y^4 - X^3 Y^3 + 2232(X^3 Y^2 + X^2 Y^3) - 1069956(X^3 Y + X Y^3)\\
&\quad + 2587918086 X^2 Y^2  + 36864000(X^3 + Y^3) + 8900222976000(X^2 Y + X Y^2)  \\
&\quad + 452984832000000(X^2 + Y^2) - 770845966336000000 XY \\
&\quad + 1855425871872000000000(X + Y)\\
\Phi_5(X, Y) &= X^6 + Y^6 -X^5 Y^5 + 3720(X^5 Y^4 + X^4 Y^5) - 4550940(X^5 Y^3 + X^3 Y^5) \\
&\quad + 1665999364600 X^4 Y^4 + 2028551200(X^5 Y^2 + X^2 Y^5) \\
&\quad + 107878928185336800(X^4 Y^3 + X^3 Y^4) - 246683410950(X^5 Y + X Y^5) \\
&\quad + 383083609779811215375(X^4 Y^2 + X^2 Y^4) - 441206965512914835246100 X^3 Y^3 \\
&\quad + 1963211489280(X^5 + Y^5) + 128541798906828816384000(X^4 Y + X Y^4) \\
&\quad + 26898488858380731577417728000(X^3 Y^2 + X^2 Y^3) \\
&\quad +  1284733132841424456253440(X^4 + Y^4) \\
&\quad - 192457934618928299655108231168000(X^3 Y + X Y^3) \\
&\quad + 5110941777552418083110765199360000 X^2 Y^2 \\
&\quad + 280244777828439527804321565297868800(X^3 + Y^3) \\
&\quad + 36554736583949629295706472332656640000(X^2 Y + X Y^2) \\
&\quad + 6692500042627997708487149415015068467200( X^2 + Y^2) \\
&\quad - 264073457076620596259715790247978782949376 XY \\
&\quad + 53274330803424425450420160273356509151232000(X + Y) \\
&\quad + 141359947154721358697753474691071362751004672000
\end{aligned}$$
et l'on s'arrête ici de peur de rapidement doubler la longueur de ce mémoire: le stockage du polynôme modulaire $\Phi_N$ en mémoire a un coût d'environ $\Theta(N^3)$. On peut en fait calculer ces polynômes plus efficacement, comme il est indiqué dans \cite{Elkies}.
Des bases de données de polynômes modulaires se trouvent dans le logiciel de calcul formel Sage \cite{Sage} (qui contient également une riche base de formes modulaires) jusqu'à environ $N = 100$, ou sur la page web d'Andrew Sutherland (\url{math.mit.edu/~drew}), par exemple.

En pratique, lorsque la taille de ces polynômes devient problématique, on peut utiliser d'autres fonctions modulaires qui donnent de meilleurs résultats que $j(\tau)$ et $j(N\tau)$, par exemple $j(\tau)$ et la fonction de Weber. On ne gagne asymptotiquement qu'un facteur constant, et les mêmes problèmes apparaissent à partir d'environ $N=500$.
\v

\textbf{Le cas général.} Cette méthode est adaptable directement au cas d'autres courbes modulaires, dès que l'on est capable d'exhiber des fonctions modulaires du bon niveau qui engendrent son corps des fonctions. On peut ensuite faire de l'algèbre linéaire sur les $q$-expansions si l'on ne connaît pas a priori d'équation polynomiale satisfaite par ces fonctions modulaires. On verra dans ce mémoire deux exemples de cette méthode, pour les courbes $X_0(30)$ et $X_1(N)$.

Nous avons travaillé jusqu'à présent sur $\C$, mais il est crucial de savoir que ces équations, en particulier celles de $X_0(N)$, restent valables sur un corps quelconque (en particulier sur les corps finis). Pour cela, il faut faire intervenir des notions plus avancées de géométrie algébrique.


\subsection{Le formalisme des problèmes modulaires}


Il est impossible de définir ici tout le vocabulaire de la théorie des schémas utilisé dans cette partie. Deux références pour cela: \cite{Hart} et surtout \cite{Stack}.


Soit $S$ un schéma. On appelle \emph{courbe elliptique} sur $S$ un $S$-schéma
$$\E \overset{\pi}{\vers} S$$
qui est une courbe propre et lisse, dont les fibres géométriques sont connexes de genre 1, munie d'une section notée $\O_\E\de S\vers \E$. Cette définition étend celle que l'on a considérée jusqu'à présent d'une courbe elliptique définie sur un corps.

Par \emph{courbe lisse}, on entend que $\pi$ est un morphisme lisse de dimension relative 1 qui est séparé et de présentation finie. La condition sur les fibres signifie que pour tout point géométrique $x$ de $S$ (c'est à dire tout morphisme $x\de\Spec k\vers S$ où $k$ est un corps algébriquement clos), la \emph{fibre} de $\E$ en $x$, c'est à dire le $k$-schéma
$$\E \times_S \Spec k\ \vers\ \Spec k$$
est connexe et de genre 1. C'est de plus une courbe propre et lisse (ces propriétés sont préservées par changement de base) munie de la section $0 \times x$, donc une courbe elliptique sur le corps $k$ au sens usuel. Ainsi, on peut voir une courbe elliptique relative $\E/S$ comme une famille de courbes elliptiques \og compatibles \fg\ sur des corps variables, les points géométriques de $S$.

On peut montrer que $\E$ admet une structure naturelle de $S$-schéma en groupes commutatif qui étend la structure de schéma en groupe définie sur les fibres géométriques : c'est le théorème d'Abel.
\v

On note $(\Ell)$ la catégorie dont les objets sont les courbes elliptiques relatives
$$\E \overset{\pi}{\vers} S$$
pour des schémas arbitraires $S$, et dont les morphismes sont les carrés commutatifs cartésiens
$$
\shorthandoff{;:!?}
\xymatrix @!=8mm {
\E' \ar[d]^{\pi'} \ar[r]  & \E \ar[d]^{\pi} \\
 S' \ar[r] & S
}
$$
c'est à dire ceux pour lesquels on a $\E' \simeq \E \times_S S'$.

On appelle \emph{problème modulaire} pour les courbes elliptiques un foncteur contravariant $\Pr$ de $(\Ell)$ vers la catégorie des ensembles. Un élément de $\Pr(\E/S)$ est appelé \emph{structure de niveau} $\Pr$ sur $\E/S$. Lorsque l'on se restreint à prendre $S$ dans les $R$-schémas et aux morphismes de $R$-schémas, on obtient la sous-catégorie $(\Ell/R)$. Un \emph{problème modulaire sur $R$} est un foncteur contravariant comme ci-dessus défini sur cette sous-catégorie.

\begin{ex} On peut définir les analogues des exemples étudiés dans le cas complexe.
\begin{enumerate}
\item Pour tout entier $N\geq 1$, on note $[Y(N)]$ le problème modulaire
$$ \E/S\ \longmapsto \left\{
\begin{matrix}
\text{Morphismes\ de\ schémas\ en\ groupes}\\
\phi\de (\Z/N\Z)^2_S\vers \E[N]\ \\
\text{générateurs\ de}\ \E[N]\ \text{et\ compatibles}\\
 \text{à\ l'accouplement\ de\ Weil}
\end{matrix}
\right\}$$
On laisse ici de côté la définition de \emph{générateur}, mais une conséquence en est la suivante: pour tout point géométrique $x\de\Spec k\vers S$, l'application
$$\phi \times x\de (\Z/N\Z)^2 \vers \E[N](k)$$
est un isomorphisme de groupes. Il s'agit bien d'un problème modulaire, car ces deux conditions sont clairement fonctorielles.

Dans le cas particulier $\Qr_3 = [Y(3)]$, lorsque 3 est inversible (i.e. en se restreignant à la sous-catégorie $(\Ell/\Z[1/3]))$, on peut donner une description universelle de ce problème à l'aide d'une équation de Weierstrass, à savoir la courbe
$$E\de y^2 + a_1 x y + a_3 y = x^3$$
où $a_1 = 3C - 1$ et $a_3 = -3 C^2 - B - 3 BC$, munie des deux points
$$P_3 = (0,0), \quad Q_3 = (C, B+C)$$
sur l'anneau
$$\frac{\Z[1/3, B, C][1/(a_1^3 - 27 a_3)a_3 C]}{B^3 = (B+C)^3}.$$
Cela signifie qu'un élément de $\Qr_3(\E)$ peut être identifié avec un morphisme de $S$-schémas de $\E$ vers $E\times S$.
\item On note $[Y_1(N)]$ le problème modulaire du point d'ordre exact $N$ : il s'agit du foncteur
$$ \E/S\ \longmapsto \left\{
\begin{matrix}
\text{Morphismes\ de\ schémas\ en\ groupes}\\
\phi\de (\Z/N\Z)_S\vers \E[N]\ \\
\text{qui\ sont\ une}\ \Z/N\Z \text{-structure\ sur}\ \E[N]
\end{matrix}
\right\}$$
Encore une fois, on laisse ici la définition de $\Z/N\Z$-structure de côté. Elle implique que pour tout point géométrique $x\de\Spec k\vers S$, l'application
$$\phi \times x\de \Z/N\Z \vers E[N](k)$$
est injective.
\item On note $[Y_0(N)]$ le problème modulaire du sous-groupe cyclique d'ordre $N$ :
$$ \E/S\ \longmapsto \left\{
\begin{matrix}
\text{Sous-groupes finis plats}\ K\subset E[N]\\
 \text{localement libres de rang}\ N\ \text{et cycliques}
\end{matrix}
\right\}$$
où \emph{cyclique} signifie que localement f.p.p.f, il admet un générateur dans le sens de ci-dessus. De manière équivalente, on peut voir ce problème comme celui de la $N$-isogénie cyclique, c'est à dire une isogénie $\E \overset{f}{\vers} \E'$ telle que Ker($f$) est cyclique d'ordre $N$.
\item Le problème modulaire de Legendre $\Qr_2$ classifie les courbes munies d'un couple de deux points de 2-torsion indépendants. Lorsque l'on se restreint aux schémas où 2 est inversible, on a une courbe universelle pour ce problème donnée par l'équation de Legendre
$$ E\de y^2 = x (x-1) (x-\lambda)$$
sur l'anneau $\Z[1/2,\lambda][1/\lambda(\lambda -1)].$
\end{enumerate}
\end{ex}

Remarquons que, sur les complexes, donc une fois débarassés du formalisme parfois un peu lourd de la théorie des schémas, ces problèmes modulaires sont exactement ceux que nous avons regardés dans la section précédente.

\subsection{Représentabilité et rigidité}

Comme précédemment où l'on classifiait des structures complexes par des points sur des surfaces, on s'intéresse à la représentabilité des problèmes modulaires. Par exemple, il serait agréable de trouver une courbe $Y_0(N)$ sur $\Spec\Z$ telle que pour tout schéma $S$, les $S$-points de $Y_0(N)$ soient en bijection avec les courbes elliptiques relatives $\E/S$ munies d'une $[Y_0(N)]$-structure à isomorphisme près. Dans l'idéal, on aimerait obtenir des courbes algébriques qui, sur $\C$, sont les surfaces que nous avons trouvées précédemment. Cependant, des obstructions existent.
\v

On dit qu'un problème modulaire $\Pr$ est \emph{relativement représentable} si pour toute courbe elliptique $\E/S$, le foncteur (contravariant) des $S$-schémas vers les ensembles
$$T \ \longmapsto\ \Pr(\E\times_S T / T)$$
est représentable par un $S$-schéma noté $\Pr_{\E/S}$. Cela signifie que l'on a un isomorphisme fonctoriel en $T$ :
$$\Pr(\E\times_S T / T) \simeq \Pr_{\E/S}(T) = \Hom_{S-\text{sch.}}(T, \Pr_{\E/S}).$$
Si $\Pr$ est relativement représentable, on dit qu'il vérifie une certaine propriété si tous les morphismes structurels $\Pr_{E/S}\vers S$ ont cette propriété (par exemple, être étale, fini, surjectif, affine). Pour résumer, un problème modulaire $\Pr$ est relativement représentable si pour toute courbe elliptique fixée, les structures de niveau $\Pr$ de cette courbe sur une base variable $T$ correspondent aux $T$-points d'un certain schéma.

On dit que $\Pr$ est \emph{représentable} s'il est représentable en tant que foncteur sur $(\Ell$), c'est à dire qu'il existe une courbe elliptique relative
$$E \overset{p}\vers \M(\Pr)$$
telle qu'on ait un isomorphisme fonctoriel dans $(\Ell)$ :
$$\Pr(\E/S) \simeq \Hom_{(\Ell)}(\E/S, E/\M(\Pr)).$$
Dans ce cas, on dispose d'un élément universel $a\in \Pr(E/\M(\Pr))$ induit par l'identité de $E/\M(\Pr)$. Par exemple, le problème modulaire $\Qr_3$ de niveau 3 sur $\Z[1/3]$ est représentable par $E/\M(\Qr_3)$, la courbe 
universelle donnée à l'exemple précédent. De même, le problème de Legendre $\Qr_2$ est représentable sur $\Z[1/2]$ par la courbe universelle donnée sous forme de Legendre. En dehors de ces exemples très simples, il est difficile d'exhiber des représentants donnés par des équations explicites : il faut recourir à des arguments plus abstraits.
\v

On remarque que si le problème $\Pr$ est représentable par $E\vers \M(\Pr)$, alors $\M(\Pr)$ représente le foncteur suivant dans la catégorie des schémas :
$$S\ \longmapsto\ \left.
\begin{cases}\ \text{Classes\ d'isomorphisme\ des\ paires}\ (\E/S,\alpha)\ \text{où}\ \\
 \ \E/S\ \text{est\ une\ courbe\ elliptique\ relative\ et}\ \alpha\in \Pr(\E/S)
\end{cases} \right\}.$$
En effet, si l'on désigne par $a$ la $\Pr$-structure canonique de $E/\M(\Pr)$, les applications
$$\begin{aligned}
(\E/S, \alpha)\ &\overset{\phi}{\longmapsto}  &(\alpha\de S\vers \M(\Pr)) \\
(E \times_{\M(\Pr)} S,\ \Pr(f)(a))\ &\overset{\psi}{\mathrel{\reflectbox{\ensuremath{\longmapsto}}}} &(f\de S\vers \M(\Pr))
\end{aligned}$$
sont des isomorphismes réciproques. Vérifions-le : par fonctorialité, l'application
$$\Pr(E/\M(\Pr)) \overset{\Pr(f)} \vers \Pr(E\times S/S)$$
utilisée dans la définition de $\psi$ est la précomposition par le changement de base par $f$ :
$$\Hom(E/\M(\Pr), E/\M(\Pr)) \vers \Hom(E\times S/S, E/\M(\Pr)).$$
Ainsi, comme $a$ est l'identité dans cette description, $\Pr(f)(a)$ est le changement de base par $f$, et ainsi on a bien $\phi\circ\psi = \id.$ Réciproquement, on se donne $(\E/S, \alpha)$ et on veut montrer que cette paire est isomorphe à
$$(E \times_{\M(\Pr)} S,\ \Pr(\alpha)(a)).$$
On a ici identifié $\alpha\in \Pr(\E/S)$ à une flèche $\E/S \vers E/\M(\Pr)$, car le foncteur $\Pr$ est représentable. D'un côté, on a un carré cartésien
$$
\shorthandoff{;:!?}
\xymatrix @!=8mm {
\E \ar[d]^{\pi} \ar[rr] & & E \ar[d]^{p} \\
 S \ar[rr]^{\alpha} & & \M(\Pr)
}
$$
donné par $\alpha$ qui est un morphisme de $(\Ell)$, et de l'autre, un carré cartésien
$$
\shorthandoff{;:!?}
\xymatrix @!=8mm {
E\times_{\M(\Pr)} S \ar[d]^{\pr_2} \ar[rr]^{\pr_1} & & E \ar[d]^{p} \\
 S \ar[rr]^{\alpha} & & \M(\Pr)
}
$$
Cela montre l'isomorphisme demandé par unicité du produit, et le fait que les structures de niveau $\Pr$ coïncident provient du fait que $\Pr(\alpha)(a)$ est bien le changement de base par $\alpha$.
\v

On peut également montrer qu'un foncteur représentable est toujours relativement représentable, et expliciter le schéma en question. De plus, si $\Pr$ est représentable et $\Pr'$ est relativement représentable, on montre que le problème modulaire \og simultané\fg\ $\Pr\times\Pr'$ est représentable par le schéma 
$$\M(\Pr, \Pr') = \Pr'_{E/\M(\Pr)}.$$

Un premier pas vers la représentabilité consiste à montrer la relative représentabilité.

\begin{lem}
Soit $N\geq 1$ un entier et $S$ un schéma dans lequel $N$ est inversible (c'est à dire un schéma sur $\Z[1/N]$). Soit $\E/S$ une courbe elliptique. Alors les foncteurs
$(S\text{-}\mathrm{sch})\vers(\mathrm{Ens})$
$$T \longmapsto
\begin{cases}
[Y(N)]   \\
[Y_1(N)] \qquad \text{-structures sur}\ \E \times_S T/T \\
[Y_0(N)]
\end{cases}$$
sont représentables par un $S$-schéma fini étale. Autrement dit, les problèmes modulaires $[Y(N)]$, $[Y_1(N)]$ et $[Y_0(N)]$ sur $\Z[1/N]$ sont relativement représentables, finis et étales.
\end{lem}

\begin{proof}
Comme $N$ est inversible, le schéma en groupes $\E[N]$ est étale, et localement isomorphe au schéma constant $(\Z/N\Z)^2$ pour la topologie étale. La relative représentativité des deux premiers foncteurs devient alors purement formelle. C'est un peu plus délicat pour le troisième : on renvoie à \cite{KaMa}, (3.7).
\end{proof}


On peut noter une première obstruction à la représentabilité de certains problèmes modulaires : on dit que $\Pr$ est \emph{rigide} si pour toute structure $\alpha$ de niveau $\Pr$ sur $\E/S$, la paire $(\E/S, \alpha)$ n'admet pas d'automorphismes non triviaux. On a alors le résultat suivant :

\begin{lem}
Tout problème modulaire représentable est rigide.
\end{lem}

\begin{proof}
Si $\Pr$ un problème modulaire représentable par $E\vers \M(\Pr)$, et si $\phi\in \mathrm{Aut}(\E/S)$ laisse la $\Pr$-structure $\alpha\in \Hom_{(\Ell)}(\E/S, E/\M)$ invariante, alors on a un diagramme commutatif :

$$
\shorthandoff{;:!?}
\xymatrix @!=2mm {
& & \E \ar[ddd]^{\pi} \ar[rrr]^{\alpha} & & & E \ar[ddd]^{p} \\
 \\
\E \ar[ddd]^{\pi} \ar[rrr]^{\alpha} \ar@{.>}[uurr]^\phi & & & E \ar[ddd]^{p} \ar@{.>}[uurr]^\id & & \\
& & S \ar[rrr]^{\alpha} & & & \M(\Pr) \\
 \\
 S \ar[rrr]^{\alpha} \ar@{.>}[uurr]^\id & & & \M(\Pr) \ar@{.>}[uurr]^\id & &
}
$$
d'où l'on déduit $\phi = \id$, car les carrés avant et arrière sont cartésiens.
\end{proof}

Un fait important est que cette obstruction est essentiellement la seule.

\begin{thm}
Tout problème modulaire affine relativement représentable et rigide est représentable.
\end{thm}

Pour montrer ceci, on cherche à construire un $\Z$-schéma $\M(\Pr)$ qui représente le foncteur $\Pr$. Pour cela, il suffit de montrer que $\Pr$ est représentable à la fois sur $\Z[1/2]$ et $\Z[1/3]$ ; la rigidité de $\P$ permettra ensuite de recoller les deux schémas obtenus en un schéma sur $\Z$, puisque leurs restrictions à $\Z[1/6]$ devront coïncider. Pour ces deux questions, on utilise le lemme suivant :

\begin{lem}
Soit $N\geq 1$ un entier, $G$ un groupe fini (concret), et $\Qr$ un problème modulaire affine et relativement représentable. On suppose de plus que $\Qr$ est représentable par un schéma affine sur $\Z[1/N]$ et que $G$ agit sur $\Qr$ de telle sorte que pour toute courbe elliptique
$$\E\vers S\vers \Spec \Z[1/N],$$
le $S$-schéma $\Qr_{\E/S}$ soit un $G$-torseur fini étale. Alors $\Pr$ est représenté par le $\Z[1/N]$-schéma affine
$$\M(\Pr,\Qr)/G.$$
\end{lem}

On l'applique ensuite d'une part au problème modulaire de Legendre $\Qr_2$ pour $N = 2$, et au problème $\Qr_3$ pour $N = 3$.

%La notion de torseur nécessite sans doute une définition.
%
%\begin{defi}
%Soit $X$ un $S$-schéma, et $G$ un schéma en groupes sur $S$. On dit que $X$ est un $G$-torseur fini étale si :
%\begin{itemize}
%\item $X$ est un $S$-schéma fini étale.
%\item On a une action de $G$, donc une flèche
%$$\begin{aligned}
%X \times_S G &\vers X \times_S X \\
%(x, g) &\longmapsto (x, g\cdot x)
%\end{aligned}$$
%\item Celle-ci soit un isomorphisme de $S$-schémas.
%\item $X$ est localement trivial (par exemple, pout la topologie fppf ou fpqc), c'est à dire qu'il existe localement une section de la %flèche $G\times X\vers X$.
%\end{itemize}
%On dit également que $\Qr_{\E/S}$ est un \emph{espace principal homogène sous $G$}. Dans notre situation, $G$ est un groupe concret %fini, donc on choisit simplement le schéma en groupe constant $G_S$ dans cette définition.
%\end{defi}

\begin{proof}[Démonstration du lemme.] On a dit que le problème simultané $(\Pr, \Qr)$ est représentable par
$$\M(\Pr, \Qr) = \Pr_{E/\M(\Qr)}.$$
Comme $\Pr$ est supposé affine, ce schéma est affine sur $\M(\Qr)$ qui est lui-même affine, donc est affine. De plus, le groupe $G$ agit sur $\M(\Pr, \Qr)$, car celui-ci représente le foncteur $\Pr\times\Qr$, et on peut faire agir fonctoriellement $G$ sur la composante $\Qr$.

\newcommand{\univ}{\mathrm{univ}}

Considérons alors la courbe elliptique universelle pour $\Pr\times\Qr$ :
$$E \vers \M(\Pr,\Qr)$$
munie de $(\alpha_{\univ}, \beta_{\univ}) \in (\Pr\times\Qr)(E/\M(\Pr,\Qr)).$ La preuve se déroule alors en trois temps :
\begin{itemize}
\item[•] Montrer l'existence du quotient $\M(\Pr,\Qr)/G$ ;
\item[•] Descendre $(E, \alpha_{\univ})$ à $\M(\Pr,\Qr)/G$, c'est à dire remplir le diagramme cartésien
$$
\shorthandoff{;:!?}
\xymatrix {
(E,\alpha_{\univ}) \ar[d] \ar@{.>}[r] &\ \textbf{?}\ \ar@{.>}[d] \\
 \M(\Pr,\Qr) \ar[r] & \M(\Pr,\Qr)/G
}
$$
\item[•] Montrer que la courbe descendue représente bien le problème modulaire $\Pr$ sur $\Z[1/N]$.
\end{itemize}

Pour le premier point, il suffit de montrer que le groupe fini $G$ agit librement sur le schéma affine $\M(\Pr,\Qr)$; alors on sait que le quotient existe par le théorème de Grothendieck, et que la projection
$$\M(\Pr,\Qr)\vers \M(\Pr,\Qr)/G$$
est un $G$-torseur fini étale (et surjectif, donc fidèlement plat). Pour cela, il suffit de montrer que l'action est \emph{universellement libre}. Soit $T$ un $\Z[1/N]$-schéma. On a vu que $\M(\Pr, \Qr)(T)$ est l'ensemble des courbes elliptiques sur $T$ munies d'une $(\Pr,\Qr)$-structure à isomorphisme près. Si un élément $g\in G$ laisse un élément invariant, cela signifie que l'on a un certain isomorphisme
$$(\E/T, \alpha, \beta) \simeq (\E/T, \alpha, g\beta).$$
Par rigidité de $\Pr$, cet isomorphisme doit être l'identité. Ainsi on a $g\beta = \beta$, et par hypothèse $\Qr_{E/T}$ est un $G$-torseur, ce qui implique que $g$ est lui-même trivial. Cela conclut la preuve que l'action de $G$ sur $\M(\Pr,\Qr)$ est libre.

Pour descendre $E, \alpha_{\univ}$ par ce quotient, on est dans le cadre général de la descente fidèlement plate. On cherche une \emph{donnée de descente}, c'est à dire un ensemble d'isomorphismes pour $g\in G$:
$$\theta(g)\ :\ g^*(E, \alpha_{\univ}) \overset{\sim}{\vers} (E, \alpha_{\univ})$$
qui forment un 1-cocycle (i.e. ils sont compatibles à la composition). On peut obtenir une telle donnée de la façon suivante. Pour tout $g\in G$, $(E, \alpha_{\univ}, g\beta_{\univ})$ est une courbe elliptique sur $\M(\Pr, \Qr)$ munie d'une $(\Pr,\Qr)$-structure, donc est classifiée par un unique morphisme
$$\theta(g) \de \M(\Pr,\Qr)\vers \M(\Pr,\Qr)$$
qui donne un isomorphisme au-dessus de $\M(\Pr,\Qr)$:
$$g^*(E, \alpha_\univ, \beta_\univ) \overset{\sim}{\vers} (E, \alpha_\univ, g\beta_\univ)$$
et il suffit s'oublier $\beta_\univ$. La compatibilité avec la composition provient directement de la rigidité de $\Pr$.

Alors $E$ descend car elle est projective, donc propre (c'est le deuxième cas d'application du théorème de Grothendieck, avec les schémas affines). Comme $\Pr$ est relativement affine, $\alpha_\univ$ descend également. On a ainsi rempli le carré cartésien précédent avec un objet $(E_0,\alpha_{\univ, 0})$ sur $\M(\Pr,\Qr)/G$.

Il reste maintenant à montrer que l'on obtient ainsi un représentant du problème $\Pr$. Soit $S$ un $\Z[1/N]$-schéma et $\E/S$ une courbe elliptique munie d'une structure $\alpha$ de niveau $\Pr$. On a un $G$-torseur fini étale sur $S$ : $\Qr_{\E/S}\vers S$ sur lequel $\E$ acquiert sa structure universelle de niveau $\Qr$ notée $\beta_\univ$. Alors on dispose d'une $(\Pr\times\Qr)$-structure
$$(E\times_S \Qr_{E/S}, \alpha, \beta_\univ).$$
On a donc un morphisme de classification associé $f \de \Qr_{\E/S}\vers \M(\Pr,\Qr)$. Il est tautologiquement $G$-équivariant donc en passant au quotient, il existe un morphisme $f_0$ rendant le diagramme commutatif :
$$
\shorthandoff{;:!?}
\xymatrix {
\Qr_{\E/S} \ar[d]_\pi \ar[r]^f &\ \M(\Pr,\Qr) \ar[d] \\
 S \ar[r]^{f_0} & \M(\Pr,\Qr)/G
}
$$

Il faut montrer que $f_0$ est le morphisme de classification de $(\E, \alpha)$ recherché, c'est  à dire :
\begin{enumerate}
\item $f_0^*(E_0, \alpha_{\univ, 0})$ est isomorphe à $(E, \alpha)$.
\item $f_0$ est l'unique application vérifiant cette propriété.
\end{enumerate}
Pour cette vérification finale, on renvoie à la preuve originale dans \cite{KaMa}.
%Pour 1., comme $\pi$ est étale et surjective (donc fidèlement plate), et comme $\Pr$ est rigide, il suffit de montrer que %$\pi^*f_0^*(E_0, \alpha_{\univ, 0})$ et $\pi^*(E,\alpha)$ sont isomorphes. C'est clair vu la définition de $f$.
%
%Regardons maintenant 2. Soit $h_0$ un morphisme satisfaisant cette propriété, et notons $X$ le produit fibré :
%$$
%\shorthandoff{;:!?}
%\xymatrix {
%X \ar[d] \ar[r]^h &\ \M(\Pr,\Qr) \ar[d]^{\pi_\univ} \\
% S \ar[r]^{h_0} & \M(\Pr,\Qr)/G
%}
%$$
%Alors $X/S$ est un $G$-torseur, et le pullback de $(E/S, \alpha)$ à $X$ acquiert une $\Qr$-structure $\beta$ (?). Cette $\Qr$-structure %est classifiée par un morphisme $G$-équivariant :
%$$
%\shorthandoff{;:!?}
%\xymatrix {
%X \ar[dr] \ar[rr] & &\ \Qr_{E/S} \ar[dl] \\
%& S &
%}
%$$
%qui est un isomorphisme (?), étant une application $G$-équivariante entre $G$-torseurs. Ainsi, on a un diagramme cartésien de %$G$-torseurs :
%$$
%\shorthandoff{;:!?}
%\xymatrix {
%\Qr_{\E/S} \ar[d]_\pi \ar[r]^h &\ \M(\Pr,\Qr) \ar[d]^{\pi_\univ} \\
% S \ar[r]^{h_0} & \M(\Pr,\Qr)/G
%}
%$$
%et un isomorphisme $h^*(E_\univ, \alpha_\univ, \beta_\univ) \simeq (E,\alpha, \beta_\univ)$ sur $\Qr_{E/S}$. Ainsi, on doit avoir $h = %f$ puisque ces deux applications classifient la structure $(E, \alpha, \beta_\univ)$. Par commutativité du diagramme ci-dessus, on en %déduit $h_0\pi = f_0 \pi$, d'où $f_0 = h_0$ car $\pi$ est étale et surjective, ce qui termine la preuve.
\end{proof}

\begin{cor}
Sous les hypothèses du théorème précédent, si de plus $\Pr$ est étale, alors $\M(\Pr)$ est une courbe affine lisse sur $\Z$.
\end{cor}
En effet, on a obtenu dans la preuve une construction du schéma affine $\M(\Pr)$ par recollement de quotients finis de $\M(\Pr, \Qr_2)$ et $\M(\Pr, \Qr_3)$. Or, les deux morphismes
$$\M(\Pr, \Qr_2)\vers \M(\Qr_2),\quad \M(\Pr, \Qr_3)\vers \M(\Qr_3)$$
sont étales, car $\Pr$ est supposé étale. Il suffit donc de montrer que $\M(\Qr_2)$ et $\M(\Qr_3)$ sont des courbes lisses sur $\Z$, car cette propriété est préservée par les morphismes étales et les quotients par un groupe fini agissant librement. Cette dernière assertion se vérifie directement sur les anneaux en question.

\begin{cor}
Pour tout entier $N\geq 3$, les problèmes modulaires $[Y(N)]$ et $[Y_1(N)]$ sont représentables par des courbes affines lisses sur $\Z$, notées respectivement $Y(N)$ et $Y_1(N)$.
\end{cor}

En effet, on a vu que ces deux problèmes modulaires sont relativement représentables, finis et étales donc en particulier affines. D'autre part, ils sont rigides (c'est une conséquence de la théorie de Hasse et du fait que le degré est une application quadratique) : on peut donc appliquer le résultat précédent.

Ainsi, la question est résolue pour ces deux problèmes rigides: le formalisme des schémas permet d'étendre les résultats obtenus sur $\C$ à n'importe quel corps. En revanche cela ne s'applique pas à $[Y_0(N)]$, qui est non rigide car l'action de $[-1]$ est triviale, ni à $[Y(1)]$ par exemple : ces problèmes modulaires ne sont tous simplement pas représentables. C'est le point de départ de la théorie des champs algébriques, dont un des buts est de représenter ce type de problèmes malgré leur non-rigidité. On se contente ici d'introduire brièvement la notion d'espaces de modules grossiers, qui permettent de montrer que tout se passe raisonnablement bien lorsque l'on regarde des corps uniquement.


\subsection{Espaces de modules grossiers et théorème d'Igusa}

Les premiers résultats concernant la réduction de courbes modulaires en caractéristique positive sont dus à Igusa. Il montre que les courbes modulaires classiques de niveau $N$ sont définies sur $\Q$, et ont bonne réduction et une interprétation modulaire en tous les premiers $p$ ne divisant pas $N$. Les résultats précédents permettent de retrouver cela dans le cas rigide, par exemple pour $Y_1(N)$ ou $Y(N)$.
\v

Lorsque l'on a affaire à un problème modulaire non représentable comme $[Y_0(N)]$, on peut utiliser la construction suivante. Soit $R$ un anneau, et $\Pr$ un problème modulaire relativement représentable et affine sur la catégorie $(\Ell/R)$. On construit l'\emph{espace de modules grossiers} $M(\Pr)$ de la façon suivante.
\begin{itemize}
\item[•] Localement sur $R$, on peut supposer qu'un entier $N\geq 3$ est inversible. Soit $\Qr$ un problème modulaire représentable fini, étale et galoisien sur $(\Ell/R)$, par exemple $[Y(N)]$ de groupe de Galois $GL(2, \Z/N\Z)$. On définit alors $M(\Pr)$ localement comme le schéma quotient (dont on a démontré l'existence lors de la preuve du théorème précédent):
$$M(\Pr) = \M(\Pr,\Qr)/G.$$
\item[•] Le schéma obtenu est indépendant du choix de $\Qr$: on peut donc recoller ces schémas en un schéma sur tout $R$.
\end{itemize}

Ce $R$-schéma est simplement $\M(\Pr)$ si $\Pr$ est représentable (dans la section précédente, on a montré que si $\Pr$ est rigide et affine, alors il est représenté par le schéma $\M(\Pr,\Qr)/G$). Si $\Pr$ n'est pas représentable, il fait office de meilleur remplaçant. Le point qui nous intéresse est que ce schéma, même s'il ne représente pas $\Pr$, a une interprétation en termes modulaires lorsque l'on se restreint à considérer les points définis sur un corps:

\begin{prop}
Si $k$ est un corps algébriquement clos, alors on a une identification
$$ M(\Pr)(k) = 
\left.
\begin{cases}
\ \text{Courbes\ elliptiques}\ \E/k \ \text{munies\ d'une\ structure} \\
\ \text{de\ niveau}\ \Pr,\ \text{à}\ k\text{-isomorphisme\ près}.
\end{cases}
\right\}$$
Si $k$ est un corps quelconque et $\bar{k}$ une clôture algébrique de $k$, alors on a l'identification
$$ M(\Pr)(k) = 
\left.
\begin{cases}
\ \text{Courbes\ elliptiques}\ \E/k \ \text{munies\ d'une\ structure} \\
\ \text{de\ niveau}\ \Pr,\ \text{à}\ \bar{k}\text{-isomorphisme\ près}.
\end{cases}
\right\}$$
\end{prop}
%\begin{proof}
%Dans le cas où $k$ est un corps algébriquement clos, le lemme suivant montre que la prise des $k$-points est compatible avec le quotient %par $G$, i.e. on a une bijection
%$$\M(\Pr, [Y(N)])(k)/G \overset{\sim}{\vers} M(\Pr)(k).$$
%Or le terme de gauche est exactement l'ensemble des classes de $k$-isomorphisme de courbes elliptiques $\E/k$ munies simultanément d'une %$\Pr$- et d'une $[Y(N)]$-structure, et quotienter par $G$ revient à oublier la seconde.
%
%Dans le cas où $k$ n'est pas algébriquement clos, les $k$-points de $M(\Pr)$ sont exactement les $\bar{k}$-points qui se factorisent par %$\Spec\bar{k}\vers\Spec k$, c'est à dire tels que la courbe $\E$ et la structure de niveau $\Pr$ peuvent être définis sur $k$. Cela %prouve le second point. 
%\end{proof}
%
%\begin{lem}
%Soit $R$ un anneau, $A$ une $R$-algèbre, et $G$ un groupe fini agissant $R$-linéairement sur $A$. On note $A^G \subset A$ la %sous-algèbre des $G$-invariants. Alors si $R'$ est une $R$-algèbre, le morphisme naturel
%$$A^G\otimes_R R' \vers (A\otimes_R R')^G.$$
%induit une bijection sur les points géométriques.
%\end{lem}
%
%Pour montrer la bijection précédente, on prend $\Spec A = \M(\Pr, [Y(N)])$ qui est une $R$-algèbre, et on choisit un point géométrique %$\Spec k\vers \Spec A \vers\Spec R$. Par le lemme,
%$$(\Spec A \times_R k)/G \simeq (\Spec A^G) \times_R k.$$
%On dit alors simplement que les points fermés de ces deux schémas sont en bijection.
%\begin{proof}[Démonstration du lemme]
%
%\end{proof}
On renvoie à la référence originale \cite{KaMa} pour la preuve de ce résultat. À l'aide de cet outil, on peut formuler le théorème d'Igusa comme suit.


\begin{thm} Soit $N\geq 3$ un entier.

\begin{itemize}

\item[(i)] La courbe modulaire $Y_0(N)_\C$ peut être définie sur $\Q$.
\item[(ii)] Pour tout $p \nmid N$, celle-ci a bonne réduction en $p$, c'est à dire qu'elle se réduit en une courbe lisse définie sur $\F_p$.
\item[(iii)] Pour tout corps $k$ dans lequel $N$ est inversible, la courbe $Y_0(N)_k$ a une interprétation modulaire. Les $k$-points de $Y_0(N)_k$ sont en bijection avec les classes d'isomorphisme (sur $\bar{k}$) d'isogénies cycliques de degré $N$ entre deux courbes elliptiques sur $k$.
\item[(iv)] Il existe un morphisme de schémas
$$Y_0(N) \vers \A^1\times \A^1$$
dont l'image est exactement le lieu des zéros du polynôme modulaire $\Phi_N$. 

\end{itemize}

\end{thm}

\begin{proof}
Soit $Y_0(N)$ le schéma de modules grossier construit de la manière précédente, défini sur $\Z[1/N]$. $\C$ étant un corps algébriquement clos, les $\C$-points de $Y_0(N)$ classifient les isogénies cycliques de degré $N$ à coefficients complexes à isomorphisme près. On a donc l'égalité
$$Y_0(N)_\C = Y_0(N) \otimes \C,$$
ce qui montre que la courbe modulaire classique $Y_0(N)_\C$ peut être définie sur $\Q$, et même sur $\Z[1/N]$. Pour le second point, on note que par construction, $Y_0(N)$ est une courbe lisse sur $\Z[1/N]$. Cette propriété étant préservée par changement de base, on en déduit que toutes les réductions de $Y_0(N)$ modulo $p$ avec $p \nmid N$ sont des courbes lisses. Le point $(iii)$ est exactement le résultat énoncé ci-dessus au sujet des points d'un schéma de modules grossier pris sur un corps.

Enfin, pour le dernier point, on peut montrer que le schéma $\A^1$ sur $\Z[1/N]$ est en fait le schéma de modules grossier associé au problème modulaire $Y(1)$. On dispose bien sûr d'un morphisme naturel de problèmes modulaires $[Y_0(N)] \vers [Y(1)]\times [Y(1)]$ en \og oubliant\fg\ l'isogénie et en gardant simplement le départ et l'arrivée, et on peut dérouler la construction ci-dessus pour obtenir un morphisme de schémas comme annoncé.

Montrons que l'image schématique de $Y_0(N)$ dans $\A^1\times \A^1$ est exactement la courbe définie par le polynôme $\Phi_N$. Notons $Z$ l'image de $Y_0(N)$, et $S$ la courbe définie par $\Phi_N$. Le cas complexe montre que $Z$ et $S$ ont même fibre générique. D'autre part, on a vu que $\Phi_N$ est irréductible donc $S$ est plat, et d'autre part $Z$ est plat car $Y_0(N)$ et $\A^1$ le sont. Ainsi ces deux schémas sont l'adhérence de leur fibre générique, donc sont égaux.
\end{proof}
\v

En réalité, on peut faire mieux et caractériser les points singuliers de $Y_0(N)\vers Y\times Y$: ce sont des réductions de points singuliers sur $\Z$, c'est à dire les mêmes que dans le cas complexe. Un argument est donné dans \cite{}. En particulier, on saura qu'en dehors de ces points singuliers, une solution dans un corps $k$ de l'équation $\Phi_N = 0$ correspondra à un unique point défini sur $k$ de la courbe $Y_0(N)$, donc à une unique isogénie cyclique définie sur $k$ à $\bar{k}$-isomorphisme près.
%En effet, soit $x$ un point singulier de $Y_0(N)$ en caractéristique $p$ qui n'est pas réduction d'un point singulier sur $\Z$. Alors il existe un ouvert pour la topologie de Zariski régulier en codimension 1 autour de ce point. Cela implique que la courbe donnée par $\Phi_N=0$ est normale sur $\F_p$. Or la normalisation de $\Phi_N=0$ est précisément la courbe modulaire $Y_0(N)$ qui est lisse: une contradiction.
\v

L'objectif de cette partie est donc atteint: on dispose d'une équation, le polynôme modulaire, dont les solutions sont les $j$-invariants de courbes liées par des isogénies cycliques de degré donné. Des arguments similaires peuvent sans doute être utilisés pour montrer que d'autres équations de courbes obtenues sur $\C$ restent valides sur $\F_p$, et l'expérience montre que c'est en effet le cas. Dans la suite de mémoire, on utilisera donc dans $\F_p$ les équations ayant un sens sur $\C$. Ces outils permettent d'effectuer les calculs nécessaires au protocole de Couveignes--Rostovtsev--Stolbunov, que l'on étudie maintenant.

\newpage

\section{Cryptosystème et algorithmes}

La sécurité de nombreux cryptosystèmes à clefs publiques est basée sur la difficulté supposée d'un problème mathématique. Prenons comme exemple l'un des protocoles d'échange de clés le plus ancien et le plus connu, proposé par Diffie et Hellman en 1976. Dans un groupe $G$, typiquement le groupe multiplicatif d'un corps fini, on dispose d'un élément $g$ d'ordre $N$, tous ces éléments étant publics. Alice et Bob souhaitent partager une clé secrète. Pour cela, Alice choisit un élément $a\in (\Z/N\Z)^\times$, Bob choisit $b\in (\Z/N\Z)^\times$, ils calculent respectivement $g^a$ et $g^b$ et les publient. Les deux protagonistes peuvent alors calculer leur secret commun $g^{ab}$. La sécurité de ce protocole repose notamment sur la difficulté du problème du logarithme discret, qui peut se formuler ainsi: étant donnés $g$ et $g^a$, trouver $a$. Le fait que ce problème soit difficile lorsque $N$ est premier fait consensus. On considère qu'un problème est \emph{difficile} s'il n'existe pas d'algorithme fonctionnant en temps polynomial qui, recevant en entrée les données du problème, en donne la solution avec probabilité supérieure à $\frac{1}{2}+\varepsilon(n)$ (pour un problème décisionnel) ou $\varepsilon(n)$ (pour un problème computationnel). Ici, $\varepsilon$ est une fonction négligeable devant les polynômes, et $n$ est le \emph{paramètre de sécurité} du système, typiquement la taille des données (par exemple $\log(N)$ pour un groupe de cardinal $N$).

On peut immédiatement reformuler le protocole de Diffie et Hellman de manière plus générale, suivant une idée de Couveignes \cite{Couv}. Dans la situation ci-dessus, le groupe abélien $G = (\Z/N\Z)^\times$ agit sur l'ensemble $X = \{g^k, k\in (\Z/N\Z)^\times\}$ de façon simplement transitive, $X$ étant muni d'un point fixé $g$. On peut alors formuler le même protocole pour toute action de ce type. La sécurité de ce protocole repose alors sur la difficulté du problème suivant: étant donnés des éléments $x_0, a\cdot x_0, b\cdot x_0\in X$, calculer $ab\cdot x_0$.

Couveignes suggère d'utiliser l'action d'un groupe de classes sur un ensemble de courbes elliptiques sur un corps fini, décrite au début de ce document. Il s'agit du cryptosystème décrit dans l'article de Rostovtsev et Stolbunov \cite{RoSt}. L'objectif du reste de ce mémoire est d'étudier ce protocole, les différents algorithmes utilisés et son applicabilité en pratique, et d'en rechercher quelques améliorations.


\subsection{Représentation des objets mathématiques}

On fixe un corps fini $k$ de cardinal $p$, un nombre premier, et un entier $t\leq 2\sqrt{p}$. Ce choix d'un corps premier n'est pas motivé par des raisons théoriques, mais uniquement pratiques: c'est le corps dans lequel les opérations sont le plus efficaces, à taille comparable. On note $K$ le corps quadratique
$$K = \Q[\pi] / (\pi^2 - t\pi + q),$$
on se donne une courbe elliptique $E_0/k$ dont la trace du Frobenius est égale à $t$, et on note $\O$ l'ordre de $K$ qui est son anneau d'endomorphismes. On détaille maintentant la façon dont sont représentés divers objets mathématiques \og en machine\fg.

\v

\emph{Courbes elliptiques.} Une courbe elliptique est d'abord une courbe algébrique plane, et on peut donc la représenter par une de ses équations. On utilise deux formes de représentations de courbes elliptiques : la forme de Weierstrass réduite
$$\E\de y^2 = x^3 + Ax + B$$
et la forme dite de Montgomery
$$\E\de B y^2 = x^3 + A x^2 + x.$$
Notons que toute courbe elliptique sur $k$ n'admet pas nécessairement de représentation sous forme de Montgomery : en effet, une telle courbe admet en particulier un point rationnel de 2-torsion qui est $(0,0)$, et ce n'est pas le cas de toutes les courbes.
On s'autorise également à représenter une courbe elliptique à isomorphisme près par son $j$-invariant, qui est un élément de $k$. Ce faisant, on perd de l'information par rapport aux équations ci-dessus, car on ne peut plus distinguer une courbe de sa tordue.

Sur un corps fini, une courbe a en général une unique tordue, qui lui est isomorphe sur une extension quadratique de $k$ mais pas sur $k$ lui-même. Tordre une courbe a pour effet de changer sa trace de signe. Dans le modèle de Montgomery, la tordue a une forme agréable : il s'agit de la courbe
$$E'\de B'y^2 = x^3 + Ax^2 + x$$
où $B'$ est un carré dans $k$ si $B$ ne l'est pas et réciproquement. Ajouter deux points sur une courbe elliptique sous forme de Montgomery est également plus efficace que sur une courbe sous forme de Weierstrass, ce qui est l'intérêt principal de ces courbes.
\v

\emph{Sous-groupes finis d'une courbe elliptique et isogénies.} Si $S$ est un sous-schéma en groupe fini (et plat) de $E$ d'ordre $\ell$, où $p\neq \ell$ et $\ell$ est impair, on représente le sous-groupe $S$ de la façon suivante. $S$ est étale donc déterminé par ses $\bar{k}$-points, et est stable par l'automorphisme $[-1]$ de $E$.
En écrivant une équation pour $E$, il existe donc un polynôme $K_S\in k[X]$ de degré $\frac{\ell-1}{2}$ dont les racines sont exactement les coordonnées $x$ des points de $S$. On représente alors $S$ par ce polynôme. Remarquons que cette représentation n'est univoque que si l'on se restreint aux sous-groupes étales.

Comme on l'a vu précédemment, une isogénie séparable $\phi$ est déterminée par sa courbe de départ et son noyau. Si elle est de degré impair, on peut donc la représenter par un polynôme comme ci-dessus, que l'on appelle \emph{polynôme de noyau}, noté $K_\phi$. On représentera donc uniquement des isogénies séparables de cette manière.

Dans le cas de sous-groupes d'ordre pair et d'isogénies de degré pair, on peut adopter la même stratégie et définir un polynôme en $x$ qui s'annule sur les points du sous-groupe. L'inconvénient en est que les zéros de ce polynôme $K(x)$ n'auront pas partout la même multiplicité, ce qui complique les formules. On considèrera uniquement des isogénies de degré impair dans la suite de ce mémoire. 
\v

\emph{Idéaux.} On dit qu'un nombre premier $\ell$ est \emph{Elkies} si le polynôme $X^2 - tX + q$ est scindé à racines simples modulo $\ell$, c'est à dire si son discriminant est un carré non nul modulo $\ell$. Si $\ell$ est un nombre premier d'Elkies, alors $\ell$ se scinde dans $\O$ sous la forme $(\ell) =~\frak l \bar{\frak l},$
et ces deux idéaux sont exactement les idéaux de norme $\ell$ dans $\O$. Pour les distinguer, on peut utiliser la remarque suivante. Si $E$ est une courbe elliptique ayant les bons paramètres, on sait que $E[\ell](\bar{k})$ est un $\O/\ell\O$-module de rang 1. D'autre part, on a un isomorphisme canonique par le théorème chinois :
$$\O/\ell\O \simeq \O/\frak l \O \times \O/\bar{\frak l} \O.$$
L'endomorphisme de Frobenius $\pi$, vu comme endomorphisme du $\F_\ell$-espace vectoriel $E[\ell](\bar{k})$ de dimension 2, admet donc deux valeurs propres (nécessairement distinctes car $\ell$ est d'Elkies) qui sont exactement les éléments $\pi\mod \frak l$ et $\pi\mod \bar{\frak l}$ : on les appelle \emph{valeurs propres du Frobenius} modulo $\ell$. On représentera alors un idéal de norme $\ell$ par le couple $(\ell, v)$ où $v$ est la valeur propre du Frobenius associée. D'un point de vue pratique, les valeurs propres du Frobenius sont les racines dans $\F_\ell$ du polynôme $X^2 - tX + q$ ; on reviendra plus tard sur le calcul de $t$.

Une isogénie de degré $\ell$ définie sur $k$ partant de $E$, à isomorphisme près, est simplement un sous-groupe cyclique de $E(\bar{k})$ de cardinal $\ell$ stable par Galois, c'est à dire le Frobenius. On vient de voir que l'action du Frobenius sur $E[\ell](\bar{k})$ est diagonale, avec des valeurs propres distinctes. On en déduit que les seules isogénies rationnelles de degré $\ell$ partant de $E$ sont celles qui proviennent de l'action du groupe de classes.
\v

\emph{Idéaux fractionnaires et groupe de classes.} On utilisera toujours des idéaux fractionnaires écrits sous la forme $\prod {\frak l}_i^{r_i}$, où les ${\frak l}_i$ sont des idéaux dont la norme est un petit nombre premier d'Elkies. Cela implique, en particulier, que l'on ne travaille qu'avec un sous-groupe du groupe des idéaux fractionnaires.

On choisit de donner un élément de $\Cl(\O)$ par un de ses représentants, que l'on écrit comme ci-dessus. Notons que l'on pourrait toujours atteindre le groupe de classes en entier même en se restreignant à un sous-groupe des idéaux fractionnaires. On décomposera donc toujours l'action d'un élément du groupe de classes en actions successives d'idéaux \og simples\fg, c'est à dire de la forme $(\ell, v)$ où $\ell$ est un petit nombre premier d'Elkies.

On peut donner à cette décomposition un aspect plus visuel à l'aide du concept de \emph{graphe d'isogénies}. Les sommets de ce graphe sont des courbes elliptiques sur $k$ à isomorphisme près (donc des $j$-invariants), et l'on relie ces sommets par une arête labellée par un premier $\ell$ s'il existe une isogénie de degré $\ell$ les reliant. Comme on l'a dit, l'action du groupe de classes par isogénies est simplement transitive, et on obtient donc simplement un graphe de Cayley pour le groupe de classes relativement aux \og générateurs\fg\ $(\ell, v)$ pour de petits $\ell$. Cela explique sa régularité:

\begin{center}
\begin{tikzpicture}[line cap=round,line join=round, x=3cm,y=3cm]

\draw[]
 (1, 0) node(1) {2}
 (0.62, 0.78) node(2) {162}
 (-0.22,0.97) node(3) {36}
 (-0.9,0.43) node(4) {117}
 (-0.9,-0.43) node(5) {134}
 (-0.22,-0.97) node(6) {116}
 (0.62,-0.78) node (7) {167};
\draw[]
 (1) edge[thick, blue] (2)
 (2) edge[thick, blue] (3)
 (3) edge[thick, blue] (4)
 (4) edge[thick, blue] (5)
 (5) edge[thick, blue] (6)
 (6) edge[thick, blue] (7)
 (7) edge[thick, blue] (1);
\draw[]
 (1) edge[thick, red] (3)
 (3) edge[thick, red] (5)
 (5) edge[thick, red] (7)
 (7) edge[thick, red] (2)
 (2) edge[thick, red] (4)
 (4) edge[thick, red] (6)
 (6) edge[thick, red] (1);

\end{tikzpicture}
\end{center}


On montre ici un graphe d'isogénies sur $\F_{173}$; les sommets sont des $j$-invariants, les arêtes bleues représentent des isogénies de degré 3, et les arêtes rouges des isogénies de degré 7. La décomposition précédente revient simplement à découper un voyage complet dans ce graphe en plusieurs pas. Le graphe de couverture est un graphe d'isogénies sur $\F_{503}$, avec le degré 3 en bleu, 11 en rouge et 13 en vert. 
\v

On peut maintenant proposer plusieurs manières d'effectuer un pas dans ce graphe d'isogénies.

\subsection{Calcul à l'aide d'un polynôme de division}

Si $\phi\de E\vers E'$ est une isogénie séparable de degré $\ell$, alors son noyau est constitué de points de $E$ d'ordre $\ell$. Le polynôme $K_\phi$ est donc un facteur de degré $\frac{\ell-1}{2}$ à coefficients dans $k$ du $\ell$-ième \emph{polynôme de division} de la courbe $E$, qui est simplement le polynôme de noyau de l'isogénie $[\ell]$ de degré $\ell^2$. Avec cette idée, on obtient l'algorithme suivant.

\begin{algorithm}
\caption{un pas dans le graphe d'isogénies à l'aide des polynômes de division}
\label{alg:div}
\KwIn{Une courbe elliptique $E\in \Ell_k(\O)$, et un idéal représenté par $(\ell, v)$}
\KwOut{Une courbe elliptique $E'$ telle que l'isogénie $E\vers E'$ corresponde à cet idéal}
$\psi_\ell \gets \textsc{PolynômeDeDivision}(E, \ell)$
\label{alg:div:poldiv}
\;
$F \gets \textsc{Factorisation}(\psi_\ell)$
\label{alg:div:fact}
\;
$K \gets \textsc{SousGroupe}(F, \frac{\ell-1}{2}, v)$
\label{alg:div:rec}
\;
\lElse{\Return{$\textsc{Quotient}(E, K)$}}

\end{algorithm}

Seul le but de l'étape~\ref{alg:div:rec} nécessite peut-être une explication : il consiste à retrouver à partir des facteurs irréductibles de $\psi_l$ le polynôme de degré $\frac{\ell-1}{2}$ définissant le sous-groupe cycliques de $E$ qui est le sous-espace propre du Frobenius sur $E[\ell](\bar{k})$ associé à la valeur propre $v$. Cette étape est nécessaire car ces polynômes ne sont pas nécessairement irréductibles, et $\psi_l$ peut avoir beaucoup d'autres facteurs qui ne déterminent pas des sous-groupes mais seulement des sous-ensembles de $E[\ell](\bar{k})$.

Examinons maintenant chacune des étapes de l'algorithme~\ref{alg:div}, en commençant par le calcul des polynômes de division.

\begin{lem}[Polynômes de division]
Soit $E$ une courbe elliptique sous forme de Weierstrass réduite
$$y^2 = x^3 + Ax + B.$$
Il existe des polynômes dits \emph{de division} $\Psi_n \in \Z[x,y]/(y^2 - x^3 - Ax - B)$, $n\geq 1$ tels que pour tout $n\geq 1$, l'endomorphisme $[n]_E$ soit donné sur l'ouvert affine par les applications rationnelles
$$(x,y) \longmapsto \left(\frac{x\Psi_n^2 - \Psi_{n-1}\Psi_{n+1}}{\Psi_n^2},
\frac{\Psi_{n+2}\Psi_{n-1}^2 - \Psi_{n-2} \Psi_{n+1}^2}{4y \Psi_n^3}\right),$$
en posant $\Psi_0 = 0,\ \Psi_{-1} = -1,$ et tels que l'on ait :
$$\begin{aligned}
 \Psi_1&= 1,\\
 \Psi_2&= 2y,\\
 \Psi_3&= 3x^4 + 6Ax^2 + 12Bx - A^2, \\
 \forall n\geq 2,\ 2y\Psi_{2n} &= \Psi_n(\Psi_{n+2}\Psi_{n-1}^2 - \Psi_{n-2} \Psi_{n+1}^2), \\
\forall n\geq 2 ,\ \Psi_{2n+1} &= \Psi_{n+2}\Psi_n^3 - \Psi_{n+1}^3\Psi_{n-1}.
\end{aligned}$$
\end{lem}

Lorsque $n$ est impair, on montre aisément par récurrence que $\Psi_n$ peut être vu comme un polynôme en $x$ uniquement et de degré $\frac{n^2-1}{2}$. Si de plus $p\nmid n$, on sait que $\#E[n](\bar{k})=n^2$, donc $\Psi_n$ est un polynôme séparable dont les racines sont exactement les coordonnées $x$ des points de $n$-torsion de la courbe. 
%\begin{proof}
%Il s'agit d'une simple récurrence en utilisant la forme explicite de la loi de groupe sur la courbe. Rappelons les formules d'addition %sur une courbe elliptique en forme de Weierstrass, données par le mécanisme \og corde et tangente \fg\ : si $P$, $Q$ sont des points %affines de $E(\bar{\F_p})$ de coordonnées, $(x_P, y_p),\ (x_Q, y_Q)$ avec $x_P\neq x_Q$, on a
%$$-P = (x_P, -y_P),\quad P+Q = (\lambda^2 - (x_P + x_Q), -\lambda(x_P + \mu))$$
%où
%$$\begin{cases}
%\lambda = \frac{y_P - y_Q}{x_P - x_Q},\\
%\mu = \frac{x_P y_Q - x_Q y_P}{x_P - x_Q}.
%\end{cases}$$
%Ainsi, la droite d'équation $y = \lambda x + \mu$ est la droite passant par $P$ et $Q$. D'autre part, on a
%$$2P = \left(\frac{(3x_P^2 + A)^2 - 8x_P(x_P^3 + Ax_p + B)}{4(x_P^3+Ax_P+B)},\frac{x_P^3-Ax-2B-(3x_P^2+A)x_{2P}}{2y_P}\right).$$
%En écrivant $2nP = nP + nP$ et $(2n+1)P = nP + (n+1) P$, on trouve les formules de récurrence annoncées (?).
%\end{proof}
On utilise ces relations de récurrence, obtenues à partir des formules d'additions de points sur la courbe, pour écrire l'algorithme~{\sc PolynômeDeDivision} dans l'étape~\ref{alg:div:poldiv}.
\v

Pour l'étape \ref{alg:div:fact}, {\sc Factorisation}, on utilise l'algorithme de Cantor--Zassenhaus \cite{vzGG}. La première étape est de séparer les facteurs irréductibles d'un polynôme $P$ par degré, ce que l'on fait en prenant le pcgd avec les polynômes $X^p - X$, puis $X^{p^2} - X$, etc. que l'on calcule directement modulo $P$ par mises au carré répétées. Ensuite, on tente de factoriser le reste obtenu $R$ en prenant un polynôme au hasard $a$ et en calculant $\mathrm{pgcd}(a^{\frac{p^r - 1}{2}} + 1, R)$. L'idée de l'algorithme est que si $R$ est un produit de facteurs irréductibles de degré $r$, alors $k[X]/R$ est un produit de corps de cardinal $p^r$, et le calcul de cette puissance donne aléatoirement 1 ou -1 dans chacun de ces corps.
\v

Dans l'étape \ref{alg:div:rec}, il s'agit de retrouver le polynôme à coefficients dans $k$ définissant un certain sous-groupe cyclique d'ordre $\ell$. Pour cette opération, on peut utiliser l'astuce suivante. Comme on connaît la valeur propre du Frobenius $v$ sur le sous-groupe recherché, on peut calculer l'action du Frobenius sur chacun des facteurs irréductibles de $\psi_\ell$ et rassembler les morceaux. Pour un facteur irréductible $Q$, on calcule $[v](X, Y)$ pour le point tautologique $(X, Y)$ sur l'anneau $k[X, Y]/(Q,\ Y^2 - X^3 - AX - B).$ Si ce point est égal à $(X^p, Y^p)$, on sait que l'on a trouvé un facteur du polynôme de noyau $K$.
\v
%
%Dans l'étape \ref{alg:div:check}, on souhaite vérifier si le polynôme de noyau $K_1$ correspond bien à l'action de l'idéal $\frak l$ donné par $(\ell,v)$. $v$ étant défini par $\pi = v \mod \frak l$, l'endomorphisme $\pi - [v]$ de $E$ est un élément de $\frak l$, et doit donc s'annuler sur le sous-groupe défini par $K_1$. C'est cette vérification qui consiste l'algorithme {\sc CompareDirection}. Concrètement, il s'agit du même calcul que ci-dessus: le test est validé si on a l'égalité $[v_1](X, Y) = (X^p, Y^p)$ dans l'anneau $k[X, Y]/(K_1, Y^2 - X^3 - aX - b)$.

Enfin, on souhaite calculer une équation de la courbe image connaissant $E$ et le noyau $K$ de l'isogénie. C'est là qu'intervient la preuve \og directe\fg\ de l'existence du quotient évoquée plus tôt dans ce document.

\begin{lem}[Formules de Vélu]
Soit $E$ une courbe elliptique donnée par une équation sous forme de Weierstrass
$$y^2 + a_1 x y + a_3 y = x^3 + a_2 x^2 + a_4 x + a_6,$$
et soit $K = \sum_{i=0}^n (-1)^{n-i} \sigma_i X^i$ un sous-groupe de $E$, cyclique d'ordre $2n+1$. Notons $b_2, b_4, b_6, b_8$ les $b$-invariants de l'équation de $E$, et 
$$ t = 6(\sigma_1^2 - 2 \sigma_2) + b_2 \sigma_1 + n b_4,$$
$$ w = 10 (\sigma_1^3 - 3 \sigma_1 \sigma_2 + 3 \sigma_3) + 2  b_2 (\sigma_1^2 - 2 \sigma_2) + 3 b_4 \sigma_1 + n b_6.$$
Alors, en notant $E'$ la courbe donnée par les invariants
$$ a_1' = a_1,\ a_2' = a_2,\ a_3' = a_3,\ a_4' = a_4 - 5 t,\ a_6' = a_6 - b_2 t - 7 w,$$
il existe une isogénie séparable $E\vers E'$ de noyau $K$.
\end{lem}

Les $b$-invariants sont des quantités qui apparaissent lors de la réduction de $E$ en forme de Weierstrass courte; ce sont des expressions polynomiales simples en les $a_i$. L'idée de la preuve est la suivante.
Soit $G$ le sous-groupe étale de $E$ défini par $K$. On définit deux fonctions rationnelles sur $E$:
$$\begin{aligned}
x_G(P) = x(P) + \sum_{Q\in G\backslash \{0\}} x(P+Q) - x(Q),\\
y_G(P) = y(P) + \sum_{Q\in G\backslash \{0\}} y(P+Q) - y(Q).
\end{aligned}$$
Ces fonctions rationnelles sont bien définies et invariantes par $G$ : elles définissent donc des fonctions sur la courbe $E/G$. On montre alors qu'elles satisfont l'égalité donnée par l'équation de Weierstrass ci-dessus, et qu'elles engendrent le corps des fonctions de la courbe quotient.
Une discussion plus détaillée sur ces formules se trouve dans \cite{Kohel}, notamment leur extension aux isogénies cycliques de degré pair.

\v
On dispose depuis récemment de meilleures formules dans le cas de la forme de Montgomery, données dans \cite{VeluMontgomery}: si $P$ est un point d'ordre $2n+1$ sur une courbe de Montgomery d'équation $By^2 = x^3 + Ax^2 + x$, alors il existe une isogénie normalisée dont le noyau est le sous-groupe engendré par $P$ vers la courbe d'équation
$$B'y^2 = x^3 + A'x^2 + x$$
avec $A' = (6\sigma + A)\pi^2,\ B' = B\pi^2$, où
$$
\pi = \prod_{i = 1}^n x_{iP}, \quad
\sigma = \sum_{i = 1}^n \left(\frac{1}{x_{iP}} - x_{iP}\right).
$$
Comme ci-dessus, on peut réexprimer $\pi$ et $\sigma$ en termes des coefficients du polynôme $K$.

Sans utiliser ces formules, une autre méthode est disponible : étant donné une courbe sous forme de Montgomery, on peut calculer une équation de Weierstrass de cette courbe, et utiliser les formules de Vélu pour cette courbe. En se souvenant du point de 2-torsion, on peut ensuite retrouver une équation de Montgomery pour la courbe quotient sans extraction de racine carrée.

\v
D'un point de vue pratique, factoriser entièrement un polynôme de degré $\frac{\ell^2 - 1}{2}$ sur $k$ lorsque $p$ est grand et $\ell$ n'est pas très petit, est une opération trop coûteuse. L'étape de reconstruction du sous-groupe n'est pas non plus gratuite.%; en face de ces coûts, celui des formules de Vélu est tout à fait négligeable.
Cela pousse préférer une autre méthode à l'aide d'une équation modulaire, ce que l'on décrit maintenant.


\subsection{Calcul à l'aide d'une équation modulaire}

On a vu que la courbe modulaire $Y_0(\ell)_k$ paramétrise les isogénies cycliques de degré $\ell$ définies sur $k$. Le polynôme modulaire de degré $\ell$ relie les $j$-invariants de ces courbes ; on a vu de plus que les points singuliers sur $k$ de l'application
$$Y_0(\ell) \vers \A^1 \times \A^1,$$
dont l'image et définie par le polynôme modulaire, sont nécessairement des réductions modulo $p$ de points singuliers complexes, donc peu nombreux (et dont l'ordre a un petit discriminant). En utilisant ces outils, on obtient l'algorithme suivant pour un calcul d'isogénie, qui est valide lorsque le discriminant de la courbe est supérieur à $\ell^2 + 1$, selon la borne grossière obtenue précédemment.

\begin{algorithm}
\caption{un pas dans le graphe d'isogénies en utilisant une équation modulaire}
\label{alg:mod}
\KwIn{Une courbe elliptique $E\in \Ell_k(\O)$, et un idéal représenté par $(\ell, v)$}
\KwOut{Une courbe elliptique $E'$ telle que l'isogénie $E\vers E'$ corresponde à cet idéal}
$\Phi_\ell(X, Y) \gets \textsc{PolynômeModulaire}(\ell)$
\label{alg:mod:polmod}
\;
$(j_1, j_2) \gets \textsc{Racines}(\Phi_\ell(j(E), Y), k)$
\label{alg:mod:roots}
\;
$K_1 \gets \textsc{Noyau}(E, \ell, j_1)$
\label{alg:mod:ker}
\;
\lIf{$\textsc{ValeurPropre}(K_1,v)$}{\Return{$\textsc{Courbe}(j_1)$}}
\label{alg:mod:check}
\lElse{\Return{$\textsc{Courbe}(j_2)$}}

\end{algorithm}

De la même façon, on étudie les étapes de l'algorithme l'une après l'autre. On a vu une manière de calculer le polynôme modulaire, mais le plus rapide reste d'utiliser une base de données préfabriquée. Pour le calcul des racines, on utilise à nouveau l'algorithme de Cantor--Zassenhaus, mais le polynôme à factoriser est cette fois de degré seulement $\ell + 1$. Enfin, l'algorithme {\sc ValeurPropre} est analogue à la reconstruction du sous-groupe dans l'algorithme 1, mais le calcul du Frobenius est effectué modulo un unique polynôme de degré $\frac{\ell-1}{2}$. 

\v
Examinons maintenant le calcul du noyau. Soit $\phi\de E\vers E'$ une isogénie. On a une application $k$-linéaire \og pullback\fg, des formes différentielles sans pôle de $E'$ vers celle de $E$:
$$\phi^*\de H^0(E', \Omega^1)\vers H^0(E, \Omega^1).$$
Rappelons que $E$ et $E'$ sont de genre 1, donc ces espaces sont de dimension 1. On sait que cette application est nulle si et seulement si $\phi$ est non séparable. Si l'on fixe une équation de Weierstrass pour $E$ et $E'$, on dispose de 1-formes privilégiées et donc d'un isomorphisme (non canonique, donc)
$$\Hom(H^0(E', \Omega^1), H^0(E, \Omega^1))\simeq k.$$
On dit alors que $\phi$ est \emph{normalisée} si $\phi^* = 1$ dans cette identification. Lorsque $\phi$ est normalisée, la condition de pullback donne une équation différentielle satisfaite par $\phi$ qui est donnée par la proposition qui suit, et qui est à la base de l'algorithme de calcul de noyau. Cette idée est due à Elkies et Stark.

\begin{prop}
Soit $\phi\de E\vers E'$ une isogénie normalisée de degré $\ell$ entre courbes sous forme de Weierstrass telles que $a_1 = a_3 = a_1' = a_3' = 0$ (ainsi les équations sont de la forme $y^2 = f(x)$). Notons $\phi_x(x,y),\ \phi_y(x,y)$ les applications rationnelles définissant $\phi$ sur l'ouvert affine ; alors $\phi_x$ est une fonction paire, donc peut s'écrire sous forme irréductible
$$\phi_x(x, y) = \frac{N(x)}{D(x)}$$
où $N$ et $D$ sont des polynômes de degré au plus $\ell$. Notons
$$G(X) = a_6 X^3 + a_4 X^2 + a_2 X + 1,\ H(X) =a_6' X^3 + a_4' X^2 + a_2' X + 1.$$
Alors la fraction rationnelle 
$$T(X) = \frac{D(1/X)}{N(1/X)}$$
est solution de l'équation différentielle
$$\frac{T}{X} H(T) - G(X) T'^2 = 0.$$
\end{prop}

\begin{proof}
Comme $\phi$ est normalisée, on sait que $\phi_y (x, y) = y \phi_x'(x).$ En écrivant 
$$\phi_x(x) = \frac{N(x)}{D(x)} = \frac{1}{T(1/x)},$$
on a donc
$$\phi_y(x, y) = \frac{y}{x^2} \frac{T'(1/x)}{T(1/x)^2}$$
et
$$ \left(\frac{y}{x^2} T'(1/x)\right)^2 = \left(\frac{1}{T^3(1/x)} + \frac{a_2'}{T^2(1/x)} + \frac{a_4'}{T(1/x)} + a_6'\right)T(1/x)^4$$
ce qui donne, vu l'équation de $E$ :
$$ \frac{1}{x} G(1/x) T'^2(1/x) =T(1/x) H(T(1/x))$$
et on obtient l'équation annoncée après un changement de variable.
\end{proof}
\v

Remarquons que dans cette description, lorsque $\ell$ est impair, le polynôme $D$ est précisément le carré du polynôme de noyau $K_\phi$. Précisons maintenant comment se déroule le calcul du noyau.


\newpage

\begin{algorithm}
\caption{{\sc Noyau} : Calcul du noyau de l'isogénie}
\label{alg:ker}
\KwIn{Une courbe elliptique $E\in \Ell_k(\O)$, un nombre premier $\ell\neq p$, un élément $j'\in k$ qui est le $j$-invariant d'une courbe reliée à $E$ par une isogénie cyclique de degré $\ell$}
\KwOut{Le noyau de cette isogénie}
$E' \gets \textsc{EquationNormalisée}(E, \ell, j')$
\label{alg:ker:eq}
\;
$T \gets \textsc{SolutionEquaDiff}(E, E')$
\label{alg:ker:newt}
\;
$D \gets \textsc{Dénominateur}(1/T(1/X))$
\label{alg:ker:bm}
\;
$K\gets \textsc{RacineCarrée}(D)$
\label{alg:ker:sqrt}
\;
\Return{$K$}

\end{algorithm}

A nouveau, prenons ces étapes dans l'ordre, à commencer par le calcul d'une équation normalisée.

\begin{lem}
Avec les notations de l'algorithme, soit $\Phi_ \ell(X, Y)$ le polynôme modulaire de degré $\ell$. On suppose que $E$ est donnée sous forme réduite
$$y^2 = x^3 + A x + B.$$
On définit
$$\lambda = \frac{-18}{\ell}\cdot\frac{B}{A}\cdot\frac{\frac{\partial \Phi_\ell}{\partial X} (j(E), j')}{\frac{\partial \Phi_\ell}{\partial Y} (j(E), j')} \cdot j(E) $$
Alors, si $E'$ désigne la courbe d'équation
$$y^2 = x^3 + A' x + B' $$
avec
$$A' = \frac{-\lambda^2}{48 \ell^4 j' (j' - 1728)},\ B' = \frac{-\lambda^3}{864 \ell^6 j'^2(j'-1728)},$$
on a $j(E') = j'$ et il existe une isogénie normalisée $E\vers E'$ de degré $\ell$.
\end{lem}

Dans le cas complexe, on peut en effet supposer que la courbe $E$ est écrite sous la forme $E_\tau = \C/\Lambda_\tau$, et que l'isogénie de degré $\ell$ s'écrit le la manière suivante:
$$\C/(\Z\oplus \Z\tau) \vers \C/(\Z \oplus \Z\frac{\tau}{\ell}).$$
 On écrit alors des égalités entre séries formelles :
$$A(\tau) = \frac{- E_4}{48}, \quad B(\tau) = \frac{E_6}{864}$$
où $E_4$ et $E_6$ sont les séries d'Eisenstein classiques, et on peut tout écrire explicitement en fonction de $q$ comme dans \cite{Schoof}. Cela donne les formules ci-dessus.

Dans le cas d'un corps fini, on peut utiliser le théorème de Deuring, avec le même fait admis que dans la première section. On sait alors que l'isogénie définie sur $\F_p$ est la réduction de l'isogénie définie sur les complexes, ce qui justifie de réduire dans $\F_p$ les équations donnant $A'$ et $B'$. Il peut arriver que la dérivée $\frac{\partial \Phi_\ell}{\partial Y}$ s'annule modulo $p$; dans ce cas, l'algorithme de calcul du noyau échoue. Dans l'algorithme \ref{alg:mod}, on tente alors de calculer le noyau de l'autre isogénie. Il faudrait être bien malchanceux pour que le second calcul échoue également.

%Remarquons que l'on peut tout à fait utiliser ce calcul pour la fonction {\sc CourbeElliptique} précédente. On pourrait en fait choisir $\lambda$ arbitrairement, mais le coût de ce calcul est de toute façon négligeable.
\v
Pour résoudre l'équation différentielle jusqu'à une précision donnée, on utilise une itération de Newton. On sait déjà en effet que la solution $T$ de l'équation n'a pas de terme constant, et cette itération permet de déterminer une solution dans $k[[X]]/(X^{2^{i+1}})$ à partir d'une solution dans $k[[X]]/(X^{2^i})$. On commence avec la solution $0$ dans $k[[X]]/(X)$. Concrètement, le calcul prend la forme suivante : avec les notations de l'algorithme~\ref{alg:ker}, on définit
$$T_0 = 0, \quad \forall i\geq 0,\ T_{i+1} = T_i + T_i' \sqrt{G} \sqrt{X} \int \frac{T_i(X)}{2\sqrt{X}}.$$
Alors pour tout $i\geq 0$, $T_i$ est une solution de l'équation différentielle modulo $X^{2^i}$.

Les algorithmes de calcul de noyau proposés à l'origine par Stark et Elkies calculent les coefficients de $T$ de proche en proche, ce qui entraînait un coût quadratique en $\ell$. L'idée d'utiliser une itération de Newton pour résoudre cette équation est due à Bostan, Morain, Salvy et Schost \cite{BMSS}. Le coût de cet algorithme devient quasi-linéaire, c'est à dire quasi-optimal, en $\ell$. Remarquons toutefois que le calcul d'une équation normalisée demande encore de manipuler le polynôme modulaire en deux variables, qui est donné par $\frac{(\ell + 1)^2}{2}$ coefficients.

\v

 Pour récupérer le dénominateur, un simple algorithme d'Euclide étendu permet de récupérer les polynômes $N$ et $D$ lorsque l'on connaît les coefficients de la série formelle $\frac{N}{D}$ jusqu'au degré $2\ell + 1$. En effet, si l'on connaît un polynôme $U$ tel que $U = \frac{N}{D} \mod X^{2\ell + 1}$, alors il existe un polynôme $P$ tel que $D U = N + P X^{2\ell+1}$, ce que l'on peut réécrire
 $$D U - P X^{2\ell + 1} = N.$$
On lance alors l'algorithme d'Euclide étendu avec $U$ et $X^{2\ell + 1}$, que l'on arrête dès que les coefficients $N$ et $D$ sont tous deux de degré inférieur à $\ell$, ce qui survient nécessairement au cours du calcul.

Enfin, pour le calcul de racine carrée, on utilise la remarque suivante : si $D = K^2$ où $K$ est un polynôme séparable, alors $K$ et $\mathrm{pgcd}(D, D')$ sont associés. En effet, on a $D' = 2 K K'$ et pgcd$(K, K')$ = 1 puisque $K$ est séparable.
\v

Le coût principal de l'algorithme~\ref{alg:mod} est partagé entre le calcul des racines et l'algorithme {\sc CompareDirection}, qui impliquent tous deux de calculer une puissance $p$-ième modulo un polynôme de degré de l'ordre de $\ell$. C'est une amélioration importante d'un point de vue pratique par rapport au $\ell^2$ précédent. Cependant, on peut proposer encore mieux dans certains cas particuliers.

\subsection{Calcul à l'aide de torsion rationnelle}

Pour certaines valeurs propres spéciales du Frobenius, on peut proposer une autre méthode qui évite complètement le calcul d'une puissance $p$-ième. Par exemple, si 1 est valeur propre modulo $\ell$, cela signifie qu'il existe un sous-groupe cyclique d'ordre $\ell$ constitué de points rationnels sur $k$ (et un seul, car les valeurs propres sont distinctes). On obtient alors l'algorithme \ref{alg:tors}.

\begin{algorithm}
\caption{un pas dans le graphe d'isogénies à l'aide de points de torsion rationnels}
\label{alg:tors}
\KwIn{Une courbe elliptique $E\in \Ell_k(\O)$, et un idéal représenté par $(\ell, v)$ tel que $v = 1$}
\KwOut{Une courbe elliptique $E'$ telle que l'isogénie $E\vers E'$ corresponde à cet idéal}
$P \gets \textsc{PointDeTorsion}(E, \ell)$
\label{alg:tors:tors}
\;
$K \gets \textsc{SousGroupeEngendré}(E, P)$
\label{alg:tors:sg}
\;
$E' \gets \textsc{Quotient}(E, K)$
\label{alg:tors:quo}
\;
\Return{$E'$}

\end{algorithm}

Remarquons qu'en dehors de la recherche du point de torsion, cette méthode est très peu coûteuse. Le coût de la construction du sous-groupe est linéaire en $\ell$, mais indépendant de $p$, et le quotient final est simplement un appel aux formules de Vélu, ce qui est quasiment instantané. La recherche d'un point de torsion rationnel ne pose pas non plus beaucoup de problèmes:

\begin{algorithm}
\caption{{\sc PointDeTorsion} : calcul d'un point de torsion rationnel}
\label{alg:ptors}
\KwIn{Une courbe elliptique $E\in \Ell_k(\O)$ de trace $t$, et un nombre premier d'Elkies $\ell$ tel que $E$ admet un point de $\ell$-torsion rationnel primitif}
\KwOut{Un point de $\ell$-torsion primitif}
$C \gets p + 1 - t$
\label{alg:ptors:card}
\;
\Repeat{$P \neq 0$}{
$Q \gets \textsc{PointAuHasard}(E,k)$
\;
$P \gets \left[\frac{C}{\ell}\right]Q$
\label{alg:ptors:scal}
    }
\Return{$P$}

\end{algorithm}

Nous reviendrons plus tard sur le calcul de la trace $t$. Cette quantité est la même pour toutes les courbes rencontrées au cours du calcul: en effet, selon un théorème de Tate, deux courbes sur $k$ reliées par une isogénie définie sur $k$ ont le même nombre de points sur $k$. La trace $t$ dépend donc uniquement de la courbe d'origine choisie, et n'est ensuite plus recalculée.

Le coût principal de la recherche d'un point de torsion est la multiplication scalaire à l'étape~\ref{alg:ptors:scal}. Pour calculer ce type de multiplications, on utilise un mécanisme de chaîne d'additions (double-and-add par exemple, en utilisant l'expression du cofacteur en base binaire) qui utilise environ $\log(C)$ (c'est à dire $\log(p)$) additions sur la courbe. Cela représente un gain important par rapport aux algorithmes~\ref{alg:div} et~\ref{alg:mod}. Lorsque $\ell$ devient trop grand, l'étape {\sc SousGroupeEngendré} devient coûteuse. Néanmoins on peut s'autoriser des nombres premiers beaucoup plus grands que pour les algorithmes précédents, de l'ordre de 10 000 environ en lieu et place de 500 pour un même coût.
\v

L'algorithme~\ref{alg:tors} n'est utilisable que pour les nombres premiers $\ell$ pour lesquels 1 est valeur propre. Intuitivement, il y a environ une chance sur $\ell - 1$ que cela survienne pour une courbe choisie \og au hasard \fg : c'est donc plutôt rare. On peut donner une variante lorsque les points de $\ell$-torsion ne sont pas nécessairement $k$-rationnels, mais seulement définis sur une extension de $k$ de petit degré. Il faut alors multiplier un point choisi au hasard par $\frac{C'}{\ell}$, où $C'$ est le cardinal de la courbe sur cette extension (qui peut se calculer directement à partir de $C$ par la théorie de Hasse). Lorsque l'extension est de degré $d$, $C'$ est de l'ordre de $p^d$. Associé au fait qu'il est plus coûteux de manipuler des points définis sur une extension, le coût de cet algorithme étendu devient quadratique en $d$: en pratique, il n'est intéressant que lorsque $d<10$ environ. De plus, le fait que $\ell$ soit d'Elkies n'empêche plus les deux sous-groupes d'apparaître simultanément, ce qui ajoute une nouvelle restriction.

En pratique, il est rare qu'une courbe admette beaucoup de nombres premiers pour lesquels l'algorithme \ref{alg:tors} ou une variante peut être utilisé. Pour profiter pleinement l'accélération permise par cette méthode, il faut donc choisir des courbes exceptionnelles, qui ont plus de points de torsion rationnelle de petit ordre que la moyenne.

\v
D'autre part, l'algorithme \ref{alg:tors} permet de voyager dans le graphe d'isogénies uniquement dans la direction associée à la valeur propre 1, et non dans la direction de l'autre valeur propre. On peut proposer trois astuces:
\begin{itemize}
\item[•] Utiliser le modèle de Montgomery permet d'avoir des opérations plus efficaces, ce qui réduit le coût de la multiplication scalaire. On peut ainsi ajouter deux points génériques en 6 additions, 2 mises au carré et 4 multiplications dans $k$, contre 8 additions, 5 multiplications, une mise au carré et une division pour un modèle de Weierstrass \og classique\fg\ non optimisé.

\item[•] Un autre avantage du modèle de Montgomery est de pouvoir travailler uniquement sur la coordonnée $x$. Il devient alors très facile de passer d'une courbe à sa tordue. En pratique, cela revient à dire que l'on tire autant d'avantages d'une valeur propre égale à $-1$ qu'une valeur propre égale à 1.

\item[•] Enfin, une dernière amélioration est de forcer l'égalité $p = -1 \mod\ell$. De ce fait, si $v_1 = 1$ dans une direction, alors on aura $v_2 = -1$ dans l'autre, ce qui garantit une méthode efficace pour pouvoir faire un pas dans les deux sens.
\end{itemize}
\v

En utilisant une méthode générique, on ne peut pas espérer beaucoup mieux que de devoir chercher les racines d'un polynôme de degré $\ell + 1$ sur $\F_p$. En effet, il existe toujours $\ell + 1$ isogénies distinctes de degré $\ell$ au départ d'une courbe donnée qui sont définies sur $\bar{k}$: c'est le nombre de sous-groupes cycliques de taille $\ell$ dans $E(\bar{k})$. Utiliser des points de torsion rationnelle d'ordre $\ell$ permet de contourner ce calcul lorsqu'ils existent. La recherche de courbes ayant beaucoup de tels points devient alors critique pour l'amélioration des performances.

Il est également intéressant de disposer d'une courbe ayant beaucoup de nombres premiers d'Elkies, car on sera amené à manipuler des isogénies de degré moins élevé. Cependant, avant de pouvoir chercher des courbes intéressantes, différents paramètres restent à déterminer

%On peut donner une intuition plus étayée du temps de recherche d'une courbe ayant une valeur propre du %Frobenius égale à 1 pour plusieurs nombres premiers.
%
%\begin{prop}
%Soit $p>3$ un nombre premier, et $\ell\neq p$ un nombre premier tel que $p\neq 1 \mod\ell$. Alors on a
%$$ \#'\left\{E/\F_p \right\}/\mathrm{isom.} = p.$$
%De plus, il existe une constante explicite $C$ telle que
%$$ \left| \frac{p}{\ell - 1}  - \#' \left\{E/\F_p\ :\ \#E(\F_p) = 0 \mod\ell\right\}/\mathrm{isom.} \right|%\leq C\ell\sqrt{p}.$$
%\end{prop}
%
%Dans ce résultat, $\#'$ désigne la cardinalité \emph{pondérée}, c'est à dire que la classe d'isomorphisme d'une courbe $E$ est comptée avec le poids $\frac{1}{\#\mathrm{Aut}(E)}$. Cette correction correspond aux seules courbes de $j$-invariant 0 ou 1728. Ainsi, la proportion de courbes ayant un point rationnel de $\ell$-torsion lorsque $p\neq 1 \mod \ell$ est $\frac{1}{\ell - 1} + O(\frac{\ell}{\sqrt{p}}).$
%
%Bien sûr, le fait d'admettre de la $\ell_1$-torsion rationnelle et de la $\ell_2$-torsion rationnelle ne sont pas des événements indépendants, mais on peut disposer d'un résultat analogue lorsque $\ell$ n'est pas premier. Ce qui ressort de cette proposition, ou même de l'intuition grossière, est qu'il sera très difficile de trouver une courbe ayant toutes les bonnes propriétés que l'on pourrait souhaiter. Par conséquent, la recherche d'une bonne courbe initiale devient critique pour l'amélioration des performances de ce cryptosystème.
%
%Pour assurer qu'il n'y ait qu'un seul groupe cyclique d'ordre $\ell$ défini sur cette extension minimale, il faut alors demander la %propriété suivante :
%
%\begin{defi}
%Soit $E$ une courbe elliptique sur $k$ et $\ell$ un nombre premier. On dit que $\ell$ est \emph{adapté} si $\ell$ est un nombre premier %d'Elkies et si les ordres multiplicatifs des valeurs propres du Frobenius modulo $\ell$ sont distincts.
%\end{defi}
%
%De plus, utiliser l'algorithme~\ref{alg:tors} n'est intéressant que lorsque le degré de l'extension est suffisamment petit, c'est à dire %lorsque le minimum de ces deux ordres est petit. Pour rechercher des courbes ayant beaucoup de nombres premiers adaptés, on utilise une %variante de l'algorithme utilisé pour calculer le nombre de points d'une courbe elliptique sur un corps fini, que l'on peut présenter %sous la forme suivante.
%
%
%\begin{algorithm}
%\caption{{\sc EstAdapté} : décide si un nombre premier est adapté à une courbe}
%\label{alg:adapt}
%\KwIn{Une courbe elliptique $E$, et un nombre premier $\ell$}
%\KwOut{{\bf True} si $\ell$ est adapté, {\bf False} sinon}
%$\Phi_\ell \gets \textsc{PolynômeModulaire}(\ell)$
%\label{alg:adapt:polmod}
%\;
%$(j_1,j_2) \gets \textsc{Racines}(\Phi_\ell(j(E), Y), k)$
%\label{alg:adapt:roots}
%\;
%$(K_1, K_2) \gets (\textsc{Noyau}(E,\ell,j_1), \textsc{Noyau}(E,\ell,j_2))$
%\label{alg:adapt:ker}
%\;
%$(v_1, v_2) \gets (\textsc{ValeurPropre}(E, K_1), \textsc{ValeurPropre}(E, K_2))$
%\label{alg:adapt:eigen}
%\;
%$(o_1, o_2) \gets (\textsc{Ordre}(v_1),\textsc{Ordre}(v_2))$
%\label{alg:adapt:ord}
%\;
%\Return{$o_1\neq o_2$}
%\end{algorithm}
%
%L'étape~\ref{alg:adapt:eigen} est analogue à l'algorithme {\sc CompareDirection} précédent : on teste simplement chacun des éléments de %$\F_\ell$ les uns après les autres jusqu'à trouver la valeur propre. Enfin, comme $\ell \ll p$ dans notre utilisation, on peut se %contenter d'une recherche exhaustive pour la fonction {\sc Ordre}.
%
%Procéder ainsi se révèle beaucoup moins coûteux que de calculer le nombre de points de $E$, puis décider si $\ell$ est adapté ou non en %observant la factorisation de $X^2 - tX + q$. Cela permet de tester beaucoup de courbes, en les évacuant rapidement s'il n'y a pas assez %de nombres premiers adaptés parmi les petites valeurs.
%
%Une fois que l'on a trouvé une courbe qui semble intéressante, on calcule son cardinal entièrement à l'aide de l'algorithme de Schoof--Atkin--Elkies \cite{Schoof}. L'idée de cet algorithme polynomial est de calculer la trace du Frobenius modulo plusieurs petits nombres premiers grâce à un calcul comme ci-dessus, puis de récupérer la trace entière grâce au théorème chinois, connaissant les bornes de Hasse. Enfin, on regarde quels nombres premiers sont adaptés parmi les plus grandes valeurs en regardant la factorisation de $X^2 - tX + q$.

\newpage
\section{Attaques et recherche de paramètres}

Plusieurs paramètres restent à fixer dans le cryptosystème décrit à la section précédente. D'une part, la taille de $k$, qui conditionne la taille du graphe d'isogénies et donc celle du groupe de classes. D'autre part, il reste également à définir quels idéaux utiliser pour se déplacer dans le graphe d'isogénies. Plus précisément, il faut fixer un ensemble $L$ de petits premiers d'Elkies, et une borne $M_\ell$ pour tout $\ell\in L$, pour utiliser la méthode de génération aléatoire suivante dans le groupe de classes:
\begin{itemize}
\item[•] Pour tout $\ell\in L$, choisir un entier $- M_\ell\leq n_\ell\leq M_\ell$ uniformément;
\item[•] Retourner l'élément $\prod_{\ell\in L} {\frak l}^{n_\ell}$ du groupe de classes, où $\frak l$ est un idéal de $\O$ de norme $\ell$ choisi arbitrairement (autrement dit, une valeur propre du Frobenius choisie arbitrairement).
\end{itemize}

Ces paramètres, comme leurs analogues dans d'autres protocoles cryptographiques, doivent être choisis de telle sorte à assurer une bonne sécurité contre les meilleures attaques connues. Déterminer ces paramètres permettra également de mesurer les véritables performances de ce protocole. En effet, cette mesure n'a de sens qu'à un niveau de sécurité donné: par exemple, un intérêt de beaucoup de protocoles utilisant des courbes elliptiques par rapport au RSA classique est de pouvoir utiliser des groupes plus petits pour atteindre le même niveau de sécurité.

\v
Pour quantifier cette sécurité, on parle de \emph{bits}. On dit qu'un protocole présente $n$ bits de sécurité si la meilleure attaque connue nécessite au moins $2^n$ opérations pour réussir avec une probabilité non négligeable. Même sans préciser ce que signifie \og opération \fg, c'est une façon agréable de quantifier le concept de sécurité. En informatique, on aime les puissances de 2, c'est pourquoi les niveaux de sécurités standard sont aujourd'hui 128, 192 ou 256 bits de sécurité (ce dernier étant tout de même un peu paranoïaque). L'objectif de cette partie est de déterminer des paramètres de notre cryptosystème et d'en déduire ses performances réelles au niveau d'environ 128 bits de sécurité.


\subsection{L'attaque classique}


Commençons donc par décrire les attaques existantes. Rappelons les données du problème:
\begin{itemize}
\item[•] Les éléments publics sont des courbes elliptiques $E_0$, $E_a$ et $E_b$, ainsi que la liste $L$ des degrés d'isogénies utilisés et les bornes $M_\ell$ correspondantes.
\item[•] Les éléments secrets sont les éléments du groupe de classe $a, b$ tels que $E_a = a\cdot E_0$ et $E_b = b\cdot E_0$.
\end{itemize}

Une façon d'attaquer le protocole est de chercher à déterminer $a$ à partir des données publiques $E_0$ et $E_a$. Autrement dit, il s'agit de chercher une isogénie entre deux courbes données; en termes de graphe d'isogénies, il s'agit de retrouver un chemin dans un graphe reliant deux points donnés.

\v
Pour cela, on peut construire une attaque fondée sur le \og birthday paradox\fg. Cette attaque a été proposée par Galbraith dans \cite{Galbraith}, qui est la référence principale de cette section et dont sont tirés tous les résultats cités ici. Le résultat principal en est le suivant:

\begin{thm}
Soient $E_1, E_2$ des courbes elliptiques ordinaires définies sur $\F_p$ telles que $\#E_1(\F_p) = \#E_2(\F_p)$. En admettant l'hypothèse de Riemann pour les corps quadratiques, et sous heuristiques, il existe un algorithme probabiliste construisant une isogénie rationnelle entre $E_1$ et $E_2$. Le coût de cet algorithme est dans le pire des cas $O(p^{3/2}\ln(p))$ en temps et $O(p \ln(p))$ en espace.
\end{thm}

Le principe de cet algorithme est de construire deux arbres dans le graphe d'isogénies enraciné l'un en $E_1$, l'autre en $E_2$. Pour construire ces arbres, on calcule l'action d'un groupe de classes avec les algorithmes de la section précédente. Lorsque deux branches entrent en collision, on dispose d'un chemin entre $E_1$ et $E_2$, ce qui survient avec une bonne probabilité une fois que le nombre de branches est de l'ordre de $\sqrt{h}$, où $h$ est le nombre de classes. L'hypothèse de Riemann intervient ici pour s'assurer de pouvoir parcourir tout le groupe de classes avec des idéaux dont la norme est inférieure à une borne explicite. Cet algorithme repose de plus sur une hypothèse heuristique, à savoir que les branches de chaque arbre ne se recouvrent pas trop: autrement dit, les produits de la forme
$$\prod_{i = 1}^n {\frak l}_i^{a_i}, \quad a_i\in \{-1,0,1\}$$
se répartissent de manière raisonnablement uniforme dans le groupe de classes pour une grande proportion de suites $({\frak l}_i)$.

Avant de construire ces arbres, toutefois, il faut s'assurer de disposer de deux courbes ayant le même anneau d'endomorphismes, et ce n'est pas nécessairement le cas des deux courbes d'origine. Une solution consiste à utiliser des isogénies dont le degré divise le conducteur pour se ramener tout d'abord à des courbes dont l'anneau d'endomorphismes est maximal: les propriétés de ces isogénies \og exceptionnelles \fg\ sont décrites par exemple dans \cite{Kohel}. Cette étape est la plus coûteuse dans le pire des cas.
\v

L'attaque de Galbraith peut être simplifiée dans notre cadre. On sait déjà en effet que les deux courbes ont le même anneau d'endomorphismes, et que seuls des premiers de la liste $L$ ont été utilisés pour voyager entre elles. Cela permet d'éviter de se ramener à l'anneau d'endomorphismes maximal, et on peut aussi profiter du fait que les idéaux dont les normes apparaissent dans $L$ n'engendrent pas nécessairement la totalité du groupe de classes. On peut donc reformuler l'attaque ainsi.

\begin{prop}
Avec les notations précédentes, il existe un algorithme probabiliste qui, étant donnés $E_0$, $E_a$ et $L$, construit une chaîne d'isogénies entre $E_0$ et $E_a$ dont les degrés sont choisis dans $L$. En notant $N$ le cardinal du sous-groupe du groupe de classes engendré par les idéaux dont la norme figure dans $L$, le coût de cet algorithme est en moyenne $\Theta(\sqrt{N}\, \ln(N) \ln(p)^5)$.
\end{prop}

Remarquons que l'hypothèse heuristique n'est plus nécessaire, car les bornes $M_\ell$ donnent précisément le nombre de branches à construire pour chaque premier. Le facteur $\ln(p)^5$ reflète le coût des calculs d'isogénies comme ils ont été décrits à la section précédente, en utilisant les polynômes modulaires.

\v
Cette attaque jouant le rôle de la meilleure connue, le choix des paramètres $p$ et $L$ doit être fait de telle sorte que la quantité $N$ de la proposition précédente soit de l'ordre de 250 bits, c'est à dire $2^{250}$. 



\subsection{Choix des paramètres}

On a vu que la sécurité du cryptosystème dépend de manière critique du cardinal d'un certain sous-groupe du groupe de classes. Trois questions se posent alors:
\begin{itemize}
\item[•] Comment peut-on contrôler, en fonction de $p$, la taille du groupe de classes ? Plus précisément, connaissant $p$ et la trace $t$ du Frobenius la courbe $E_0$, peut-on contrôler le cardinal du groupe de classes des ordres de $K = \Q(\pi)/(\pi^2 - t\pi + p)$ ?
\item[•] Pour un ordre $\O$ donné, comment peut-on déterminer une borne $M$ telle que l'ensemble des idéaux inversibles de $\O$ de norme inférieure à $M$ engendre un grand sous-groupe du groupe de classes ?
\item[•] Une fois cette borne définie, peut-on donner pour chaque idéal ${\frak l}_i$ de norme $\ell_i\leq M$, une borne $M_i$ telle que les produits
$$\prod_{i} {\frak l}_i^{a_i}, \quad - M_i\leq a_i\leq M_i$$
se répartissent à peu près uniformément dans ce sous-groupe ? Plus généralement, peut-on donner un critère pour que des bornes $M_i$ soient convenables ?
\end{itemize}

Ces questions d'arithmétiques sont complexes, et souvent les seules réponses que l'on sait apporter sont heuristiques. Lorsque des théorèmes existent, il s'agit de théorie analytique des nombres, et leurs preuves sont hors de portée de ce mémoire.
\v

\textbf{Contrôle du nombre de classes.} Un premier élément est qu'au sein d'un même corps quadratique imaginaire, le nombre de classes augmente à mesure que le conducteur augmente. On peut en trouver une preuve chez Lang \cite{Lang}, th. 8.7.

\begin{prop}
Soit $\O$ un ordre de conducteur $c$ dans un corps quadratique imaginaire $K$. Notons $h_\O$ et $h_K$ le nombre de classes de $\O$ et $\O_K$ respectivement. Alors on a l'égalité
$$h_\O = h_K c\, [\O_K^*:\O^*] \prod_{p|c} \left(1 - \left(\frac{K}{p}\right)p^{-1}\right)$$
où l'on note $\left(\frac{K}{p}\right) = -1, 0, 1$ selon si $p$ est inerte, ramifié ou scindé dans $K$, respectivement.
\end{prop}

Ainsi le nombre de classes minimal est atteint pour l'anneau d'entiers: $h_\O$ est de l'ordre de $h_K c$. Cela amène à réviser la \og meilleure attaque\fg: il peut être intéressant de remonter à des courbes dont l'anneau d'endomorphismes est l'anneau d'entiers même en ayant au départ l'information que les anneaux d'endomorphismes sont les mêmes. Cela pousse à ne pas profiter de l'augmentation du nombre de classes avec le conducteur: autrement dit, on cherche de bons paramètres dans le cas où l'ordre d'endomorphismes est $\O_K$.

D'autre part, on dispose de plusieurs résultats sur le nombre de classes d'un corps quadratique imaginaire. Tout d'abord, on peut facilement le majorer: si $K$ est un corps quadratique imaginaire de discriminant $D_K$, alors
$$h_K \leq \frac{1}{\pi} \sqrt{|D_K|}\, \ln(|D_K|).$$
On peut également dire des choses pour la minoration de ce nombre de classes. Le théorème de Brauer--Siegel donne une première indication, malheureusement non effective:
\begin{thm}[Brauer--Siegel pour les corps quadratiques imaginaires]
Soit $(K_i)_{i\in \N}$ une suite de corps quadratiques imaginaires de discriminants respectifs $D_i$ et de nombres de classes $h_i$. On suppose que
$|D_i| \underset{i\to\infty}{\vers} \infty.$
Alors
$$\frac{\log(h_i)}{\log(\sqrt{|D_i|})} \underset{i\to\infty}{\vers} 1.$$
\end{thm}

Ce résultat était en réalité connu avant les travaux de Brauer et Siegel, qui l'ont généralisé à des corps de nombres généraux. Stark a ensuite donné des versions effectives de ce théorème, ce qui amène à la conclusion (heuristique) suivante: si $K$ est un corps quadratique de discriminant $D_K$ et de nombre de classes $h_K$, et si $D_K$ est de l'ordre de $2^{512}$, alors $h_K$ est au moins de l'ordre de $2^{250}$; autrement dit, l'anneau d'entiers de $K$ fournit une bonne sécurité au vu de l'attaque précédente.

\v
Dans notre contexte, $K$ est obtenu comme le quotient $\Q[\pi]/(\pi^2  - t\pi + p)$. Le discriminant de l'ordre $Z[\pi]$ de $K$, engendré par l'endomorphisme de Frobenius, est $D = t^2 - 4 p$ et est donc en général de l'ordre de grandeur de $p$. Le discriminant $D_K$ de $K$ (qui est celui de $\O_K$) est alors l'entier dans facteur carré qui apparaît dans la décomposition $D = c^2 D_K,$ et $c$ est le conducteur de $\Z[\pi]$. Ainsi, déterminer le discriminant de $K$ revient à déterminer le facteur carré maximal de $D$.

Lorsque $D$ est un entier de l'ordre de 512 bits, factoriser $D$ est une opération difficile. Cela signifie que pour un discriminant donné, nous ne sommes pas capables de déterminer le discriminant $D_K$ de l'anneau d'entiers de $K$ à moins d'y consacrer des ressources non négligeables.

On a cependant tendance à penser qu'un entier choisi \og au hasard\fg\ a peu de chances de présenter un grand facteur carré; en moyenne, le conducteur de $\Z[\pi]$ sera donc, au pire, un petit entier. On peut donc espérer que ce sera le cas pour la courbe elliptique $E_0$ que nous finirons par choisir dans ce mémoire. On laisse donc ce problème de côté pour l'instant, et l'on choisira un nombre premier $p$ de 512 bits, de sorte que $D$ et donc $D_K$ soient de cet ordre également. Il faudrait vérifier que le conducteur de la courbe choisie est bien un petit entier, mais ce n'est pas fait ici.

\v
\textbf{Générateurs du groupe de classes.}
Soit $K$ un corps quadratique imaginaire de discriminant $D$, et soit $\O_K$ son anneau d'entiers. En admettant l'hypothèse de Riemann pour $K$, les idéaux de $\O_K$ dont la norme est inférieure à $6\ln(|D|)^2$ engendrent le groupe de classe de $\O_K$. Cela donne la borne cherchée.

Pour un nombre premier $p$ de 512 bits, utiliser cette borne amènerait à calculer des isogénies de degré près d'un million, ce qui est déraisonnable. En pratique, il est tout à fait possible que le nombre minimal de générateurs soit très inférieur à cette quantité: par exemple, un générateur suffit lorsque le groupe de classes est de cardinal premier. On considèrera donc que si $K$ un corps quadratique imaginaire, et $\O$ un ordre de $K$ de discriminant $D$ de notre ordre de grandeur, alors les idéaux inversibles de $\O$ dont la norme est inférieure à $\ln(|D|)$ engendrent $\Cl(\O)$, ou au moins un grand sous-groupe de celui-ci.

\v
Pour étudier la répartition des produits d'idéaux, on peut réfléchir à nouveau en termes de graphe d'isogénies. La répartition des produits d'idéaux dans le groupe de classes est alors directement reliée à la dispersion d'une certaine marche aléatoire dans ce graphe, et on sait que les propriétés d'\emph{expansion} de certains graphes garantissent la dispersion rapide des marches aléatoires. En admettant l'hypothèse de Riemann à nouveau, on peut montrer que le graphe d'isogénies satisfait ce type de propriété. 

Quelques définitions: pour un graphe non orienté $G$, $k$-régulier (c'est à dire que tout sommet admet $k$ voisins), on définit l'opérateur d'adjacence $A$ de la façon suivante:
$$A(f)\de x\longmapsto \sum_{y \text{ voisin de } x} f(y).$$
La fonction constante 1 est un vecteur propre de cet opérateur associé à la valeur propre $k$, dite valeur propre triviale; c'est la plus grande en valeur absolue. Le graphe $G$ est dit $\delta$-expanseur si les autres valeurs propres $\lambda$ de $A$ vérifient $|\lambda|\leq (1-\delta) k$. Lorsque $G$ est $\delta$-expanseur, les marches aléatoires dans $G$ de longueur $\frac{1}{\delta} \log(|G|)$ (où $|G|$ est le nombre de sommets) se répartissent uniformément dans le graphe, au sens où la probabilité de terminer dans un ensemble de sommets $S$ est au moins proportionnelle à $\#S$ (uniformément en $S$, bien sûr). Le résultat suivant est tiré de \cite{GRH}.

\begin{thm}
Soit $E_0/\F_p$ une courbe elliptique ayant multiplication complexe par $\O$. On se donne $B>2$. Soit $G$ le graphe dont les sommets sont les éléments de $\Ell_{\F_p}(\O)$, et dont deux sommets sont reliés par une arête s'ils sont reliés par une isogénie de degré inférieur à $(\log(4p))^B$. Alors $G$ est un graphe expanseur, dans le sens où les valeurs propres de l'opérateur d'adjacence vérifient
$$|\lambda| = O\left((\lambda_{\text{triv}} \log(\lambda_{\text{triv}}))^{\frac{1}{2} + \frac{1}{B}}\right).$$
Par conséquent, on peut donner une constante $C>0$ telle que lorsque $p$ est assez grand, une marche aléatoire de longueur au moins
$$C \frac{\log(h_\O)}{\log\log p}$$
termine dans un sous-ensemble fini $S$ de $\Ell_{\F_p}(\O)$ avec probabilité au moins $\frac{1}{2} \frac{\#S}{h_\O}$.
\end{thm}

L'hypothèse de Riemann intervient ici car les autres vecteurs propres de l'opérateur d'adjacence $A$, lorsque $G$ est le graphe de Cayley d'un groupe $H$, sont des caractères de $H$.

On peut faire deux remarques sur ce résultat. Premièrement, on ne veut certainement pas calculer des isogénies de degré $(\log p)^2$, mais seulement $\log(p)$. Lorsque l'on durcit cette contrainte sur les idéaux considérés, le résultat ci-dessus ne s'applique plus, mais il semble plausible qu'une propriété d'expansion soit toujours vérifiée. Deuxièmement, on ne considèrera pas véritablement des marches alétoires uniformes dans le graphe ci-dessus: en effet, les isogénies de degré plus élevé sont en général plus difficiles à évaluer que les isogénies de petit degré. Ainsi, on aimerait effectuer plus de pas dans le graphe en utilisant les petits degrés que les grands, pour minimiser le coût total de l'opération.

On propose donc une hypothèse optimiste, mais intuitivement plausible, sur la répartition des produits d'idéaux dans notre cadre. 

\begin{hyp}
Avec les notations précédentes, soit $L$ un ensemble de nombres premiers, et $N$ le cardinal du sous-groupe de $\Cl(\O)$ engendré par les idéaux dont la norme figure dans $L$. Pour tout $\ell\in L$, on se donne une borne $M_\ell$ et l'on suppose que
\begin{itemize}
\item[•] Pour tout $\ell\in L$, on a $M_\ell \leq \ln(N)$;
\item[•] On a $\prod_{\ell\in L} (2M_\ell + 1) \geq N$.
\end{itemize}
Alors la plupart du temps, les produits $\prod {\frak l}^{a_\ell}$ pour $-M_\ell \leq a_\ell\leq M_\ell$ se répartissent de manière environ uniforme dans le sous-groupe, et $N$ est proche de $h_\O$.
\end{hyp}

Ainsi, on choisira un nombre premier $p$ de 512 bits, une liste $L$ de longueur environ 100 et des bornes $M_\ell$ vérifiant $\prod (2 M_\ell +1) \geq 2^{256}$. Montrer la sécurité de ce genre de paramètre semble impossible, mais utiliser les bornes prouvées données ci-dessus n'est pas raisonnable; on se contente donc de l'approche heuristique.

\v
Après cette discussion, on peut poser
$$\begin{aligned}
p =\ &120373407382088450343833839782228011370920294512701979230713977354082515866699 \\ &38291587857560356890516069961904754171956588530344066457839297755929645858769,
\end{aligned}
$$
un nombre premier de 512 bits congru à $-1$ modulo tous les nombres premiers compris entre 2 et 380. Notons que le produit de ces premiers est déjà un nombre de 509 bits. Afin de pouvoir utiliser le plus possible la méthode rapide de calcul d'isogénie, on souhaite maintenant exhiber une courbe elliptique $E/\F_p$ admettant:
\begin{itemize}
\item[•] Un modèle de Montgomery,
\item[•] Beaucoup de nombres premiers d'Elkies,
\item[•] Beaucoup de points de torsion rationnels de petit ordre.
\end{itemize}

Malheureusement, trouver une telle courbe est un problème difficile. Par exemple, étant donné un entier $C$ compris entre les bornes de Hasse, on ne sait pas exhiber une courbe de cardinalité $C$, bien que l'on puisse démontrer son existence. On peut toutefois faire un peu mieux que tester des courbes au hasard et observer leurs propriétés, comme on va le voir maintenant.

\subsection{Recherche de courbes}

On donne tout d'abord une stratégie (très) naïve pour trouver ce genre de courbe sur $k = \F_p$:
\begin{itemize}
\item[•] Choisir $a_4$ et $a_6$ au hasard dans $k$;
\item[•] Définir la courbe $E\de y^2 = x^3 + a_4x + a_6$;
\item[•] Regarder si $E$ a un modèle de Montgomery;
\item[•] Si oui, calculer le nombre de points $C$ de $E$ sur $k$ (ou sa trace $t$, de manière équivalente);
\item[•] Pour les premiers $\ell\leq 500$, regarder si $t^2 - 4p$ est un carré non nul modulo $\ell$ et si $C = 0 \mod\ell$;
\item[•] Estimer le temps nécessaire à un parcours dans le graphe d'isogénies en fonction des résultats.
\end{itemize}

Pour que $E$ admette un modèle de Montgomery, il faut que $E$ admette un point rationnel de 2-torsion, donc que le polynôme $X^3 + a_4 X + a_6$ ait une racine dans $\F_p$. La courbe $E$ admet un modèle de Montgomery si, et seulement si, il existe une racine $\alpha$ de ce polynôme telle que le polynôme dérivé $3\alpha^2 + a_4$ soit un carré dans $\F_p$.

Bien sûr, le temps nécessaire pour effectuer un parcours dépend de l'implémentation choisie, et du choix du nombre de pas $M_\ell$ pour chaque nombre premier $\ell$ utilisé. On donnera des valeurs précises du temps d'exécution à la section suivante. Notons que même en connaissant le temps de calcul pour chaque premier, il est difficile de trouver les bornes $M_\ell$ optimales: en pratique, on se donne un certain temps $T$ que l'on accepte de dépenser pour chaque premier, on calcule alors la borne $M_\ell$ associée pour chaque $\ell$, et on obtient une majoration du temps de parcours lorsque le produit des $M_\ell$ devient assez grand.

\v
Pour déterminer les propriétés de $E[\ell]$ pour de petits premiers $\ell$, cette première stratégie nécessite de calculer la cardinalité d'une courbe sur $\F_p$. Pour cela, on utilise l'algorithme de Schoof, amélioré par Atkin et Elkies \cite{Schoof}. L'idée de cet algorithme polynomial est de calculer la trace $t$ modulo $\ell_i$, pour des nombres premiers $\ell_i$ tels que $\prod \ell_i > 4\sqrt{p}$. Connaissant les bornes de Hasse $|t|\leq 2\sqrt{p}$, on peut ensuite déterminer uniquement la trace dans $\Z$ modulo le théorème chinois.

Pour déterminer $t \mod \ell_i$, l'idée de Schoof est de regarder le Frobenius agissant sur l'espace $E[\ell_i](\bar{k})$. On a vu que c'est un endomorphisme de cet $\F_{\ell_i}$-espace vectoriel de dimension 2, dont la trace est $p\mod \ell_i$ et le déterminant $p$. On peut donc calculer le $\ell_i$-ième polynôme de division $\psi_i$ de la courbe $E$, et se demander pour quel élément $t$ l'égalité
$$(X^{p^2}, Y^{p^2}) - t (X^p, Y^p) + [p] (X, Y)$$
a lieu, où $(X, Y)$ est le point canonique de $E$ sur l'anneau $k[X, Y]$ quotienté par $\psi_i$ et l'équation de $E$. Notons que les sommes ci-dessus sont des sommes de poins sur une courbe elliptique. 

Cet algorithme est certes polynomial en $\log p$, mais n'est pourtant pas très praticable pour de grandes valeurs de $p$. L'amélioration proposée par Elkies est essentiellement celle qui a déjà été utilisée plus tôt dans ce document: lorsque $\ell_i$ est un nombre premier d'Elkies, on peut utiliser une équation modulaire pour travailler avec des polynômes de degré $\ell_i$ plutôt que $\ell_i^2$. En effet, on peut alors calculer une isogénie comme précédemment, calculer son noyau, puis calculer la valeur propre $v_i$ du Frobenius sur celui-ci. On saura alors que $t = v_i + \frac{p}{v_i} \mod \ell_i$.

L'amélioration proposée par Atkin concerne les autres premiers. Elle ne permet pas de déterminer directement $t \mod \ell_i$, mais détermine un petit ensemble de valeurs parmi lesquels $t \mod\ell_i$ doit se trouver. On ne peut alors plus appliquer directement le théorème chinois, et il faut effectuer une recherche supplémentaire pour déterminer la valeur exacte de $t$. Cette dernière recherche n'est pas un algorithme polynomial, mais cela améliore tout de même les performances en pratique.

L'algorithme de Schoof--Atkin--Elkies (SEA) a été bien étudié, et des implémentations matures sont aujourd'hui disponibles. Nous avons utilisé PARI \cite{PARI} pour ce calcul, à travers l'interface offerte par Sage \cite{Sage}. Pour le nombre premier $p$ choisi plus haut, une à deux minutes sont nécessaires environ pour calculer le cardinal d'une courbe sur $\F_p$.

\v
Revenons à la stratégie naïve donnée plus haut. Une première remarque est que l'on pourrait éviter le test du modèle de Montgomery, en tirant dès le début une courbe elliptique sous forme de Montgomery. On peut reformuler cela en disant que l'on dispose d'une paramétrisation (évidente) qui fournit des courbes sous forme de Montgomery.

Une question naturelle est alors la suivante: peut-on donner des paramétrisations qui donnent des courbes elliptiques ayant un point rationnel de $\ell$-torsion, pour certains nombres premiers $\ell$? On a vu que les courbes elliptiques munies d'un tel point sont des points de la courbe modulaire $Y_1(\ell)$; plus généralement on peut considérer des courbes $Y_1(N)$ où $N$ est un produit de nombres premiers distincts. La question est donc de savoir si l'on peut donner une paramétrisation de $Y_1(N)$. Cela n'est possible que si la courbe $X_1(N)$ est de genre zéro. Or, selon \cite{Coreens}:

\begin{prop}
Soit $N\geq 1$ un entier. Alors le genre de $X_1(N)$ est 
$$  g(N) = 
\begin{cases}
0 &\text{si}\ 1\leq N\leq 4 \\
\displaystyle 1 + \frac{N^2}{24} \prod_{\substack{p|N \\ p\ \mathrm{premier}}} \left(1 - \frac{1}{p^2}\right) + \sum_{\substack{d|N\\ d\geq 0}} \varphi(d)\varphi\left(\frac{N}{d}\right)& \text{sinon,}
\end{cases}$$
où $\varphi$ désigne la fonction indicatrice d'Euler.
\end{prop}

Par conséquent, le genre de $X_1(N)$ est nul lorsque $N\leq 10$ ou $N = 12$. Pour ces valeurs, on peut en effet exhiber des paramétrisations de $Y_1(N)$ pour lesquelles on est capable d'expliciter le morphisme
$$j \de Y_1(N) \vers Y(1)$$
afin de pouvoir retrouver la courbe elliptique correspondante. Remarquons qu'utiliser une paramétrisation de $Y_1(12)$, par exemple, est strictement supérieur à utiliser directement le modèle de Montgomery: en effet, une courbe ayant un point de 4-torsion admet toujours un modèle de Montgomery qui est facile à déterminer, et on gagne en plus la présence d'un point rationnel de 3-torsion.

Bien sûr, plus le niveau $N$ est grand, plus l'information gagnée a de la valeur, ce qui pousse à utiliser des courbes modulaires pour lesquelles aucune paramétrisation n'est disponible. Lorsque l'on connaît une équation plane $g(x, y) = 0$ de $Y_1(N)$, on peut tout de même générer des points de cette courbe modulaire sur $\F_p$ par le procédé suivant: on tire $x\in \F_p$ au hasard, puis on recherche les racines du polynôme $g(x, Y)$ dans $\F_p$. Cela a de bonnes chances de donner un résultat, puisqu'un polynôme à coefficients sur $\F_p$ a de bonnes chances d'admettre une racine.

À quel point cette méthode est-elle plus avantageuse que la recherche de courbes au hasard? Lorsque l'on tire une courbe au hasard, elle a environ une chance sur $N$ d'admettre un point rationnel de $N$-torsion, et vérifier sa présence a un coût essentiellement proportionnel à $N$: en utilisant une équation modulaire, le calcul coûteux est la recherche de racines d'un polynôme de degré $N + 1$. Le coût total est donc essentiellement quadratique. En utilisant une équation de $Y_1(N)$, il faut en premier lieu la stocker, et ensuite chercher les racines d'un polynôme de degré $d$, le degré de l'équation $g$ en la variable $y$. Plus ce degré est faible, plus l'équation est petite et plus cette méthode est intéressante.

On arrive donc à la notion de \emph{gonalité}. On appelle gonalité d'une courbe algébrique $X$ sur un corps algébriquement clos $K$ le plus petit entier $k\geq 1$ tel qu'il existe une application rationnelle de degré $k$ de $X$ vers la droite projective. Autrement dit, c'est le plus petit entier $k$ tel qu'il existe une extension $K(X)/K(f)$ de degré $k$ pour un certain $f\in K(X)$. La gonalité mesure donc un défaut de rationalité, et les courbes de gonalité 1 sont exactement les courbes birationnelles à $\P^1$.

Lorsque $X$ est donné par une équation plane, la gonalité est majorée par le degré de l'équation en chacune de ses variables. Or, les résultats connus sur la gonalité de $X_1(N)$ sont décourageants.

\begin{prop}
La gonalité $k$ de $X_1(N)$ vérifie l'inégalité $\frac{21}{100}(g_1(N)-1)\leq k.$
\end{prop}

Dans ce résultat, tiré de \cite{gon}, $g_1(N)$ est le genre de $X_1(N)$, qui est quadratique en $N$ vu la proposition précédente. Une équation plane de $X_1(N)$ est donc nécessairement de degré quadratique en chacune de ses variables: le stockage seul de cet équation est trop coûteux pour envisager de l'utiliser pour de grandes valeurs de $N$.

La situation est sensiblement la même pour les courbes modulaires $X_0(N)$. Il s'agit donc de trouver le juste milieu entre les très petites valeurs, pour lesquelles utiliser les courbes modulaires est intéressant, et les valeurs plus élevées où le coût explose. En pratique, on peut utiliser les courbes $X_0(30)$ ou $X_1(17)$, de gonalité respectivement 2 et 4, et dont il faut alors calculer une équation.

\v
On peut apporter une autre amélioration à la stratégie naïve du début de cette section. En effet, au cours de l'algorithme SEA, on peut choisir de s'arrêter lorsque les valeurs de $t \mod \ell$ pour de petites valeurs de $\ell$ ne sont pas satisfaisantes, ou s'il y a trop peu de nombres premiers d'Elkies. On obtient alors l'algorithme de recherche que nous avons utilisé en pratique:
\begin{itemize}
\item[•] Générer des courbes candidates $E$ à l'aide d'une équation modulaire du bon niveau;
\item[•] Regarder si la courbe admet un modèle de Montgomery;
\item[•] Si oui, lancer l'algorithme SEA et s'arrêter s'il n'y a pas de torsion rationnelle pour de petits premiers;
\item[•] Une fois le cardinal calculé, estimer le temps nécessaire à un parcours dans le graphe d'isogénies en fonction des résultats.
\end{itemize}

Le nombre de conditions à satisfaire dans l'étape 3 est déterminé empiriquement avec l'intuition suivante: l'algorithme sera le plus efficace s'il passe autant de temps à générer des candidats, vérifier les premières conditions et calculer la cardinalité en entier, sachant que moins de courbes sont concernées à mesure que l'on descend dans la liste ci-dessus. Par exemple, si l'on utilise la courbe modulaire $X_1(17)$,
on peut tester la présence de $\ell$-torsion rationnelle pour $\ell\in\{3, 5, 7, 11, 13\}$.

\v

Voyons maintenant comment calculer une équation pour ces courbes modulaires, ainsi que l'expression de la fonction $j$ sur ces courbes.

\subsection{Calcul d'une équation de $X_0(30)$}

L'espace des formes modulaires paraboliques de poids 2 et de niveau $\Gamma_0(30)$ est de dimension 3, et une base de cette espace est donnée par
$$\begin{aligned}
f_1 &= q - q^4 - q^6 - 2q^7 + q^9 + q^{10} - 2q^{11} + 2q^{12} + 2 q^{14} +\ldots\\
f_2 &= q^2 - q^4 - q^6 - q^8 + q^{10} + q^{12} + 3q^{16} + q^{18} - q^{20} - \ldots \\
f_3 &= q^3 + q^4 - q^5 - q^6 - 2q^7 - 2q^8 + q^{10} + 2q^{11} + 2q^{13} + \ldots
\end{aligned}$$
La forme $f_2$ est \emph{vieille} car elle s'écrit $g(2\tau)$ où $g$ est de niveau $\Gamma_0(15)$, ce qui n'est pas le cas de $f_1$ et $f_3$. Ces formes sont de poids 2, donc correspondent à des formes différentielles holomorphes sur la courbe $X_0(30)$. Le terme \emph{parabolique} signifie que ces formes s'annulent sur les pointes de $X_0(30)$, et on peut d'ailleurs lire l'ordre de leur zéro à la pointe $\infty$ sur la première puissance de $q$ apparaissant dans leur développement. On définit alors
$$ x = \frac{f_2}{f_3}, \quad y = \frac{dx/d\tau}{f_3}.$$
Les fonctions modulaires $x$ et $y$ sont de poids zéro, ce sont donc des fonctions méromorphes sur la surface de Riemann $X_0(30)$. On trouve une relation reliant $x$ et $y$ grâce à l'algèbre linéaire:
$$ y^2 =  x^8 + 6x^7 + 9x^6 + 6x^5 - 4x^4 - 6x^3 + 9x^2 - 6x +1. $$
Bien sûr, on ne peut tester cette égalité que sur un nombre fini de coefficients. Pour montrer que cette égalité a bien lieu, on peut utiliser l'argument suivant: la différence
$y^2 - P(x)$
est une fonction méromorphe sur $X_0(30)$ dont les pôles sont concentrés aux pointes, et les ordres de ses pôles sont bornés par une quantité explicite. De plus, le calcul montre que cette fonction admet un zéro à la pointe $\infty$ dont l'ordre est supérieur à, disons, 100. Si cette fonction était non nulle, elle aurait autant de zéros que de pôles, ce qui n'est pas le cas: elle est donc identiquement nulle.

Il reste à voir que $x$ et $y$ engendrent bien le corps des fonctions de $X_0(30)$. La courbe hyperelliptique $\Cl$ donnée par l'équation ci-dessus est de genre 3, et on dispose d'une application rationnelle donnée par les fonctions $x, y$:
$$X_0(30) \vers \Cl.$$
Or, le genre de $X_0(30)$ est également 3 (la dimension de l'espace des formes paraboliques ci-dessus), ce qui force cette application à être de degré 1 par la formule de Hurwitz.

Pourquoi avoir choisi ces fonctions $x$ et $y$ particulières? Pour le comprendre, on peut suivre le chemin inverse. Si l'on souhaite une équation de la forme $y^2 = f(x)$, alors les deux formes différentielles sur cette courbe hyperelliptique
$$\omega_3 =  \frac{dx}{y},\quad \omega_1 = \frac{x\, dx}{y}$$
engendrent l'espace des formes différentielles d'ordre 1, et ont un zéro d'ordre $g$ et 1 respectivement à l'infini, où $g$ désigne le genre de la courbe en question. On a donc choisi $f_3$ et $f_1$ de cette façon. C'est le raisonnement suivi dans la thèse de Galbraith \cite{Galbraith}.

Cette équation de $X_0(30)$ ne nous est utile que si l'on connaît l'expression de la fonction modulaire $j$ en fonction de $x$ et $y$, ce qui permet de retrouver une courbe elliptique à partir d'un point de la courbe modulaire $X_0(30)$. Comme on connaît la $q$-expansion de $j$, on peut faire de l'algèbre linéaire comme précédemment, et on trouve que $j$ peut s'exprimer en fonction de $x$ et $y$ par une fraction rationnelle de degré 36, de numérateur
$$ \begin{aligned}
 &11390626x^{36} + 136688232x^{35} + 615281004x^{34} + 1157841640x^{33} + 11390624x^{32}y + 385027410x^{32} \\
 &+ 102514896x^{31}y + 3899199936x^{31} + 307361808x^{30}y + 68878885856x^{30} + 164026064x^{29}y  \\
 &+ 461330789952x^{29} - 1401072176x^{28}y + 2024096667816x^{28} - 8580123696x^{27}y + 6544388993120x^{27} \\
 &- 54692903792x^{26}y + 16121144156304x^{26} - 290282276144x^{25}y + 30519217044192x^{25} \\
 &- 1110816170032x^{24}y + 44213136417480x^{24} - 3094506362864x^{23}y + 48084954873408x^{23} \\
 &- 6419993916592x^{22}y + 37447747660704x^{22} - 10015247114224x^{21}y + 18293916069056x^{21} \\
 &- 11672985621488x^{20}y + 2315694303708x^{20} - 9885812131312x^{19}y - 4389588785808x^{19} \\
 &- 5658074812208x^{18}y - 4676088269240x^{18} - 1674540316912x^{17}y - 3342328078608x^{17} \\
 &+ 354269688656x^{16}y - 2056348964388x^{16} + 760401538672x^{15}y - 837671190464x^{15} + 580040840752x^{14}y \\
 & + 34891362336x^{14} + 300727188592x^{13}y + 90210852288x^{13} + 129394057072x^{12}y + 159283365960x^{12} \\
  &- 333927056x^{11}y - 3950452128x^{11} - 2917217872x^{10}y + 33962685456x^{10} - 10863538576x^9y \\
  &- 10436233760x^9 + 144372848x^8y + 1058338344x^8 - 342561616x^7y - 907019328x^7 + 79366768x^6y \\
  &+ 209996384x^6 + 4157616x^5y - 19141824x^5 + 4326064x^4y + 4478610x^4 - 286544x^3y - 2160280x^3 \\
  &- 402192x^2y + 883116x^2 + 134064xy - 196248x - 14896y + 16354
\end{aligned}
$$
et de dénominateur
$$ \begin{aligned}
&x^{35} - 13x^{34} + 56x^{33} + 21x^{32} - x^{31}y - 1106x^{31} + 16x^{30}y + 4207x^{30} - 104x^{29}y \\
&- 4268x^{29} + 294x^{28}y - 16637x^{28} + 165x^{27}y + 64174x^{27} - 4184x^{26}y - 62529x^{26} \\
 &+ 14224x^{25}y - 131200x^{25} - 18436x^{24}y + 478757x^{24} - 18529x^{23}y - 645986x^{23} \\
 &+ 102680x^{22}y + 1004763x^{22} - 95688x^{21}y + 5962092x^{21} - 225622x^{20}y - 6407093x^{20} \\
 &+ 572493x^{19}y - 14435032x^{19} + 1166032x^{18}y + 11257145x^{18} - 1185120x^{17}y + 15043608x^{17} \\
 &- 3222648x^{16}y - 10395833x^{16} + 1481173x^{15}y - 6881894x^{15} + 4046208x^{14}y + 5269581x^{14} \\
 &- 1639608x^{13}y + 1177004x^{13} - 2375590x^{12}y - 1236463x^{12} + 1434343x^{11}y - 566142x^{11} \\
 &+ 409992x^{10}y + 341845x^{10} - 686768x^9y + 516016x^9 + 205628x^8y - 349313x^8 + 115085x^7y \\
 &- 61286x^7 - 88744x^6y + 105001x^6 + 9064x^5y - 35308x^5 + 4374x^4y + 6561x^4 - 729x^3y - 729x^3.
\end{aligned}
$$
La situation se complique légèrement, en revanche, si l'on veut connaître une équation de la courbe et les coordonnées du point de 2-torsion garanti par $X_0(30)$, par exemple. Connaître ce point de 2-torsion permet d'éviter une recherche de racine lorsque l'on se demande si la courbe elliptique obtenue admet un modèle de Montgomery. La difficulté est que les coordonnées $a_4$ et $a_6$ de l'équation de Weierstrass \og usuelle\fg, proportionnelles aux séries d'Eisenstein $E_4$ et $E_6$, ne sont pas des fonctions modulaires de poids zéro, mais $\lambda^2 a_4$ et $\lambda^3 a_6$ en sont pour toute fonction modulaire $\lambda$ de poids $-2$. Les coordonnées du point de 2-torsion sont alors un multiple de la série d'Eisenstein $E_2$, précisément $\frac{-\lambda}{6} E_2$. On renvoie à \cite{Elkies} pour les détails.

\subsection{Calcul d'une équation de $X_1(N)$}

Comme dans le cas de $X_0(30)$, on recherche une équation de $X_1(N)$ sous forme de courbe plane valable sur le corps des nombres complexes, et on espère que cette équation est à coefficients entiers et donne bien une équation de $X_1(N)_{\F_p}$ une fois réduite modulo $p$. On recherche donc deux formes modulaires $f$, $g$ de niveau $\Gamma_1(N)$, et un polynôme $\Phi_N^{(f, g)}$ à coefficients entiers tel que
$$\Phi_N^{(f, g)}(f, g) = 0.$$

On suit ici l'article \cite{Baaziz}. On note $\wp(z, \Lambda_\tau)$ la fonction $\wp$ de Weierstrass associé au réseau $\Lambda_\tau = \Z\oplus\Z \tau.$

\begin{thm}
On définit deux fonctions méromorphes de la variable complexe :
$$f(\tau) = \frac{(\wp(1/N, \Lambda_\tau) - \wp(1/N, \Lambda_\tau))^3}{\wp'(1/N, \Lambda_\tau)^2}, \quad
g(\tau) = \frac{\wp'(2/N, \Lambda_\tau)}{\wp'(1/N,\Lambda_\tau)}.$$
Alors $f$ et $g$ sont deux fonctions modulaires pour $\Gamma_1(N)$, et engendrent le corps des fonctions de la courbe $X_1(N)$. Elles vérifient l'équation
$$\psi_N(1+g, f, f) = 0$$
où $\psi_N(a_1, a_2, a_3)$ est le $N$-ième polynôme de division de la courbe d'équation
$$y^2 + a_1xy + a_3y = x^3 + a_2x^2$$
évalué au point $(x, y) = (0, 0)$.
\end{thm}

Remarquons que le calcul de ce polynôme de division évalué en $(0, 0)$ est facile, en utilisant des formules récursives comme précédemment. Rappelons que le polynôme de division s'annule exactement sur les points affines de $N$-torsion de la courbe. Le lemme clé est le suivant:

\begin{lem}
Toute classe d'isomorphisme de courbes elliptiques $(E, P)$ où $P$ est un point d'ordre $N$ sur un corps $K$ algébriquement clos contient un représentant de la forme
$$E\de y^2 + ((1+g) x + f)y = x^3 + fx^2,\ P = (0,0).$$
avec $f\in K^\times,\ g\in K$. Ce représentant est unique à unique isomorphisme près.
De plus, si $K=\C$ et $(E, P) = (\C/\Lambda_\tau, 1/N)$, on a
$$f = \frac{(\wp(1/N, \Lambda_\tau) - \wp(1/N, \Lambda_\tau))^3}{\wp'(1/N, \Lambda_\tau)^2}, \quad  g = \frac{\wp'(2/N, \Lambda_\tau)}{\wp'(1/N,\Lambda_\tau)}.$$
\end{lem}
Ce lemme s'établit à l'aide de calculs sur les modèles de Weierstrass. L'ingrédient clé est que la courbe elliptique $\C/\Lambda_\tau$ est isomorphe à la courbe elliptique $y^2 = x^3 + g_2(\tau) x + g_3(\tau)$, l'isomorphisme étant donné par 
$z\mapsto (\wp(z), \wp'(z))$.

\begin{proof}[Démontration du théorème] On vérifie que les fonctions méromorphes
$$\tau\mapsto \wp(1/N, \Lambda_\tau),\quad \tau\mapsto\wp'(1/N,\Lambda_\tau),\quad \tau\mapsto\wp''(1/N, \Lambda_\tau)$$
sont des formes modulaires pour $\Gamma_1(N)$ de poids respectifs 2, 3 et 4. On montre alors que $f$ et $g$ sont des fonctions modulaires (de poids 0) pour $\Gamma_1(N)$ à l'aide des formules de duplication, donc des fonctions sur $X_1(N)_\C = \overline{\Gamma_1(N)\backslash \H}$. Si $a$ est un point affine de cette courbe modulaire, selon le lemme, $a$ admet un représentant $(E, P)$ qui s'écrit
$$E\de y^2 + ((1 + g(a))x + f(a))y = x^3 + f(a)x^2,\quad P = (0, 0).$$
$P$ étant un point de $N$-torsion, on a bien $\psi_N(1+g(a), f(a), f(a)) = 0$. \'Etant vraie sur un ouvert affine, cette égalité est vraie sur tout $X_1(N)$.

Il reste à montrer que $f$ et $g$ engendrent le corps des fonctions de $X_1(N)_\C$. Pour cela, il suffit de trouver un ouvert non vide $U$ de la surface de Riemann compacte $X_1(N)_\C$ tel que $(f, g)$ sépare les points sur $U$. Or, c'est le cas sur l'ouvert affine $Y_1(N)$, car le lemme donne un représentant d'un tel point en fonction uniquement de $f$ et $g$.
\end{proof}

Le fait que $f$ et $g$ engendrent le corps des fonctions montrent que l'application rationnelle
$$X_1(N)_\C \overset{(f,\ g)}{\vers} \left\{(u, v)\in \P^2(\C) \de \psi_N(1+v, u, u) = 0\right\}$$
est un isomorphisme birationnel sur son image : on a donc obtenu un modèle plan de la courbe modulaire $X_1(N)$.

Vu les résultats précédents sur la gonalité de $X_1(N)$, on doit s'attendre à trouver des équations de degré de l'ordre de $N^2$ en chaque variable. En pratique, on peut s'arranger pour réduire ce degré d'un facteur constant. Une première étape est de remarquer que $\psi_N(1+g, f, f)$ possède un facteur $f^{v_N}.$ On peut déterminer cette valuation à partir des formules de récurrence, et on obtient
$$ v_N =
\begin{cases}
3k^2 & \text{si}\ N = 3k, \\
3k^2 + 2k &\text{si}\ N = 3k+1, \\
3k^2 + 4k + 1 &\text{si}\ N = 3k+2.
\end{cases}$$
En pratique, on calcule directement la quantité $f^{-v_N}\psi_N$, pour laquelle on peut donner une formule récursive semblable. De plus, si $M$ divise $N$, alors $\psi_M(1+g, f, f)$ divise $\psi_N(1+g, f, f)$ (si $MP = 0$, on a aussi $NP = 0$). On définit donc des polynômes $\Phi^{(f, g)}$ vérifiant :
$$\forall N\geq 1,\ f^{-v_N} \psi_N(1+g, f, f) = \prod_{M | N} \Phi_M^{(f, g)}.$$

On a alors $\Phi_N^{(f, g)}(f, g) = 0$. Une fois cette équation calculée, on peut tenter un changement de variables pour la simplifier. On pose ainsi successivement :
$$\begin{matrix} f = st(t-1), & g = s(t-1)\\
 s = q\frac{r+1}{r} - 1, & t = q(r+1) + 1.
 \end{matrix}$$
On peut aisément inverser ces changements de variables pour vérifier que $q$ et $r$ engendrent toujours le corps des fonctions de $X_1(N)$. L'équation obtenue reste à coefficients entiers, et est de plus petit degré que $\Phi^{(f, g)}$ après simplification; expliquer cette constatation expérimentale reste une question ouverte. Une autre approche serait d'effectuer des changements de variables \og aléatoires \fg\ jusqu'à en trouver de plus intéressants : c'est par exemple l'approche de \cite{Sutheq}. Bien sûr, tout ceci est une affaire de facteurs constants, et on ne peut pas échapper à la minoration de la gonalité donnée précédemment.

\v
Avec ces équations explicites de $X_0(30)$ et $X_1(17)$ et une base de données de polynômes modulaires, tous les ingrédients sont réunis pour pouvoir implémenter le cryptosystème de Couveignes--Rostovtsev--Stolbunov et la recherche de courbe initiale décrite plus haut.


\newpage

\section{Implémentation et performances}

\subsection{Systèmes utilisés}

Nous avons principalement utilisé deux systèmes de calcul formel lors de la mesure des performances de ce protocole cryptographique: Sage \cite{Sage} et Nemo \cite{Nemo}.

Sage (ou SageMath) est un logiciel de calcul formel libre et open-source sous licence GPL. Codé principalement dans le langage Python, sa particularité est de faire appel à de nombreuses bibliothèques C spécialisées dans divers domaines, comme GMP \cite{GMP} pour les opérations arithmétiques de base, Singular pour les polynômes à plusieurs variables, ou des systèmes plus vastes comme FLINT \cite{FLINT}, PARI \cite{PARI} ou NTL \cite{NTL}. Sage est extrêment vaste et couvre de nombreux domaines des mathématiques pures et appliquées, et se veut une alternative à des systèmes de calcul formel payants comme Maple, Mathematica, Magma ou Matlab. Du fait de sa richesse, Sage n'est malheureusement pas toujours bien contrôlé: son installation est parfois difficile (sous Windows notamment), et il souffre de nombreuses fuites de mémoire. D'autre part, le langage Python, bien que très facile d'accès, ne présente pas des performances optimales: cela pose problème lorsqu'une partie coûteuse du calcul n'est pas liée directement vers une bibliothèque C spécialisée.

Nemo est un système de calcul formel pour le langage de programmation Julia, libre et open-souce sous licence BSD. Julia est un langage de programmation jeune qui, contrairement à Python, dispose d'une compilation JIT (pour \og just in time\fg): cela implique qu'il est très facile d'écrire du code qui s'approche des performances de C. Dans le même temps, Julia est un langage de très haut niveau qui présente une syntaxe très naturelle similaire à celle de Python. Un dernier avantage est la présence d'un typage paramétrique, ce qui permet d'écrire un code très générique en gardant de bonnes performances. Par exemple, si ce code générique est finalement appelé sur des entiers 32 bits, le compilateur sera capable de le reconnaître et d'optimiser la gestion de la mémoire pour ce type de données précis.

Comme Sage, Nemo fait appel à des bibliothèques C spécialisées, principalement FLINT. Le système Nemo est très jeune, et est beaucoup plus restreint que Sage; certaines opérations \og basiques\fg\ ne sont pas toujours disponibles. C'est parfois un désavantage, mais cette simplicité permet de mieux contrôler le code que l'on écrit et de comprendre facilement ses performances.

\v
Sage contenait déjà des outils pour manipuler les courbes elliptiques, ce qui en faisait le système le plus accessible pour écrire les premiers tests. Il dispose également d'une base de données de polynômes modulaires facile d'accès, et d'une interface avec le logiciel de calcul formel PARI pour les calculs coûteux comme la recherche de racines d'un polynôme à coefficients dans $\F_p$. En revanche, il représente les courbes elliptiques sous forme de Weierstrass uniquement, alors qu'utiliser le modèle de Montgomery améliore significativement les performances de la multiplication scalaire. De plus, les opérations sur une courbe elliptique ne font pas appel à une bibliothèque C spécialisée, ce qui impacte les performances. 

Nous avons donc fait le choix de développer un module au-dessus de Nemo, afin de disposer de ces nouvelles fonctionnalités dans un langage, Julia, qui fournit des performances comparables à un code C natif tout en restant un langage de très haut niveau. Ce module a été utilisé pour la mesure de performances; à terme, il devrait devenir un module open-source contenant les grandes primitives nécessaires à la cryptographie sur courbes elliptiques, et pourra être utilisé dans un but pédagogique.


\subsection{Contenu}

Afin de pouvoir mesurer les performances du cryptosystème de Rostovtsev et Stolbunov, le module de courbes elliptiques pour Nemo dipose des fonctionnalités suivantes:
\begin{itemize}
\item[•] Définir des courbes elliptiques dans divers modèles, sur des corps ou des anneaux
\item[•] Définir des points sur ces courbes, les ajouter, les multiplier par un entier
\item[•] Calculer le $j$-invariant, et des isomorphismes entre différents modèles
\item[•] Définir des isogénies entre courbes et appliquer les formule de Vélu
\item[•] Avoir accès à une base de données de polynômes modulaires (cela fait appel à un autre module)
\item[•] Calculer le noyau d'une isogénie par l'algorithme de Bostan--Morain--Salvy--Schost
\item[•] Calculer la valeur propre du Frobenius sur un sous-groupe
\item[•] Étendre le corps de base d'une courbe elliptique
\item[•] Tirer aléatoirement un point sur une courbe elliptique sur un corps fini.
\end{itemize}

\v
À ce point du développement, on dispose des modèles de Weierstrass et de Montgomery, et le calcul d'isogénies fonctionne lorsque le degré de l'isogénie est impair. Afin que ce module puisse être utilisé par ailleurs, nous avons le projet d'y ajouter les choses suivantes:

\begin{itemize}
\item[•] D'autres modèles de courbes elliptiques, par exemple le modèle d'Edwards
\item[•] Le calcul de couplages entre deux points
\item[•] Le calcul de la cardinalité d'une courbe elliptique, avec un appel au logiciel PARI
\item[•] Un algorithme de calcul d'isogénie en petite caractéristique
\item[•] Et éventuellement d'autres choses en lien avec la cryptographie sur courbes elliptiques.
\end{itemize}

\v
Le code est en libre accès à l'adresse \url{github.com/JKieffer95/EllipticCurves.jl}. Pour l'utiliser, il suffit d'installer Julia 0.6 et Nemo, d'ajouter le dossier EllipticCurves dans le répertoire .../.julia/v0.6 et de demander: 
\begin{verbatim}
julia> Pkg.add("EllipticCurves")

julia> using EllipticCurves
\end{verbatim}
Le module est alors directement accessible.


\subsection{Mesure de performances}

On a vu qu'il est important de mesurer précisément les performances du calcul d'isogénies. Cela permet en effet de déterminer les performances globales du cryptosystème de Couveignes--Rostovtsev--Stolbunov pour une courbe initiale donnée. On peut ainsi d'une part choisir la meilleure courbe initiale lors de la recherche des paramètres, et d'autre part comparer les performances de ce protocoles à d'autres échanges de clés existants. Toutes les mesures sont effectuées sur $\F_p$ où $p$ est le nombre premier déterminé précédemment, d'une taille de 512 bits; la machine utilisée dispose d'un processeur de 3.20GHz Intel Xeon.

Rappelons les deux méthodes de calcul à notre disposition pour calculer une isogénie de degré $\ell$ au départ d'une courbe $E$:
\begin{itemize}
\item[•] Lorsque la courbe elliptique admet un unique sous-groupe d'ordre $\ell$ dont les points sont définis sur une extension de degré $d$, on peut utiliser une multiplication scalaire sur $E$ pour trouver ce sous-groupe, puis utiliser les formules de Vélu;
\item[•] Dans le cas général, on utilise une équation modulaire de degré $\ell + 1$ dont on cherche les racines, on calcule le noyau de l'isogénie correspondante puis on regarde la valeur propre du Frobenius sur ce noyau.
\end{itemize}
Les étapes coûteuses sont alors respectivement
\begin{itemize}
\item[•] Une multiplication par $p^d$ (environ) sur $E(\F_{p^d})$,
\item[•] Une mise à la puissance $p$ modulo un polynôme de degré $\ell + 1$, puis une mise à la puissance $p$ modulo un polynôme de degré $\frac{\ell - 1}{2}$.
\end{itemize}
Ce sont ces deux opérations que l'on choisit de mesurer. Tout d'abord, on présente le coût de la multiplication scalaire dans trois cas: les opérations fournies par Sage (sur les modèles de Weierstrass), ainsi que notre implémentation en Nemo pour les modèles de Weierstrass et de Montgomery. Rappelons que l'on s'attend à trouver un coût quadratique en $d$.

\begin{center}
\begin{tikzpicture}[x = 1.4cm, y=4cm]

\draw[blue] plot file {benchmark_sage_scalarmult.txt};
\draw[red] plot file {benchmark_weierstrass_scalarmult.txt};
\draw[darkgreen] plot file {benchmark_montgomery_scalarmult.txt};
\draw[color=gray!50, very thin] (1,0) grid[xstep = 1, ystep = 0.25] (10, 2);
\foreach \k in {1,2,...,10}
 \draw  (\k, -0.1) node  {\small\k};
 
\foreach \t in {0,0.25,...,2}
 \draw (0.6, \t) node {\small \t};
 
\draw[->] (1,0) -- (1, 2.2);
\draw[->] (1,0) -- (10.5, 0);
\draw (11, 0) node {$d$};
\draw (1, 2.35) node {$t\ (s)$};
\end{tikzpicture}

\small Performance pour le calcul de $[p^d] P$, avec $P \in E(\F_{p^d})$ en utilisant  \textcolor{blue}{Sage},
\textcolor{red}{le modèle de Weierstrass} et
\textcolor{darkgreen}{le modèle de Montgomery}.

\end{center}

Il en ressort qu'utiliser le modèle de Montgomery apporte un gain non négligeable par rapport au modèle classique de Weierstrass. Pour ce dernier, l'écart peut s'expliquer par le fait que le modèle de Weierstrass en Nemo utilise des opérations moins optimisées que celles de Sage. On voit également qu'il est beaucoup plus intéressant de disposer de points de $\ell$-torsion rationnels que sur une extension non triviale, même de degré 3 par exemple.

\v
D'autre part, on présente les performances du calcul de $X^p$ modulo $Q$ lorsque le degré de $Q$ varie, en utilisant la méthode évidente en Sage et en Nemo; pour comparaison, on montre également les performances de PARI pour calculer les racines d'un polynôme du même degré dans $\F_p$, qui réalise essentiellement le même calcul.

\begin{center}
\begin{tikzpicture}[x = 0.025cm, y=1.8cm]
\draw[blue] plot file {benchmark_sage_frob.txt};
\draw[darkgreen] plot file {benchmark_pari_frob.txt};
\draw[red] plot file {benchmark_nemo_frob.txt};
\draw[color=gray!50, very thin] (0,0) grid[xstep = 100, ystep = 0.5] (500, 5);
\foreach \k in {0,100,...,500}
 \draw  (\k, -0.2) node  {\small\k};
 
\foreach \t in {0,0.5,...,5}
 \draw (-20, \t) node {\small \t};
 
\draw[->] (0,0) -- (0, 5.3);
\draw[->] (0,0) -- (520, 0);
\draw (550, 0) node {$\ell$};
\draw (0, 5.5) node {$t\ (s)$};
\end{tikzpicture}

\small{Performance pour le calcul de $X^p \mod{Q(X)}$, où $\deg(Q) = \ell$ dans $\F_p$, avec 
\textcolor{blue}{Sage},
\textcolor{red}{Nemo} et 
\textcolor{darkgreen}{PARI}.}

\end{center}

On voit sur ce graphique qu'il est difficile d'atteindre les performances d'une bibliothèque C spécialisée.

\v
Afin de quantifier les performances du cryptosystème, on choisit la meilleure performance pour chacune de ces deux opérations, c'est à dire le modèle de Montgomery pour la multiplication scalaire et le logiciel PARI pour la recherche de racines. Nous pouvons maintenant donner une mesure de temps précise pour chacune des courbes initiales; c'est cette mesure que l'on utilise pour décider de la meilleure courbe initiale.

\subsection{Catalogue de courbes}

On peut maintenant donner les meilleures courbes obtenues par la méthode de recherche décrite précédemment. Après un premier tri, les meilleures candidates sont comparées grâce aux temps donnés ci-dessus.

Le calcul a été effectué par Sage sur les machines de l'équipe Grace d'Inria, à Saclay. Le temps de calcul total a été d'environ 17000 heures CPU (environ 2 ans, répartis sur différents c\oe urs de différents machines de l'équipe). Rappelons que
$$ \begin{aligned}
p = \ &120373407382088450343833839782228011370920294512701979230713977354082515866699 \\
&38291587857560356890516069961904754171956588530344066457839297755929645858769.
\end{aligned}$$

Voici donc les cinq meilleurs résultats, où l'on donne les paramètres $a, b$ de $\F_p$ définissant la courbe initiale 
$E\de y^2 = x^3 + ax + b$.

\begin{enumerate}

\item La courbe lauréate, donnée par les paramètres
$$\begin{aligned}
a =\ & 111164648488639530154045798728826155824357214700727018080372266722366593581458 \\
&61235530764279830938006566253824939581765595814774014481950202045501019370057,\\
b =\ & 79299948324245435156826219405909019495910509940412398234989651750360146055419 \\
&69152503725727477980546149790133670177545414851226849642510122057033563808087
\end{aligned}$$
donne un temps de parcours total de 241 secondes. Elle possède un point de $\ell$-torsion rationnel pour tout $\ell\in\{3, 5, 7, 11, 13, 17, 103, 523, 821, 947, 1723\}$. Son nombre de points sur $\F_p$ est
$$\begin{aligned}
&120373407382088450343833839782228011370920294512701979230713977354082515866700 \\
&85481138030088461790938201874171652771344144043268298219947026188471598838060.
\end{aligned}$$

\item La courbe donnée par les paramètres
$$\begin{aligned}
a =\ & 43840601250479313001373029130703162545657631628386541829543985783667722196819 \\
& 83224703081195403305048751140227654592240178420097979549618534551394652828125,\\
b =\ & 15496029659544806200673781148304612576874915050506487020954702483761967446835\\
& 37981569065162806573553904845624037365860256326225753166777384039680905324745
\end{aligned}$$
donne un temps de parcours total de 255 secondes. Elle possède un point de $\ell$-torsion rationnel pour tout $\ell\in\{3, 5, 7, 11, 13, 617, 823, 911, 2741\}$. Son nombre de points sur $\F_p$ est
$$\begin{aligned}
&120373407382088450343833839782228011370920294512701979230713977354082515866701 \\
&38195889195780276859464571113614769417611266612000180056269739527141220119860.
\end{aligned}$$

\item La courbe donnée par les paramètres
$$\begin{aligned}
a =\ & 29517794513388167079924232272984809284734463332655422100821259368004502646122 \\
&952457634401960 11377947505687906946702377331005567046387329096732076013487786,\\
b =\ & 20053142766210512206895831956067687278077475355845207144609905468409809112720 \\ &96530930100587341304378274081554750091335545148023305498915445491413837103327
\end{aligned}$$
donne un temps de parcours total de 268 secondes. Elle possède un point de $\ell$-torsion rationnel pour tout $\ell\in\{3, 5, 7, 11, 13, 19, 41, 53, 137, 463, 479, 1487\}$. Son nombre de points sur $\F_p$ est
$$\begin{aligned}
&120373407382088450343833839782228011370920294512701979230713977354082515866700 \\ 
&73203413637440471898330295081671080323874929304183417201801524565834682159200.
\end{aligned}$$

\item La courbe donnée par les paramètres
$$\begin{aligned}
a =\ & 90951896883679247161305392056057501142984677621212827642316340355546948094470 \\
&24864291306024183700000129771938088507462877895751345076127980672694337232911,\\
b =\ & 53702533899051600924758507744833413409468531550522028426034775831774723715610 \\
&33304215623428708757733212555261802174294804948862439221808386692046723052122
\end{aligned}$$
donne un temps de parcours total de 273 secondes. Elle possède un point de $\ell$-torsion rationnel pour tout $\ell\in\{3, 5, 7, 11, 13, 37, 43, 127, 347, 457, 1289, 1621\}$. Son nombre de points sur $\F_p$ est
$$\begin{aligned} 
&120373407382088450343833839782228011370920294512701979230713977354082515866699 \\
&55535934625878401362947903345661723969818065903127482030480913154573514334480.
\end{aligned}$$

\item Enfin, la courbe donnée par les paramètres
$$\begin{aligned}
a =\ & 87315792648930225008943587622481440451025081358306503063587707716608806180531 \\
& 79022484817139989036152987296557434567077843065574780789697405848884702097913,\\
b =\ & 63166216325315547106996156590059751380252981936070172805289816785905599938288 \\
& 29632834951338126458626870397768930797859281240868070581323508301210433040195\\
\end{aligned}$$
donne un temps de parcours total de 284 secondes. Elle possède un point de $\ell$-torsion rationnel pour tout $\ell\in\{3, 5, 7, 11, 13, 19, 23, 233, 359, 491, 631, 1481, 2579\}$. Son nombre de points sur $\F_p$ est
$$\begin{aligned}
&120373407382088450343833839782228011370920294512701979230713977354082515866700 \\
&82724204500287887625876543581163624210402478479308430356550349544933039974580.
\end{aligned}$$


\end{enumerate}


\v
Les temps annoncés sont des projections établies à partir des valeurs représentées sur les graphiques précédents; le cryptosystème \og entier\fg\ n'a pas été testé pour ces courbes initiales. Ce temps est celui d'un parcours dans le graphe d'isogénies, c'est à dire un calcul d'action pour un unique élément du groupe de classes sur une unique courbe. Ce calcul doit donc être réalisé deux fois si l'on souhaite effectuer un échange de clé. La meilleure courbe a été obtenue à partir d'un point de $X_1(17)$, les autres à partir d'un point de $X_0(30)$.

\v
Même avec une courbe initiale ayant de bonnes propriétés, effectuer un échange de clé en utilisant ce protocole nécessite donc une dizaine de minutes environ, ce qui est bien entendu inacceptable: le cryptosystème de Couveignes--Rostovtsev--Stolbunov a donc peu d'intérêt en pratique, comme on pouvait le prévoir au vu de la quantité de calculs nécessaires! Quoi qu'il en soit, on est maintenant capable de dire \emph{à quel point} ce système est inefficace.

\newpage

\bibliographystyle{plain}

\bibliography{biblio}


\end{document}
