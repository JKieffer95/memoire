\documentclass[11pt,a4paper]{article}

%Encoding and language
\usepackage[T1]{fontenc}
\usepackage[utf8]{inputenc}
\usepackage[french]{babel}

%AMS packages
\usepackage{amsmath, amsthm, amssymb}

%Fonts and graphics
\usepackage{graphicx, textcomp, lmodern, fullpage}

%Diagrams
\usepackage[all]{xy}

%Algorithms
\usepackage[ruled, vlined, linesnumbered, french]{algorithm2e}

%Additional commands
\newcommand{\Z}{\mathbb{Z}}
\newcommand{\C}{\mathbb{C}}
\newcommand{\A}{\mathbb{A}}
\newcommand{\F}{\mathbb{F}}
\newcommand{\Q}{\mathbb{Q}}
\newcommand{\E}{\mathcal{E}}
\renewcommand{\Pr}{\mathcal{P}}
\newcommand{\Qr}{\mathcal{Q}}
\renewcommand{\H}{\mathbb{H}}
\renewcommand{\P}{\mathbb{P}}
\newcommand{\M}{\mathcal{M}}
\renewcommand{\O}{\mathcal{O}}
\newcommand{\Cl}{\mathcal{C}}
\renewcommand{\b}{\backslash}
\newcommand{\vers}{\longrightarrow}
\newcommand{\End}{\mathrm{End}}
\newcommand{\Hom}{\mathrm{Hom}}
\newcommand{\Ell}{\mathrm{Ell}}
\newcommand{\Spec}{\mathrm{Spec}\,}
\renewcommand{\frak}{\mathfrak}
\newcommand{\de}{\,:\,}
\newcommand{\id}{\mathrm{id}}
\newcommand{\pr}{\mathrm{pr}}
\renewcommand{\mod}{\ \mathrm{mod}\ }

%Theorem environments
\newtheorem*{thm}{Théorème}
\newtheorem*{lem}{Lemme}
\newtheorem*{prop}{Proposition}
\newtheorem*{cor}{Corollaire}
\theoremstyle{definition}
\newtheorem*{rem}{Remarque}
\newtheorem*{defi}{Définition}
\newtheorem*{ex}{Exemples}

\title{Un cryptosystème à base d'isogénies}
\author{Jean Kieffer}
\date{\today}

\begin{document}

\maketitle
\tableofcontents
\newpage

\section{Classes d'isogénies de courbes elliptiques CM}

\subsection{Courbes elliptiques}

Rappels de définitions et des notations du cours ?

\begin{prop}[Résultats du cours de courbes elliptiques]
Soit $k$ un corps et $E,\ E_1,\ E_2/k$ des courbes elliptiques. 
\begin{enumerate}

\item L'application
$$\begin{aligned}
&\Z &\longrightarrow&\ &\End(E) &\\
&n &\longmapsto& &[n]_E \ \ &
\end{aligned}$$
est injective.

\item Pour toute isogénie $\phi\de E_1\vers E_2$ de degré $m$, il existe une unique isogénie notée $\widehat{\phi}\de E_2\vers E_1$ de degré $m$ telle que $\phi\widehat{\phi}=[m]_{E_2}$ et $\widehat{\phi}\phi=[m]_{E_1}$.
De plus, on a les relations suivantes lorsqu'elles ont un sens :
$$\widehat{\phi+\psi}=\widehat{\phi}+\widehat{\psi},\quad \widehat{\phi\psi}=\widehat{\psi}\widehat{\phi}.$$
En particulier, deg : $\End(E)\vers \End(E)$ est une forme quadratique et $[m]_E$ est de degré $m^2$.

\item On a l'alternative suivante : $\End(E)$ est soit $\Z$, soit un ordre dans un corps quadratique imaginaire, soit un ordre dans une algèbre de quaternions sur $\Q$. De plus, le dernier cas ne survient qu'en caractéristique positive. Si $k$ est un corps fini, alors on est dans le second cas lorsque $E$ est ordinaire, et dans le troisième cas si $E$ est supersingulière.

\end{enumerate}
\end{prop}


Dans la suite de ce document, on s'intéresse à des courbes elliptiques ordinaires sur un corps fini : on étudie donc plus en détail le cas de la multiplication complexe par un ordre dans un corps quadratique imaginaire. Pour cela, quelques résultats liminaires sur les ordres quadratiques.


\begin{defi}
Soit $K$ un corps de nombres. Un \emph{ordre} de $K$ est un sous-anneau $\O$ de $K$ qui est un $\Z$-module de type fini et engendre $K$ comme $\Q$-espace vectoriel. En particulier, $\O\subset \O_K$, $\O$ est un $\Z$-module libre de rang $[K:\Q]$, et $\O$ est d'indice fini dans $\O_K$.

On dit que $\O_K$ est \emph{l'ordre maximal} de $K$.
\end{defi}

\begin{prop}
Avec ces notations, $\O$ est un anneau noethérien de dimension 1. En revanche, si $\O\neq \O_K$ alors $\O$ n'est pas intégralement clos.
\end{prop}

Dans le cas des corps quadratiques, les ordres ont une description particulièrement simple.

\begin{defi}
Soit $\O$ un ordre d'un corps quadratique $K$ On appelle $f=[\O_K:\O]$ le \emph{conducteur} de $\O$, et on a alors $\O=\Z+f \O_K$. On définit de plus le \emph{discriminant} $D$ de $\O$ de la même façon que celui de $\O_K$ : on a donc $D=f^2 D_K$.
\end{defi}

\begin{defi}
Idéaux d'un ordre quadratique. Norme. Idéaux propres. Idéaux fractionnaires. Dans le cas d'un corps quadratique, les idéaux fractionnaires propres sont les idéaux fractionnaires inversibles. Les idéaux fractionnaires principaux sont inversibles. Groupe de classe de $\O$, noté $\Cl(\O)$.
\end{defi}

\begin{defi}
Idéaux premiers au conducteur. Ce sont des idéaux propres. Le groupe de classe reste le même si l'on se restreint aux idéaux premiers au conducteur.
\end{defi}



\subsection{Multiplication complexe sur $\C$}


Ce thème étant déjà traité en détail dans Silverman, on se contente ici de donner les résultats importants. Dans cette section, on fixe un ordre $\O$ dans un corps quadratique $K\subset \C$, et l'on note $\Ell_\C(\O)$ l'ensemble des courbes elliptiques sur $\C$ (à isomorphisme près) ayant multiplication complexe par $\O$. Comme d'habitude, on identifiera un élément de cet ensemble avec ses représentants.

Dans Silverman, on suppose que $\O$ est l'ordre maximal, mais cette supposition n'est pas essentielle.

\begin{prop}

Soit $E\in \Ell_\C(\O)$. Alors il existe un isomorphisme canonique

$$[\,\cdot\,]\de \O\vers \End(E)$$

tel que l'on ait pour tout $\omega\in H^0(E,\Omega^1)$ :

$$\forall\,\alpha\in\O,\ [\alpha]^*\omega = \alpha \omega.$$

\end{prop}

On dit qu'un tel isomorphisme est \emph{normalisé}, et on l'utilisera silencieusement dans la suite pour identifier $\O$ et $\End(E)$. On rappelle au passage que $H^0(E,\Omega^1)$ est un $\C$-espace vectoriel de dimension 1.

\begin{prop}
Pour tout sous-groupe fini $S$ de $E$, il existe une isogénie $E\vers E'$ de noyau $S$, et celle-ci est unique à isomorphisme près.
\end{prop}

\begin{defi}
Soit $\frak a$ un idéal inversible de $\O$. On définit

$$E[\frak a]=\bigcap_{\phi\in \frak a} \mathrm{Ker}\,\phi.$$

On note $\phi_{\frak a}$ l'isogénie de noyau $E[\frak a]$, et $\frak a\cdot E$ sa courbe image (qui est bien définie comme élément de $\Ell_\C(\O)$).
\end{defi}

\begin{prop}

Pour tout idéal inversible $\frak a$ de $\O$ et toute courbe $E=\C/\Lambda\in \Ell_\C(\O)$, on a :

$$\frak a\cdot E = \C/\frak a^{-1} \Lambda.$$

Le degré de $\phi_{\frak a}$ est la norme de l'idéal $\frak a$, et la courbe $\frak a\cdot E$ a également multiplication complexe par $\O$. Cette opération définit une action du groupe des idéaux fractionnaires inversibles de $\O$ sur $\Ell_\C(\O)$, qui est triviale pour les idéaux principaux. Elle se factorise donc en une action

$$\Cl(\O) \circlearrowright \Ell_\C(\O).$$

Cette dernière action est simplement transitive.

\end{prop}

La preuve de ce résultat utilise de manière essentielle le fait qu'une courbe elliptique complexe est un tore $\C/\Lambda$ ; elle n'est donc pas transposable telle quelle en caractéristique positive notamment. Le but est maintenant d'adapter ce résultat à la caractéristique positive, et notamment les corps finis.




\subsection{Théorèmes de relèvement}


Une première direction pour adapter ces résultats aux corps finis est d'utiliser les théorèmes suivants, dus à Deuring. On se donne ici $k$ un corps fini de caractéristique $p$.


\begin{thm}[Relèvement en caractéristique zéro]

Soit $E/k$ une courbe elliptique et $\alpha$ un endomorphisme de $E$. Alors il existe un corps de nombres $L$, une courbe elliptique $E^0/L$, un endomorphisme $\alpha^0$ de $E^0$ et un premier $\frak P$ de $L$ de corps résiduel $k$, tels que $E^0$ a bonne réduction à $\frak P$, $E$ est isomorphe à $\bar{E^0}$ et $\alpha$ correspond à $\bar{\alpha^0}$ sous cet isomorphisme.

\end{thm}

En particulier, si $\End(E)$ est isomorphe à un ordre $\O$, on peut choisir pour $\alpha$ un générateur de $\End(E)$ et on obtient une surjection $\End(E^0)\vers \End(E).$ Le résultat suivant montre en fait que dans ce cas, on obtient une bijection entre les anneaux d'endomorphismes.

\begin{thm}[Bonne réduction]

Soit $L$ un corps de nombres, $E/L$ une courbe elliptique telle que $\End(E)\simeq \O$ est un ordre dans un corps imaginaire quadratique $K$. Soit $p$ un nombre premier et $\frak P$ un premier de $L$ au-dessus de $p$, où $E$ a bonne réduction. Alors $\bar{E}$ est ordinaire si et seulement si $p$ est totalement scindé dans $K$. Dans ce cas, si $c=p^r c_0$ est le conducteur de $\O$, on a :

\begin{itemize}
\item[(i)] $\End(\bar{E})=\Z+c_0 \O_K$ est l'ordre de $K$ de conducteur $c_0$.
\item[(ii)] Si $c=c_0$, alors la réduction donne un isomorphisme $\End(E)\vers\End(\bar{E})$.
\end{itemize}

De plus, $\End(E)\vers\End(\bar{E})$ préserve le degré.

\end{thm}

Utilisons maintenant ce résultat pour adapter la section précédente au cas des courbes elliptiques sur un corps fini. On fixe à partir de maintenant un ordre quadratique $\O$ (que l'on voit dans $\C$) et un premier $\frak p$ de $\O$ au-dessus de $p$ ; il y a deux possibilités selon le second théorème. On notera par $\alpha\vers\bar{\alpha}$ la réduction mod $\frak p$.


\begin{prop}

Soit $E\in \Ell_k(\O)$. Alors il existe un unique isomorphisme

$$[\,\cdot\,] \de \O \vers \End(E)$$

tel que pour tout $\omega \in H^0(E,\Omega^1)$ on ait :

$$\forall \, \alpha \in \O,\ [\alpha]^* \omega = \bar{\alpha}\omega.$$

De plus, pour tout corps de nombre $L$ contenant $\O$, toute place $\frak P$ de $L$ au-dessus de $\frak p$, pour toute courbe elliptique $E^0/L$ se réduisant en $E$ avec un isomorphisme d'anneaux d'endomorphismes, on a un diagramme commutatif :

$$
\shorthandoff{;:!?}
\xymatrix @!=8mm {
\O \ar[rr]^\sim_{[\,\cdot\,]} \ar[rd]^\sim_{[\,\cdot\,]} & & \End(E^0) \ar[ld]^{\mathrm{mod}\,\frak P} \\
 & \End(E) & 
}
$$

où la première ligne est normalisée comme dans la section précédente.

\end{prop}

\begin{proof}

On construit cet isomorphisme à l'aide du théorème de relèvement (si l'on n'avait pas la bonne place au-dessus de $\frak p$, on peut faire agir Galois). L'unicité provient du fait qu'il n'existe que deux isomorphismes de $\O$ vers $\End(E)$ ; l'un envoie un certain élément $t\in \O$ sur le Frobenius $\pi = F_{E/k}$, et l'autre sur $\widehat{\pi}.$ On peut les distinguer avec la condition sur les pullbacks, puisque dans le cas ordinaire, $\pi$ n'est pas séparable alors que $\widehat{\pi}$ l'est.

Le diagramme commutatif découle alors immédiatement de l'unicité.
\end{proof}

A l'aide de cette identification, on peut maintenant définir comme précédemment $\frak a\cdot E$ pour $\frak a$ un idéal inversible de $\O$ et $E\in \Ell_k(\O)$. La courbe $\frak a\cdot E$ reste définie sur $k$, puisque $E[\frak a]$ l'est. On \emph{admet} alors le résultat suivant :

\begin{prop}

Pour tout idéal $\frak a$ inversible de $\O$ de norme première à $p$, le sous-groupe $E[\frak a](\bar{k})$ de $ E(\bar{k})$ est de cardinal $N(\frak a)$.

\end{prop}

On remarque que c'est vrai lorsque $\frak a$ est l'idéal engendré par un entier $n$ premier à $p$ ; pour adapter cette proposition au cas où $p$ divise la norme, il faut parler du rang du schéma en groupes fini plat $E[\frak a]$.

En particulier cela montre que $\phi_{\frak a}^0$, définie en partant d'un relèvement en caractéristique zéro, se réduit en $\phi_{\frak a}$. On peut alors retrouver l'analogue du résultat sur $\C$ :

\begin{prop}

Cette opération définit une action de $\Cl(\O)$ sur $\Ell_k(\O)$ qui est simplement transitive.

\end{prop}

\begin{proof}

On a vu que si $E^0$ est un bon relèvement de $E$ en caractéristique zéro, $\phi_{\frak a}^0$ se réduit en $\phi_{\frak a}$ et donc $\frak a\cdot E^0$ se réduit en $\frak a\cdot E$. On en déduit immédiatement que c'est une action. De plus on peut utiliser le théorème de bonne réduction pour voir que $\frak a\cdot E$ a également multiplication complexe par $\O$.

Montrons qu'elle est transitive : si $E$ et $E'$ sont données en caractéristique $p$, choisissons deux bons relèvements $E^0, E'^0$ en caractéristique 0. Il existe alors un idéal $\frak a$ envoyant $E^0$ sur $E'^0$, et en réduisant on voit que $\phi_{\frak a}$ a bien pour image $E'$.

Enfin étant donné $E$, montrons que son stabilisateur est l'ensemble des idéaux principaux : si $\frak a$ est principal, il laisse un bon relèvement $E^0$ invariant, et donc $E$ également. Réciproquement, si $\frak a$ laisse $E$ invariante, on a un diagramme commutatif

$$
\shorthandoff{;:!?}
\xymatrix @!=8mm {
E^0 \ar[rr]^{\phi_{\frak a}^0} \ar[rd]_{\mathrm{mod}\,\frak P} & & E'^0 \ar[ld]^{\mathrm{mod}\,\frak P} \\
 & E \ar@(dr,dl)^{\phi_{\frak a}} & 
}
$$

Mais $\phi_{\frak a}$ admet également un relèvement $\psi\in \End(E^0)$ ; tous ces endomorphismes sont de degré $N(\frak a)$ selon le fait admis. On voit alors que $\psi$ et $\phi_{\frak a}^0$ ont même noyau, ce qui montre $\psi=\phi_{\frak a}^0$ et $\frak a\cdot E^0 = E^0$. Ainsi $\frak a$ est principal selon la section précédente.
\end{proof}


Outre le fait d'admettre un résultat qui n'est pas élémentaire, cette méthode a le désavantage de très mal se généraliser aux corps de caractéristique positive qui ne sont pas des corps finis.

Une façon peut-être plus naturelle d'adapter aux corps finis les résultats obtenus sur $\C$ est d'utiliser les modules de Tate, comme des espaces naturels attachés aux courbes elliptiques dans lesquels $\End(E)$ se comporte effectivement comme un réseau.




\newpage

\subsection{Modules de Tate}


Pour l'instant, on se donne un corps quelconque $k$.

\begin{defi}

On rappelle la définition bien connue des modules de Tate pour une courbe elliptique $E/k$ (cette définition a un sens dans un cadre plus général, celui des variétés abéliennes). Si $\ell$ est un nombre premier, on a des applications
$$ [\ell] \de E[\ell^{m+1}](\bar{k})\vers E[\ell^{m}](\bar{k})$$

et on définit le \emph{module de Tate}
$$T_\ell(E) = \lim_{\leftarrow} E[\ell^m](\bar{k}).$$

On a donc une identification canonique $E[\ell^m]\simeq T_\ell(E)/ \ell^{m} T_\ell(E)$.

Si $\ell$ est différent de la caractéristique de $k$, on sait que $E[\ell]$ est étale de rang $\ell^2$, donc $T_\ell(E)$ est un $\Z_\ell$-module libre de rang 2. On peut donc le voir comme un réseau dans le $\Q_\ell$-espace vectoriel de dimension 2
$$V_\ell(E) = T_\ell(E) \otimes_{\Z_\ell} \Q_\ell.$$

Le diagramme suivant est alors commutatif :
$$
\shorthandoff{;:!?}
\xymatrix {
\ell^{-m} T_\ell(E)/ T_\ell(E) \ar[r]_{\ell^m}^{\sim} \ar@{^{(}->}[dd]&
 T_\ell(E)/ l^m T_\ell(E) \ar[r]^{\sim}  & 
 E[\ell^m](\bar{k}) \ar@{^{(}->}[dd] \\ 
 \\
 \ell^{-m-1} T_\ell(E)/ T_\ell(E) \ar[r]_{\ell^{m+1}}^{\sim} &
 T_\ell(E)/ \ell^{m+1} T_\ell(E) \ar[r]^{\sim}  & 
 E[\ell^{m+1}](\bar{k})
}
$$

ce qui donne une identification canonique avec les points $\ell$-primaires de la courbe,
$$ V_\ell(E)/ T_\ell(E) \simeq E(\ell)(\bar{k}).$$

Avec cette identification, les sous-groupes $\ell$-divisibles de $E(\bar{k})$ correspondent aux réseaux de $V_\ell(E)$ contenant $T_\ell(E)$. Tous ces objets sont munis de plus d'une action naturelle du groupe de Galois $G= \mathrm{Gal}(\bar{k}/k)$.

\end{defi}

Si $\phi\de E\vers E'$ est une isogénie définie sur $k$, ses restrictions $E[\ell^m](\bar{k})\vers E'[\ell^m](\bar{k})$ sont compatibles, et $\phi$ induit donc naturellement une application
$$ \phi_\ell\de T_\ell(E)\vers T_\ell(E').$$

Cette application est un morphisme de $\Z_\ell[G]$-modules, que l'on peut également considérer de $V_\ell(E)$ dans $V_\ell(E')$. Avec ces notations, on a le théorème suivant, dû à Tate :

\begin{thm} Si $k$ est un corps fini, cette application
$$\Hom_k(E, E') \vers \Hom_{\Z_\ell[G]} (T_\ell(E), T_\ell(E'))$$

est un isomorphisme.
\end{thm}

On suppose maintenant que \emph{$k$ est un corps fini}. Connaissant ce résultat, on peut faire le lien entre les isogénies et les sous-groupes de $E(\bar{k})$, les isogénies, les réseaux dans les modules de Tate et les idéaux de l'anneau d'endomorphisme, et démontrer la simple transitivité de l'action de $\Cl(\O)$ sur $\Ell_k(\O)$.

\begin{rem}

Cet isomorphisme est valable uniquement dans le cas $\ell\neq p$, si $k$ est de caractéristique $p$. Dans le cas $\ell=p$, on peut construire un autre $\Z_p[G]$-module qui vérifie le même genre de propriétés, mais sa construction est plus complexe : cf. Waterhouse. Ici on se concentre sur le cas $\ell\neq p$.

\end{rem}

Dans ce cadre, le lemme fondamental reliant l'anneau d'endomorphismes aux modules de Tate est le suivant. Ici on fixe un isomorphisme $\O\vers \End(E)$ que l'on rend cohérent dans une classe d'isogénie, par exemple celui que l'on a défini à la section précédente.

\begin{lem}

Soit $\ell\neq p$ un nombre premier, $\frak a$ un idéal de $\O$, et $\phi\de E\vers E' = E/E[\frak a]$ l'isogénie quotient. Alors on a l'égalité

$$\phi_\ell^{-1} T_\ell(E') = \bigcap_{\rho\in\frak a}\ \rho_\ell^{-1} T_\ell(E) .$$
\end{lem}

Cela permet de décrire la localisation de l'idéal $\frak a$ en $\ell$ en termes de modules de Tate. On utilise de plus le résultat bien connu qu'un réseau est déterminé par ses localisations :

\begin{lem}[Weil-Eischle] L'application

$$\begin{aligned}
\mathcal{R}(\Q^n) &\longrightarrow \prod_{\ell\in \mathcal{P}} \mathcal{R}(\Q_\ell^n) \\
\Lambda\ \ &\longmapsto (\Lambda_\ell = \Lambda\otimes_\Z \Z_\ell)_{\ell\in\mathcal{P}}
\end{aligned}$$

est une injection d'image l'ensemble des collections $(\Lambda_\ell)$ telles que $\Lambda_\ell = \Z_\ell$ pour presque tout $\ell$.
\end{lem}

On peut ensuite démontrer sans trop de difficulté le théorème suivant.

\begin{thm}

Soit $k$ un corps fini, $E_0/k$ une courbe elliptique ordinaire ayant multiplication complexe par un ordre $\O$ d'un corps quadratique $K$. Alors tout idéal de $\O$ est un idéal de noyau pour toute courbe elliptique telle que $\End(E)=\O$. De plus, $\Ell_k(\O)$ est un espace principal homogène pour $\Cl(\O)$.

\end{thm}


\begin{proof}

Comme on l'a dit, on va regarder uniquement les premiers $\ell\neq p$ ; les résultats sont similaires dans le cas $\ell = p$, et on les admet ici.

On démontre tout d'abord la transitivité. Soient $E,\ E'$ deux courbes d'ordre $\O$. En particulier $E$ et $E'$ ont même nombre de points, donc il existe une isogénie $E \vers E'$ que l'on peut supposer séparable : ainsi il existe un sous-groupe fini $G\subset E(\bar{k})$ tel que $E' = E/G$.

Cherchons un idéal $I$ tel que $G = E[I]$. Pour $\ell\neq p$, $G$ correspond à un réseau contenant $T_\ell E$, donc un sous-réseau de $S_\ell E$ : comme $S_\ell E$ est libre de rang 1, ce réseau s'écrit sous la forme $I_\ell \cdot S_\ell E$ pour un certain idéal $I_\ell$.

Comme $G$ est fini, $I_\ell$ est trivial pour presque tout $\ell$. Il existe donc un réseau $I$ de $K$ dont les localisations sont les $I_\ell$. $I$ est alors un idéal puisque c'en est un localement, et on vérifie ensuite $G = E[I]$, ce qui montre la transitivité.

\end{proof}





\newpage

\section{Courbes modulaires}


Le but de cette section est de définir les courbes modulaires, d'en expliciter les propriétés et des modèles naturels. L'intérêt de ces courbes est qu'elles paramétrisent naturellement la donnée d'une courbe elliptique munie d'une certaine structure. Nous suivons ici \cite{KaMa}, en essayant de rester aussi concret que possible. L'étude du cas complexe permet de motiver et d'introduire l'étude générale, qui impliquera les résultats dont nous avons besoin sur les corps finis.


\subsection{Courbes modulaires sur $\C$}

\begin{prop}

Notons $\H$ le demi-plan de Poincaré et pour $n\geq 1$, $\Gamma(n)$, $\Gamma_0(n)$, $\Gamma_1(n)$ les sous-groupes usuels de $\Gamma(1).$ Alors :

\begin{itemize}
\item[•] Le quotient $\Gamma(n)\b \H$ peut être identifié à l'ensemble des classes d'isomorphisme de paires $(E,\phi)$, où $E$ est une courbe elliptique complexe et $\phi\de (\Z/n\Z)^2\vers E[n]$ est un isomorphisme de groupes compatible à l'accouplement de Weil.
\item[•] Le quotient $\Gamma_0(n)\b \H$ peut être identifié à l'ensemble des classes d'isomorphisme des paires $(E,G)$, où $E$ est une courbe elliptique sur $\C$ et $G$ est un sous-groupe de $E[n]$ isomorphe à $\Z/n\Z$ (ou, de manière équivalente, à l'ensemble des classes d'isomorphisme des triplets $(E_1, E_2, \phi)$ où $\phi\de E_1\vers E_2$ est une isogénie cyclique de degré $n$).
\item[•] Le quotient $\Gamma_1(n)\b \H$ peut être identifié à l'ensemble des classes d'isomorphisme des paires $(E,P)$, où $E$ est une courbe elliptique sur $\C$ et $P$ est un point de $E$ d'ordre $n$.
\end{itemize}

\end{prop}

En effet, on sait que l'application
$$\tau \longmapsto \C/\Lambda_\tau\quad \text{où}\ \Lambda_\tau = \Z \oplus \Z\tau$$
induit une bijection entre le quotient $\Gamma(1)\b \H$ et l'ensemble des courbes elliptiques sur $\C$ à isomorphisme près. On peut ensuite expliciter chacune des autres bijections :

$$\begin{aligned}
\tau\in\ &\Gamma(n)\b \H &\longmapsto\ &\left(\C/\Lambda_\tau,\ \left(\frac{1}{n},\frac{\tau}{n}\right)\right) \\
\tau\in\ &\Gamma_0(n)\b \H &\longmapsto\ &\left(\C/\Lambda_\tau, \left\{\frac{a}{n},\ 0\leq a<n\right\} \right) \\
\tau\in\ &\Gamma_1(n)\b \H &\longmapsto\ &\left(\C/\Lambda_\tau,\ \frac{1}{n}\right).
\end{aligned}$$

On dit que \emph{les surfaces $\Gamma(n)\b \H$, $\Gamma_0(n)\b \H$ et $\Gamma_1(n)\b \H$ représentent les problèmes modulaires} de classification de ces structures complexes à isomorphisme près. On note ces surfaces respectivement $Y(n)_\C$, $Y_0(n)_\C$ et $Y_1(n)_\C$. On se restreindra à ces problèmes modulaires particuliers, mais on pourrait en définir pour d'autres sous-groupes d'indice fini de $\Gamma(1)$.

Si $Y$ est l'un des trois quotients précédents, on peut montrer qu'il existe un nombre fini $P$ de points tel que $Y\cup P$ est muni d'une structure de surface de Riemann compacte $X$ : on appelle ces points les \emph{pointes} de $Y$. Par conséquent, ces trois quotients sont isomorphes aux surfaces de Riemann associées à une courbe algébrique projective lisse sur $\C$ (c'est une conséquence du théorème de Riemann-Roch). $Y$ correspond alors à une partie affine de $X$, et on peut se demander à quoi ressemblent des équations définissant $Y$.


\subsection{Une équation explicite pour $Y_0(n)_\C$}


Dans la suite de ce document, on utilise une courbe modulaire donnée par une équation explicite pour calculer des isogénies entre courbes elliptiques sur un corps fini. Il est donc important de disposer de telles équations en pratique.

Cependant, même dans le cas complexe, ces courbes n'admettent en général pas d'équation agréable. Deux cas font exception : les courbes $Y(1)$ et d'une certaine façon $Y_0(n)$.

\textbf{Le cas $Y(1)$.} On sait que la fonction modulaire $j$ induit un isomorphisme de surfaces de Riemann :
$$Y(1)_\C = \Gamma(1)\b \H \vers \C.$$
La courbe $Y(1)_\C$ est donc obtenue \og sans équation \fg.

\textbf{Le cas $Y_0(n)$.} On a vu que l'on peut reformuler le problème modulaire associé à $Y_0$ sous la forme du problème de $n$-isogénie cyclique $E\vers E'$. On peut alors définir une application, qui représente informellement
$$(E\vers E') \longmapsto (E, E')$$
et qui s'écrit dans le cas complexe

$$\begin{aligned}
Y_0(n)_\C &\vers \C^2 \\
 \tau &\longmapsto (j(\tau), j(n\tau)).
\end{aligned}$$

On peut montrer que l'image de cette application est la courbe définie par un polynôme à coefficients entiers, que l'on introduit maintenant.

\begin{defi}
Soit $m\geq 1$ un entier. On définit
$$C(m)=\left\{ 
\left(
\begin{matrix}
a & b \\
0 & d 
\end{matrix}
\right)
\in \M_2(\Z)\ :\ ad=m,\ a>0,\ 0\leq b<d,\ \mathrm{pgcd}(a,b,d)=1\right\}.$$
Les $(\sigma^{-1}\Gamma(1)\sigma)\cap \Gamma(1)$ pour $\sigma\in C(m)$ sont exactement les classes à droite de $\Gamma_0(m)$ dans $\Gamma(1)$.
\end{defi}


\begin{thm}[Polynômes modulaires classiques.]

Soit $m\geq 1$ un entier. Il existe un unique  polynôme $\Phi_m \in \Z[X,Y]$, appelé \emph{polynôme modulaire} de degré $m$, tel que
$$\forall \tau\in\H,\ \Phi_m(X,j(\tau))=\prod_{\sigma\in C(m)} (X-j(\sigma\tau)).$$
De plus, on a les propriétés suivantes :

\begin{itemize}
\item[(i)] $\Phi_m(X,Y)=\Phi_m(Y,X)$,
\item[(ii)] $\Phi_m(X,Y)$ est irréductible dans $\Z[X,Y]$.
\end{itemize}

\end{thm}

Ainsi l'image de l'application $Y_0(n)_\C\vers \C^2$ est exactement le lieu des zéros de ce polynôme $\Phi_n$.

Cependant, on ne peut pas vraiment dire que $\Phi_n=0$ définit une équation de la courbe modulaire $Y_0(n)_\C$, car le morphisme $Y_0(n)\vers \C^2$ n'est pas injectif. En effet, il existe des couples de réseaux $(\Lambda, \Lambda')$ tels qu'il existe plusieurs inclusions $\Lambda\vers\Lambda'$ de conoyau $\Z/n\Z$. Ces points singuliers comprennent notamment des points de $Y(1)$ ayant multiplication complexe, notamment par $\Z[i]$ ou $\Z[j]$ ce qui implique l'existence d'automorphismes non triviaux.

Ainsi la courbe $\Phi_n=0$ détermine une courbe plane singulière qui, sur $\C$, est birationnelle à $Y_0(n)$. En dehors des points singuliers, on dispose donc tout de même d'une équation qui permet, étant donnée une courbe, de déterminer les courbes que l'on peut atteindre par une isogénie de degré $n$.

Pour se débarrasser des points singuliers, on pourrait rigidifier la situation en rajoutant une structure de niveau $M$, avec $M$ grand ; on aurait alors une application
$$Y_0(n)(M)_\C\vers Y(M)_\C^2$$
qui est une immersion fermée, mais ce n'est guère utile car la courbe $Y(M)$ n'a pas vraiment d'équation explicite agréable, contrairement à $Y(1)$.

Nous avons travaillé jusqu'à présent sur $\C$, mais il est crucial de savoir que les équations données par le polynôme modulaire $\Phi_n$ restent valables sur un corps quelconque (en particulier pour les corps finis). Pour cela, il faut faire intervenir des notions plus avancées de géométrie algébrique.


\subsection{Le formalisme des problèmes modulaires}

\begin{defi}
Soit $S$ un schéma. On appelle \emph{courbe elliptique} sur $S$ un $S$-schéma
$$\E \overset{\pi}{\vers} S$$
qui est une courbe propre et lisse, dont les fibres géométriques sont connexes de genre 1, munie d'une section notée $0\de S\vers \E$.
\end{defi}

Par \emph{courbe lisse}, on entend que $\pi$ est un morphisme lisse de dimension relative 1 qui est séparé et de présentation finie. Pour ces notions, ainsi qu'à d'autres notions de géométrie algébrique dans ce qui suit, on renvoie à l'appendice.

La condition sur les fibres signifie que pour tout point géométrique $x$ de $S$ (c'est à dire tout morphisme $x\de\Spec k\vers S$ où $k$ est un corps algébriquement clos), la \emph{fibre} de $\E$ en $x$, c'est à dire le $k$-schéma
$$\E \times_S \Spec k\ \vers\ \Spec k$$
est connexe et de genre 1. C'est de plus une courbe propre et lisse (ces propriétés sont préservées par changement de base) munie de la section $0 \times x$, donc une courbe elliptique sur le corps $k$ au sens usuel. Ainsi, on peut voir une courbe elliptique relative $\E/S$ comme une famille de courbes elliptiques \og compatibles \fg\ sur des corps variables, les points géométriques de $S$.

On peut alors montrer que $\E$ admet une structure naturelle de $S$-schéma en groupes commutatif qui étend la structure de schéma en groupe définie sur les fibres géométriques : c'est le théorème d'Abel.

\begin{defi}
On note $(\Ell)$ la catégorie dont les objets sont les courbes elliptiques relatives
$$\E \overset{\pi}{\vers} S$$
pour des schémas arbitraires $S$, et dont les morphismes sont les carrés commutatifs cartésiens
$$
\shorthandoff{;:!?}
\xymatrix @!=8mm {
\E' \ar[d]^{\pi'} \ar[r]  & \E \ar[d]^{\pi} \\
 S' \ar[r] & S
}
$$
c'est à dire ceux pour lesquels on a $\E' \simeq \E \times_S S'$. Le but de cette définition est de cacher le concept de champ algébrique, qui est pourtant la bonne notion dans ce cadre.

On appelle \emph{problème modulaire} pour les courbes elliptiques un foncteur contravariant $\Pr$ de $(\Ell)$ vers la catégorie des ensembles. Un élément de $\Pr(\E/S)$ est appelé \emph{structure de niveau} $\Pr$ sur $\E/S$.

Lorsque l'on se restreint à prendre $S$ dans les $R$-schémas et aux morphismes de $R$-schémas, on obtient la sous-catégorie $(\Ell/R)$. Un \emph{problème modulaire sur $R$} est un foncteur contravariant comme ci-dessus défini sur cette sous-catégorie.
\end{defi}

\begin{ex} On peut définir les analogues des exemples étudiés dans le cas complexe.
\begin{enumerate}
\item Pour tout entier $N\geq 1$, on note $[Y(N)]$ le problème modulaire
$$ \E/S\ \longmapsto \left\{
\begin{matrix}
\text{Morphismes\ de\ schémas\ en\ groupes}\\
\phi\de (\Z/N\Z)^2_S\vers \E[N]\ \\
\text{générateurs\ de}\ \E[N]\ \text{et\ compatibles}\\
 \text{à\ l'accouplement\ de\ Weil}
\end{matrix}
\right\}$$
On peut donner plusieurs définitions de ce que signifie \emph{générateur}, notamment en termes de diviseurs de Cartier, mais la plus immédiate est celle-ci : pour tout point géométrique $x\de\Spec k\vers S$, l'application
$$\phi \times x\de (\Z/N\Z)^2 \vers \E[N](k)$$
est un isomorphisme de groupes. Il s'agit bien d'un problème modulaire, car ces deux conditions sont clairement fonctorielles.

Dans le cas particulier $\Qr_3 = [Y(3)]$, lorsque 3 est inversible, on peut donner une description universelle de ce problème à l'aide d'une équation de Weierstrass, à savoir la courbe
$$E\de y^2 + a_1 x y + a_3 y = x^3$$
où $a_1 = 3C - 1$ et $a_3 = -3 C^2 - B - 3 BC$, munie des deux points
$$P_3 = (0,0), \quad Q_3 = (C, B+C)$$
sur l'anneau
$$\frac{\Z[1/3, B, C][1/(a_1^3 - 27 a_3)a_3 C]}{B^3 = (B+C)^3}.$$
Cela signifie qu'un élément de $\Qr_3(\E)$ peut être identifié avec un morphisme de $S$-schémas de $\E$ vers $E\times S$.
\item On note $[Y_1(N)]$ le problème modulaire du point d'ordre exact $N$ : il s'agit du foncteur
$$ \E/S\ \longmapsto \left\{
\begin{matrix}
\text{Morphismes\ de\ schémas\ en\ groupes}\\
\phi\de (\Z/N\Z)_S\vers \E[N]\ \\
\text{qui\ sont\ une}\ \Z/N\Z \text{-structure\ sur}\ \E[N]
\end{matrix}
\right\}$$
Cela signifie que pour tout point géométrique $x\de\Spec k\vers S$, l'application
$$\phi \times x\de \Z/N\Z \vers E[N](k)$$
est injective.
\item On note $[Y_0(N)]$ le problème modulaire du sous-groupe cyclique d'ordre $N$ :
$$ \E/S\ \longmapsto \left\{
\begin{matrix}
\text{Sous-groupes finis plats}\ K\subset E[N]\\
 \text{localement libres de rang}\ N\ \text{et cycliques}
\end{matrix}
\right\}$$
où \emph{cyclique} signifie que localement f.p.p.f, il admet un générateur dans le sens de ci-dessus. De manière équivalente, on peut voir ce problème comme celui de la $N$-isogénie cyclique, c'est à dire une isogénie $\E \overset{f}{\vers} \E'$ telle que Ker($f$) est cyclique d'ordre $N$.
\item Le problème modulaire de Legendre $\Qr_2$ classifie les courbes munies d'un couple de deux points de 2-torsion indépendants. Lorsque l'on se restreint aux schémas où 2 est inversible, on a une courbe universelle pour ce problème donnée par l'équation de Legendre
$$ E\de y^2 = x (x-1) (x-\lambda)$$
sur l'anneau $\Z[1/2,\lambda][1/\lambda(\lambda -1)].$
\end{enumerate}
\end{ex}

\subsection{Représentabilité et rigidité}

Comme précédemment où l'on classifiait des structures complexes par des points sur des surfaces, on s'intéresse à la représentabilité des problèmes modulaires. Dans l'idéal, on aimerait obtenir des courbes algébriques qui, sur $\C$, sont les surfaces que nous avons trouvées précédemment. Cependant, des obstacles existent.

\begin{defi}
On dit qu'un problème modulaire $\Pr$ est \emph{relativement représentable} si pour toute courbe elliptique $\E/S$, le foncteur des $S$-schémas vers les ensembles
$$T \ \longmapsto\ \Pr(\E\times_S T / T)$$
est représentable par un $S$-schéma noté $\Pr_{\E/S}$. Cela signifie que l'on a un isomorphisme fonctoriel en $T$ :
$$\Pr(\E\times_S T / T) \simeq \Pr_{\E/S}(T) = \Hom_{S-\text{sch.}}(T, \Pr_{\E/S}).$$
Si $\Pr$ est relativement représentable, on dit qu'il vérifie une certaine propriété si tous les morphismes structurels $\Pr_{E/S}\vers S$ ont cette propriété (par exemple, être étale, fini, surjectif, affine).

On dit que $\Pr$ est \emph{représentable} s'il est représentable en tant que foncteur sur $(\Ell$), c'est à dire qu'il existe une courbe elliptique relative
$$E \overset{p}\vers \M(\Pr)$$
telle qu'on ait un isomorphisme fonctoriel dans $(\Ell)$ :
$$\Pr(\E/S) \simeq \Hom_{(\Ell)}(\E/S, E/\M(\Pr)).$$
Dans ce cas, on dispose d'un élément universel $a\in \Pr(E/\M(\Pr))$ induit par l'identité de $E/\M(\Pr)$. 
\end{defi}

\begin{ex}
Le problème modulaire $\Qr_3$ de niveau 3 sur $\Z[1/3]$ est représentable par $E/\M(\Qr_3)$, la courbe universelle donnée à l'exemple précédent. De même, le problème de Legendre $\Qr_2$ est représentable sur $\Z[1/2]$ par la courbe universelle donnée sous forme de Legendre.
\end{ex}

En dehors de ces exemples très simples, il est difficile d'exhiber des représentants donnés par des équations explicites : il faut recourir à des arguments plus abstraits. Un premier pas vers la représentabilité consiste à montrer la relative représentabilité.

\begin{lem}
Soit $N\geq 1$ un entier et $S$ un schéma dans lequel $N$ est inversible (c'est à dire un schéma sur $\Z[1/N]$). Soit $\E/S$ une courbe elliptique. Alors les foncteurs
$(S\text{-}\mathrm{sch})\vers(\mathrm{Ens})$
$$T \longmapsto
\begin{cases}
Y(N)\\
Y_1(N) \qquad \text{-structures sur}\ \E \times_S T/T \\
Y_0(N)
\end{cases}$$
sont représentables par un $S$-schéma fini étale. Autrement dit, les problèmes modulaires $[Y(N)]$, $[Y_1(N)]$ et $[Y_0(N)]$ sur $\Z[1/N]$ sont relativement représentables, finis et étales.
\end{lem}

\begin{proof}
Comme $N$ est inversible, le schéma en groupes $\E[N]$ est étale, et localement isomorphe au schéma constant $(\Z/N\Z)^2$ pour la topologie étale. La représentativité des deux premiers foncteurs devient alors purement formelle. C'est un peu plus délicat pour le troisième : on renvoie à \cite{KaMa}, (3.7).
\end{proof}

\begin{rem} Si le problème $\Pr$ est représentable par $E\vers \M(\Pr)$, alors $\M(\Pr)$ représente le foncteur suivant dans la catégorie des schémas :
$$S\ \longmapsto\ \left.
\begin{cases}\ \text{Classes\ d'isomorphisme\ des\ paires}\ (\E/S,\alpha)\ \text{où}\ \\
 \ \E/S\ \text{est\ une\ courbe\ elliptique\ relative\ et}\ \alpha\in \Pr(\E/S)
\end{cases} \right\}.$$
En effet, si l'on désigne par $a$ la $\Pr$-structure canonique de $E/\M(\Pr)$, les applications
$$\begin{aligned}
(\E/S, \alpha)\ &\overset{\phi}{\longmapsto}  &(\alpha\de S\vers \M(\Pr)) \\
(E \times_{\M(\Pr)} S,\ \Pr(f)(a))\ &\overset{\psi}{\mathrel{\reflectbox{\ensuremath{\longmapsto}}}} &(f\de S\vers \M(\Pr))
\end{aligned}$$
sont des isomorphismes réciproques. Vérifions-le : par fonctorialité, l'application
$$\Pr(E/\M(\Pr)) \overset{\Pr(f)} \vers \Pr(E\times S/S)$$
utilisée dans la définition de $\psi$ est la précomposition par le changement de base par $f$ :
$$\Hom(E/\M(\Pr), E/\M(\Pr)) \vers \Hom(E\times S/S, E/\M(\Pr)).$$
Ainsi, comme $a$ est l'identité dans cette description, $\Pr(f)(a)$ est le changement de base par $f$, et ainsi on a bien $\phi\circ\psi = \id.$ Réciproquement, on se donne $(\E/S, \alpha)$ et on veut montrer que cette paire est isomorphe à
$$(E \times_{\M(\Pr)} S,\ \Pr(\alpha)(a)).$$
On a ici identifié $\alpha\in \Pr(\E/S)$ à une flèche $\E/S \vers E/\M(\Pr)$, car le foncteur $\Pr$ est représentable. D'un côté, on a un carré cartésien
$$
\shorthandoff{;:!?}
\xymatrix @!=8mm {
\E \ar[d]^{\pi} \ar[rr] & & E \ar[d]^{p} \\
 S \ar[rr]^{\alpha} & & \M(\Pr)
}
$$
donné par $\alpha$ qui est un morphisme de $(\Ell)$, et de l'autre, un carré cartésien
$$
\shorthandoff{;:!?}
\xymatrix @!=8mm {
E\times_{\M(\Pr)} S \ar[d]^{\pr_2} \ar[rr]^{\pr_1} & & E \ar[d]^{p} \\
 S \ar[rr]^{\alpha} & & \M(\Pr)
}
$$
Cela montre l'isomorphisme demandé par unicité du produit, et le fait que les structures de niveau $\Pr$ coïncident provient du fait que $\Pr(\alpha)(a)$ est bien le changement de base par $\alpha$.
\end{rem}

On peut également montrer qu'un foncteur représentable $\Pr$ par $E/\M(\Pr)$ est relativement représentable, pour toute courbe elliptique $E/S$, par le schéma
$$\Pr_{E/S} = \underline{Isom}(????????)$$

De plus, si $\Pr$ est représentable et $\Pr'$ est relativement représentable, on montre que le problème modulaire \og simultané\fg\ $\Pr\times\Pr'$ est représentable par le schéma 
$$\M(\Pr, \Pr') = \Pr'_{E/\M(\Pr)}.$$

On peut noter une première obstruction à la représentabilité de certains problèmes modulaires : on dit que $\Pr$ est \emph{rigide} si pour toute structure $\alpha$ de niveau $\Pr$ sur $\E/S$, la paire $(\E/S, \alpha)$ n'admet pas d'automorphismes non triviaux. On a alors le résultat suivant :

\begin{lem}
Tout problème modulaire représentable est rigide.
\end{lem}

\begin{proof}
Si $\Pr$ un problème modulaire représentable par $E\vers \M(\Pr)$, et si $\phi\in \mathrm{Aut}(\E/S)$ laisse la $\Pr$-structure $\alpha\in \Hom_{(\Ell)}(\E/S, E/\M)$ invariante, alors on a un diagramme commutatif :

$$
\shorthandoff{;:!?}
\xymatrix @!=2mm {
& & \E \ar[ddd]^{\pi} \ar[rrr]^{\alpha} & & & E \ar[ddd]^{p} \\
 \\
\E \ar[ddd]^{\pi} \ar[rrr]^{\alpha} \ar@{.>}[uurr]^\phi & & & E \ar[ddd]^{p} \ar@{.>}[uurr]^\id & & \\
& & S \ar[rrr]^{\alpha} & & & \M(\Pr) \\
 \\
 S \ar[rrr]^{\alpha} \ar@{.>}[uurr]^\id & & & \M(\Pr) \ar@{.>}[uurr]^\id & &
}
$$
d'où l'on déduit $\phi = \id$, car les carrés avant et arrière sont cartésiens.
\end{proof}

Un fait important est que cette obstruction est essentiellement la seule.

\begin{thm}
Tout problème modulaire affine relativement représentable et rigide est représentable.
\end{thm}

\begin{proof}
On cherche à construire un $\Z$-schéma $\M(\Pr)$ qui représente le foncteur $\Pr$. Pour cela, il suffit de montrer que $\Pr$ est représentable à la fois sur $\Z[1/2]$ et $\Z[1/3]$ ; la rigidité de $\P$ permettra ensuite de recoller les deux schémas obtenus en un schéma sur $\Z$, puisque leurs restrictions à $\Z[1/6]$ devront coïncider.

Pour ces deux questions, on utilise le lemme suivant :

\begin{lem}
Soit $N\geq 1$ un entier, $G$ un groupe fini (concret), et $\Qr$ un problème modulaire affine et relativement représentable. On suppose de plus que $\Qr$ est représentable par un schéma affine sur $\Z[1/N]$ et que $G$ agit sur $\Qr$ de telle sorte que pour toute courbe elliptique
$$\E\vers S\vers \Spec \Z[1/N],$$
le $S$-schéma $\Qr_{\E/S}$ soit un $G$-torseur fini étale. Alors $\Pr$ est représenté par le $\Z[1/N]$-schéma affine
$$\M(\Pr,\Qr)/G.$$
\end{lem}

On l'applique ensuite d'une part au problème modulaire de Legendre $\Qr_2$ pour $N = 2$, et au problème de niveau 3 $\Qr_3$ pour $N = 3$. La notion de torseur nécessite sans doute une définition.

\begin{defi}
Soit $X$ un $S$-schéma, et $G$ un schéma en groupes sur $S$. On dit que $X$ est un $G$-torseur fini étale si :
\begin{itemize}
\item $X$ est un $S$-schéma fini étale.
\item On a une action de $G$, donc une flèche
$$\begin{aligned}
X \times_S G &\vers X \times X \\
(x, g) &\longmapsto (x, g\cdot x)
\end{aligned}$$
\item Celle-ci soit un isomorphisme de $S$-schémas.
\item $X$ est localement trivial (par exemple, pout la topologie fppf ou fpqc), c'est à dire qu'il existe localement une section de la flèche $G\times X\vers X$.
\end{itemize}
On dit également que $\Qr_{\E/S}$ est un \emph{espace principal homogène sous $G$}. Dans notre situation, $G$ est un groupe concret fini, donc on choisit simplement le schéma en groupe constant $G_S$ dans cette définition.
\end{defi}

\emph{Démonstration du lemme.} On a vu que le problème $(\Pr, \Qr)$ est représentable par
$$\M(\Pr, \Qr) = \Pr_{E/\M(\Qr)}.$$
Comme $\Pr$ est supposé affine, ce schéma est affine sur $\M(\Qr)$ qui est lui-même affine, donc est affine. De plus, le groupe $G$ agit sur $\M(\Pr, \Qr)$, car celui-ci représente le foncteur $\Pr\times\Qr$, et on peut faire agir fonctoriellement $G$ sur la composante $\Qr$.

\newcommand{\univ}{\mathrm{univ}}

Considérons alors la courbe elliptique universelle pour $\Pr\times\Qr$ :
$$E \vers \M(\Pr,\Qr)$$
munie de $(\alpha_{\univ}, \beta_{\univ}) \in (\Pr\times\Qr)(E/\M(\Pr,\Qr)).$ La preuve se déroule alors en trois temps :
\begin{itemize}
\item[•] Montrer l'existence du quotient $\M(\Pr,\Qr)/G$ ;
\item[•] Descendre $(E, \alpha_{\univ})$ à $\M(\Pr,\Qr)/G$, c'est à dire remplir le diagramme cartésien
$$
\shorthandoff{;:!?}
\xymatrix {
(E,\alpha_{\univ}) \ar[d] \ar@{.>}[r] &\ \textbf{?}\ \ar@{.>}[d] \\
 \M(\Pr,\Qr) \ar[r] & \M(\Pr,\Qr)/G
}
$$
\item[•] Montrer que la courbe descendue représente bien le problème modulaire $\Pr$ sur $\Z[1/N]$.
\end{itemize}

Pour le premier point, il suffit de montrer que le groupe fini $G$ agit librement sur le schéma affine $\M(\Pr,\Qr)$; alors on sait que le quotient existe par le théorème de Grothendieck, et que la projection
$$\M(\Pr,\Qr)\vers \M(\Pr,\Qr)/G$$
est un $G$-torseur fini étale (et surjectif, donc fidèlement plat). Pour cela, il suffit de montrer que l'action est \emph{universellement libre}. Soit $T$ un $\Z[1/N]$-schéma. On a vu que $\M(\Pr, \Qr)(T)$ est l'ensemble des courbes elliptiques sur $T$ munies d'une $(\Pr,\Qr)$-structure à isomorphisme près. Si un élément $g\in G$ laisse un élément invariant, cela signifie que l'on a un certain isomorphisme
$$(\E/T, \alpha, \beta) \simeq (\E/T, \alpha, g\beta).$$
Par rigidité de $\Pr$, cet isomorphisme doit être l'identité. Ainsi on a $g\beta = \beta$, et par hypothèse $\Qr_{E/T}$ est un $G$-torseur, ce qui implique que $g$ est lui-même trivial. Cela conclut la preuve que l'action de $G$ sur $\M(\Pr,\Qr)$ est libre.

Pour descendre $E, \alpha_{\univ}$ par ce quotient, on est dans le cadre général de la descente fidèlement plate. On cherche une \emph{donnée de descente}, c'est à dire un ensemble d'isomorphismes pour $g\in G$:
$$\theta(g)\ :\ g^*(E, \alpha_{\univ}) \overset{\sim}{\vers} (E, \alpha_{\univ})$$
qui forment un 1-cocycle (i.e. ils sont compatibles à la composition). On peut obtenir une telle donnée de la façon suivante. Pour tout $g\in G$, $(E, \alpha_{\univ}, g\beta_{\univ})$ est une courbe elliptique sur $\M(\Pr, \Qr)$ munie d'une $(\Pr,\Qr)$-structure, donc est classifiée par un unique morphisme
$$\theta(g) \de \M(\Pr,\Qr)\vers \M(\Pr,\Qr)$$
qui donne un isomorphisme au-dessus de $\M(\Pr,\Qr)$:
$$g^*(E, \alpha_\univ, \beta_\univ) \overset{\sim}{\vers} (E, \alpha_\univ, g\beta_\univ)$$
et il suffit s'oublier $\beta_\univ$. La compatibilité avec la composition provient directement de la rigidité de $\Pr$.

Alors $E$ descend car elle est projective, donc propre (c'est le deuxième cas d'application du théorème de Grothendieck, avec les schémas affines). Comme $\Pr$ est relativement affine, $\alpha_\univ$ descend également. (?) On a ainsi rempli le carré cartésien précédent avec un objet $(E_0,\alpha_{\univ, 0})$ sur $\M(\Pr,\Qr)/G$.

Montrons maintenant que l'on obtient ainsi un représentant du problème $\Pr$. Soit $S$ un $\Z[1/N]$-schéma et $\E/S$ une courbe elliptique munie d'une structure $\alpha$ de niveau $\Pr$. On a un $G$-torseur fini étale sur $S$ : $\Qr_{\E/S}\vers S$ sur lequel $\E$ acquiert sa structure universelle de niveau $\Qr$ notée $\beta_\univ$. Alors on dispose d'une $(\Pr\times\Qr)$-structure
$$(E\times_S \Qr_{E/S}, \alpha, \beta_\univ).$$
On a donc un morphisme de classification associé $f \de \Qr_{\E/S}\vers \M(\Pr,\Qr)$. Il est tautologiquement $G$-équivariant donc en passant au quotient (?), il existe un morphisme $f_0$ rendant le diagramme commutatif :

$$
\shorthandoff{;:!?}
\xymatrix {
\Qr_{\E/S} \ar[d]_\pi \ar[r]^f &\ \M(\Pr,\Qr) \ar[d] \\
 S \ar[r]^{f_0} & \M(\Pr,\Qr)/G
}
$$

Montrons que $f_0$ est le morphisme de classification de $(\E, \alpha)$ recherché, c'est  à dire :
\begin{enumerate}
\item $f_0^*(E_0, \alpha_{\univ, 0})$ est isomorphe à $(E, \alpha)$.
\item $f_0$ est l'unique application vérifiant cette propriété.
\end{enumerate}
Pour 1., comme $\pi$ est étale et surjective (donc fidèlement plate), et comme $\Pr$ est rigide, il suffit de montrer que $\pi^*f_0^*(E_0, \alpha_{\univ, 0})$ et $\pi^*(E,\alpha)$ sont isomorphes. C'est clair vu la définition de $f$.

Regardons maintenant 2. Soit $h_0$ un morphisme satisfaisant cette propriété, et notons $X$ le produit fibré :
$$
\shorthandoff{;:!?}
\xymatrix {
X \ar[d] \ar[r]^h &\ \M(\Pr,\Qr) \ar[d]^{\pi_\univ} \\
 S \ar[r]^{h_0} & \M(\Pr,\Qr)/G
}
$$
Alors $X/S$ est un $G$-torseur, et le pullback de $(E/S, \alpha)$ à $X$ acquiert une $\Qr$-structure $\beta$ (?). Cette $\Qr$-structure est classifiée par un morphisme $G$-équivariant :
$$
\shorthandoff{;:!?}
\xymatrix {
X \ar[dr] \ar[rr] & &\ \Qr_{E/S} \ar[dl] \\
& S &
}
$$
qui est un isomorphisme (?), étant une application $G$-équivariante entre $G$-torseurs. Ainsi, on a un diagramme cartésien de $G$-torseurs :
$$
\shorthandoff{;:!?}
\xymatrix {
\Qr_{\E/S} \ar[d]_\pi \ar[r]^h &\ \M(\Pr,\Qr) \ar[d]^{\pi_\univ} \\
 S \ar[r]^{h_0} & \M(\Pr,\Qr)/G
}
$$
et un isomorphisme $h^*(E_\univ, \alpha_\univ, \beta_\univ) \simeq (E,\alpha, \beta_\univ)$ sur $\Qr_{E/S}$. Ainsi, on doit avoir $h = f$ puisque ces deux applications classifient la structure $(E, \alpha, \beta_\univ)$. Par commutativité du diagramme ci-dessus, on en déduit $h_0\pi = f_0 \pi$, d'où $f_0 = h_0$ car $\pi$ est étale et surjective, ce qui termine la preuve.
\end{proof}

\begin{cor}
Sous les hypothèses du théorème précédent, si de plus $\Pr$ est étale, alors $\M(\Pr)$ est une courbe affine lisse sur $\Z$.
\end{cor}
En effet, on a obtenu dans la preuve une construction du schéma affine $\M(\Pr)$ par recollement de quotients finis de $\M(\Pr, \Qr_2)$ et $\M(\Pr, \Qr_3)$. Or, les deux morphismes
$$\M(\Pr, \Qr_2)\vers \M(\Qr_2),\quad \M(\Pr, \Qr_3)\vers \M(\Qr_3)$$
sont étales, car $\Pr$ est supposé étale. Il suffit donc de montrer que $\M(\Qr_2)$ et $\M(\Qr_3)$ sont des courbes lisses sur $\Z$, car cette propriété est préservée par les morphismes étales et les quotients par un groupe fini agissant librement. Cette dernière assertion est évidente lorsque l'on regarde les anneaux en question.

\begin{cor}
Pour tout entier $N\geq 3$, les problèmes modulaires $[Y(N)]$ et $[Y_1(N)]$ sont représentables par des courbes affines lisses sur $\Z$, notées respectivement $Y(N)$ et $Y_1(N)$.
\end{cor}

En effet, on a vu que ces deux problèmes modulaires sont relativement représentables, finis et étales donc en particulier affines. D'autre part, ils sont rigides (c'est une conséquence de la théorie de Hasse et du fait que le degré est une application quadratique) : on peut donc appliquer le résultat précédent.

Ainsi, le problème que nous nous étions posés d'étendre des résultats obtenus sur $\C$ est résolu pour ces deux problèmes rigides. En revanche le résultat précédent ne s'applique pas à $[Y_0(N)]$, qui est non rigide car l'action de $[-1]$ est triviale, ni à $[Y(1)]$ par exemple : ces problèmes modulaires ne sont tous simplement pas représentables. Il faudrait parler de champs algébriques pour étudier leur représentabilité, mais on se contente ici d'introduire brièvement la notion d'espaces de modules grossiers, qui permettent de montrer que tout se passe raisonnablement bien lorsque l'on regarde des corps uniquement.


\subsection{Espaces de modules grossiers}


On souhaite trouver tout de même des espaces qui paramétrisent un problème modulaire comme $[Y_0(N)]$, qui n'est pas représentable. On utilise alors la construction suivante.

\begin{defi}
Soit $R$ un anneau, et $\Pr$ un problème modulaire relativement représentable et affine sur la catégorie $(\Ell/R)$. On construit l'\emph{espace de modules grossiers} $M(\Pr)$ de la façon suivante : localement sur $R$, on peut supposer qu'un entier $N\geq 3$ est inversible. Soit $\Qr$ un problème modulaire représentable fini, étale et galoisien sur $(\Ell/R)$, par exemple $[Y(N)]$ de groupe de Galois $GL(2, \Z/N\Z)$. On définit alors $M(\Pr)$ comme le schéma quotient (dont on a démontré l'existence lors de la preuve du théorème précédent) :
$$M(\Pr) = \M(\Pr,\Qr)/G.$$
Le schéma obtenu est indépendant du choix de $\Qr$ : en effet,

On peut donc recoller ces schémas en un schéma sur tout $R$.
\end{defi}

Ce $R$-schéma est simplement $\M(\Pr)$ si $\Pr$ est représentable (dans la section précédente, on a montré que si $\Pr$ est rigide et affine, alors il est représenté par le schéma $\M(\Pr,\Qr)/G$). Si $\Pr$ n'est pas représentable, il fait office de meilleur remplaçant. Le point qui nous intéresse est que ce schéma, même s'il ne représente pas $\Pr$, a une interprétation en termes modulaires :

\begin{prop}
Si $k$ est un corps algébriquement clos, alors on a une identification :
$$ M(\Pr)(k) = 
\left.
\begin{cases}
\ \text{Courbes\ elliptiques}\ \E/k \ \text{munies\ d'une\ structure} \\
\ \text{de\ niveau}\ \Pr,\ \text{à}\ k\text{-isomorphisme\ près}.
\end{cases}
\right\}$$
Si $k$ est un corps quelconque et $\bar{k}$ une clôture algébrique de $k$, alors on a l'identification
$$ M(\Pr)(k) = 
\left.
\begin{cases}
\ \text{Courbes\ elliptiques}\ \E/k \ \text{munies\ d'une\ structure} \\
\ \text{de\ niveau}\ \Pr,\ \text{à}\ \bar{k}\text{-isomorphisme\ près}.
\end{cases}
\right\}$$
\end{prop}

\begin{proof}
Dans le cas où $k$ est un corps algébriquement clos, le lemme suivant montre que la prise des $k$-points est compatible avec le quotient par $G$, i.e. on a une bijection
$$\M(\Pr, [Y(N)])(k)/G \overset{\sim}{\vers} M(\Pr)(k).$$
Or le terme de gauche est exactement l'ensemble des classes de $k$-isomorphisme de courbes elliptiques $\E/k$ munies simultanément d'une $\Pr$- et d'une $[Y(N)]$-structure, et quotienter par $G$ revient à oublier la seconde.

Dans le cas où $k$ n'est pas algébriquement clos, les $k$-points de $M(\Pr)$ sont exactement les $\bar{k}$-points qui se factorisent par $\Spec\bar{k}\vers\Spec k$, c'est à dire tels que la courbe $\E$ et la structure de niveau $\Pr$ peuvent être définis sur $k$. Cela prouve le second point. 
\end{proof}

\begin{lem}
Soit $R$ un anneau, $A$ une $R$-algèbre, et $G$ un groupe fini agissant $R$-linéairement sur $A$. On note $A^G \subset A$ la sous-algèbre des $G$-invariants. Alors si $R'$ est une $R$-algèbre, le morphisme naturel
$$A^G\otimes_R R' \vers (A\otimes_R R')^G.$$
induit une bijection sur les points géométriques.
\end{lem}

Pour montrer la bijection précédente, on prend $\Spec A = \M(\Pr, [Y(N)])$ qui est une $R$-algèbre, et on choisit un point géométrique $\Spec k\vers \Spec A \vers\Spec R$. Par le lemme,
$$(\Spec A \times_R k)/G \simeq (\Spec A^G) \times_R k.$$
On dit alors simplement que les points fermés de ces deux schémas sont en bijection.
\begin{proof}[Démonstration du lemme]

\end{proof}

\subsection{Le théorème d'Igusa}

Récapitulons.

\begin{thm} Soit $N\geq 3$ un entier. Il existe un schéma 
$$Y_0(N)\vers \Spec \Z[1/N]$$
ayant les propriétés suivantes.
\begin{itemize}
\item[(i)] Pour tout corps $k$ dans lequel $N$ est inversible, les $k$-points de $Y_0(N)$ sont en bijection avec les classes d'isomorphisme (sur $\bar{k}$) d'isogénies cycliques de degré $N$ entre deux courbes elliptiques sur $k$.
\item[(ii)] $Y_0(N)\vers \Spec \Z[1/N]$ est lisse.
\item[(iii)] On dispose d'un morphisme de schémas 
$$Y_0(N) \vers Y(1)\times Y(1)$$
où $Y$ est l'espace affine de dimension 1, qui sur $\C$ devient l'application étudiée précédemment.
\end{itemize}

\end{thm}

\begin{proof}
$(i)$ est déjà vu. Pour $(ii)$, on note que $Y_0(N)$ est obtenu comme quotient du schéma $\M(Y_0(N), Y(N))$ par un groupe fini agissant librement, et ce schéma est fini étale sur $Y(N)$. Ainsi, il s'agit bien d'une courbe lisse et normale. (?)

Le $(iii)$ est obtenu par le morphisme fonctoriel qui à $(\E,S)$ associe $(\E, \E/S)$. (Pb car justement on n'a pas la représentation ?)
\end{proof}

Il découle alors formellement de ce théorème, et du cas complexe étudié ci-dessus, que l'image schématique de $Y_0(N)$ dans $Y\times Y$ est exactement la courbe définie par le polynôme $\Phi_N$.

Notons $Z$ l'image de $Y_0(N)$, $S$ la courbe définie par $\Phi_N$. Le cas complexe montre que $Z$ et $S$ ont même fibre générique. D'autre part, on a vu que $\Phi_N$ est irréductible donc $S$ est plat, et d'autre part $Z$ est plat car $Y_0(N)$ et $Y$ le sont. Ainsi ces deux schémas sont l'adhérence de leur fibre générique, donc sont égaux.

Il reste également à caractériser les points singuliers de $Y_0(N)\vers Y\times Y$ : on peut en fait montrer que ceux-ci sont des réductions de points singuliers sur $\Z$, c'est à dire les mêmes que dans le cas complexe. En effet, soit $x$ un point singulier de $Y_0(N)$ en caractéristique $p$ qui n'est pas réduction d'un point singulier sur $\Z$. Alors on a un ouvert Zariski autour de ce point qui serait régulier en codimension 1. Cela montre par des arguments classiques que la courbe donnée par $\Phi_N=0$ est normale sur $\F_p$. Or la normalisation de $\Phi_N=0$ est précisément la courbe modulaire $Y_0(N)$ qui est lisse : contradiction.

Ainsi en dehors des points singuliers, le polynôme modulaire permet bien de caractériser les isogénies de degré $l$ sur tout corps.

\newpage

\section{Cryptosystème et algorithmes}



\subsection{Diffie-Hellman et actions de groupes}


La sécurité de nombreux cryptosystèmes à clefs publiques est basée sur la difficulté supposée d'un problème mathématique. On considère qu'un tel problème est \emph{difficile} s'il n'existe pas d'algorithme fonctionnant en temps polynomial qui, recevant en entrée les données du problème, en donne la solution avec probabilité supérieure à $\frac{1}{2}+\varepsilon(n)$ (pour un problème décisionnel) ou $\varepsilon(n)$ (pour un problème computationnel), où $\varepsilon$ est une fonction négligeable devant les polynômes et $n$ est le \emph{paramètre de sécurité} du système, typiquement la taille des données (par exemple $\log(N)$ pour un groupe de cardinal $N$).

L'un des types de problèmes que l'on peut utiliser de fonde sur la difficulté d'"inverser" l'action d'un groupe abélien, et se formule de la manière suivante :

\textbf{Problème HHS}, pour \og hard homogeneous spaces \fg. \'Etant donné un groupe abélien $G$ agissant sur un ensemble $X$ de façon simplement transitive, et des éléments $x_0, a\cdot x_0, b\cdot x_0\in X$, calculer $ab\cdot x_0$.

On peut aussi donner une version décisionnelle de ce problème. On dispose ensuite d'un échange de clé naturel : $G, X$ et $x_0$ étant des données publiques, Alice et Bob choisissent chacun $a,b\in G$ et s'échangent $a\cdot x_0$, $b\cdot x_0$, et peuvent alors chacun calculer $a\cdot b\cdot x_0=b\cdot a\cdot x_0$, puisque le groupe $G$ est abélien. La sécurité de ce protocole repose notamment sur la non-existence d'algorithmes polynomiaux calculant $g$ à partir de $x_0$ et $g\cdot x_0$.

Le cryptosystème décrit dans l'article de Rostovtsev et Stolbunov \cite{RoSt}, suivant une idée de Couveignes \cite{Couv}, consiste à utiliser l'action du groupe de classe d'un ordre quadratique $\O$ sur l'ensemble $\Ell_t(\O)$ lorsqu'il est non vide, où $k$ est un corps fini. La question se pose alors de la représentation, lors du calcul, des objet mathématiques utilisés.

\vspace{5mm}

\subsection{Représentation des objets mathématiques}

\begin{defi}
On fixe un corps fini $k$ de cardinal $q$ et de caractéristique $p\geq 5$, un entier $t\leq 2\sqrt{q}$. On note $K$ le corps quadratique
$$K = \Q[\pi] / (\pi^2 - t\pi + q),$$
on choisit une courbe elliptique $\E_0/k$ dont la trace du Frobenius est égale à $t$, et on note $\O$ l'ordre de $K$ qui est son anneau d'endomorphismes.

On dit qu'un nombre premier $\ell$ est \emph{Elkies} si le polynôme $X^2 - tX + q$ est scindé à racines simples modulo $\ell$, c'est à dire si son discriminant est un carré non nul modulo $\ell$.
\end{defi}

\emph{Courbes elliptiques.} On utilise deux formes de représentations de courbes elliptiques : la forme de Weierstrass réduite
$$\E\de y^2 = x^3 + Ax + B$$
ou la forme dite de Montgomery
$$\E\de B y^2 = x^3 + A x^2 + x.$$
Notons que toute courbe elliptique sur $k$ n'admet pas nécessairement de représentation sous forme de Montgomery : en effet, une telle courbe admet en particulier un point rationnel de 2-torsion qui est $(0,0)$.
On s'autorise également à représenter une courbe elliptique à isomorphisme près par son $j$-invariant, qui est un élément de $k$. Ce faisant, on perd de l'information par rapport aux équations ci-dessus, car on ne peut plus distinguer une courbe de sa tordue.

\emph{Sous-groupes d'une courbe elliptique.} Si $S\geq E$ est un sous-schéma en groupe fini (et plat) d'ordre $\ell$, où $p\neq \ell$ et $\ell$ est impair, on représente le sous-groupe $S$ de la façon suivante. $S$ est étale donc déterminé par ses $\bar{k}$-points, et est stable par l'automorphisme $[-1]$ de $E$.
En écrivant une équation pour $E$, il existe donc un polynôme $K_S\in k[X]$ de degré $\frac{\ell-1}{2}$ et dont les racines sont exactement les coordonnées $x$ des points de $S$. On représente alors $S$ par ce polynôme.

\emph{Isogénies.} Comme dit précédemment, une isogénie séparable $\phi$ est déterminée par sa courbe de départ et son noyau. Si elle est de degré impair, on peut donc la représenter par un polynôme comme ci-dessus, que l'on appelle \emph{polynôme de noyau}, noté $K_\phi$.

\emph{Idéaux.} Si $\ell$ est un nombre premier d'Elkies, alors $\ell$ se scinde dans $\O$ sous la forme $(\ell) =~\frak l \bar{\frak l},$
et ces deux idéaux sont exactement les idéaux de norme $\ell$ dans $\O$. Pour les distinguer, on peut utiliser la remarque suivante. Si $E$ est une courbe elliptique ayant les bons paramètres, on sait que $E[\ell](\bar{k})$ est un $\O/\ell\O$-module de rang 1. D'autre part, on a un isomorphisme canonique par le théorème chinois :
$$\O/\ell\O \simeq \O/\frak l \O \times \O/\bar{\frak l} \O.$$
L'endomorphisme de Frobenius $\pi$, vu comme endomorphisme du $\F_\ell$-espace vectoriel $E[\ell](\bar{k})$ de dimension 2, admet donc deux valeurs propres (nécessairement distinctes car $\ell$ est d'Elkies) qui sont exactement les éléments $\pi\mod \frak l$ et $\pi\mod \bar{\frak l}$ : on les appelle \emph{valeurs propres du Frobenius} modulo $\ell$. On représentera alors un idéal de norme $\ell$ par le couple $(\ell, v)$ où $v$ est la valeur propre du Frobenius associée. D'un point de vue pratique, les valeurs propres du Frobenius sont les racines dans $\F_\ell$ du polynôme $X^2 - tX + q$ ; on reviendra plus tard sur le calcul de $t$.

\emph{Idéaux fractionnaires.} On utilisera toujours des idéaux fractionnaires écrits sous la forme $\prod {\frak l}_i^{r_i}$, où les ${\frak l}_i$ sont des idéaux dont la norme est un petit nombre premier d'Elkies. Cela implique, en particulier, que l'on ne travaille qu'avec un sous-groupe du groupe des idéaux fractionnaires.

\emph{\'Elements du groupe de classes}. On choisit de donner un élément de $\Cl(\O)$ par un de ses représentants, que l'on écrit comme ci-dessus. Notons que l'on pourrait toujours atteindre le groupe de classes en entier même en se restreignant à un sous-groupe des idéaux fractionnaires ; on reviendra sur cette question plus loin.

Nous décrivons maintenant les différents moyens de calculer l'action du groupe de classes sur $\Ell_t(\O)$.



\subsection{Calcul à l'aide d'un polynôme de division}

Si $\phi\de E\vers E'$ est une isogénie séparable de degré $\ell$, alors son noyau est constitué de points de $E$ d'ordre $\ell$. Le polynôme $K_\phi$ est donc un facteur de degré $\frac{\ell-1}{2}$ à coefficients dans $k$ du $\ell$-ième \emph{polynôme de division} de la courbe $E$. Avec cette idée, on obtient l'algorithme suivant.

\begin{algorithm}
\caption{{\sc Division} : un pas dans le graphe d'isogénies à l'aide des polynômes de division}
\label{alg:div}
\KwIn{Une courbe elliptique $E\in \Ell_t(\O)$, et un idéal représenté par $(\ell, v)$}
\KwOut{Une courbe elliptique $E'$ telle que l'isogénie $E\vers E'$ corresponde à cet idéal}
$\psi_\ell \gets \textsc{PolynômeDeDivision}(E, \ell)$
\label{alg:div:poldiv}
\;
$L \gets \textsc{Factorisation}(\psi_\ell)$
\label{alg:div:fact}
\;
$(K_1, K_2) \gets \textsc{RetrouveDegré}(L, \frac{\ell-1}{2})$
\label{alg:div:rec}
\;
\lIf{$\textsc{CompareDirection}(K_1,v)$}{\Return{$\textsc{Quotient}(E, K_1)$}}
\label{alg:div:check}
\lElse{\Return{$\textsc{Quotient}(E, K_2)$}}

\end{algorithm}

Seul le but de l'étape~\ref{alg:div:rec} nécessite peut-être une explication : il consiste à retrouver à partir des facteurs irréductibles de $\psi_l$ les deux polynômes de degré $\frac{\ell-1}{2}$ définissant des sous-groupes cycliques de $E$. Cette étape est nécessaire car ces polynômes ne sont pas nécessairement irréductibles, et $\psi_l$ peut avoir beaucoup d'autres facteurs qui ne déterminent pas des sous-groupes mais seulement des sous-ensembles de $E[\ell](\bar{k})$.

Examinons maintenant chacune des étapes de l'algorithme~\ref{alg:div}.

\begin{lem}[Polynômes de division]
Soit $E$ une courbe elliptique sous forme de Weierstrass réduite :
$$y^2 = x^3 + Ax + B.$$
Il existe des polynômes dits \emph{de division} $\Psi_n \in \Z[x,y]/(y^2 - x^3 - Ax - B)$, $n\geq 1$ tels que pour tout $n\geq 1$, l'endomorphisme $[n]_E$ soit donné sur l'ouvert affine par les applications rationnelles
$$(x,y) \longmapsto \left(\frac{x\Psi_n^2 - \Psi_{n-1}\Psi_{n+1}}{\Psi_n^2},
\frac{\Psi_{n+2}\Psi_{n-1}^2 - \Psi_{n-2} \Psi_{n+1}^2}{4y \Psi_n^3}\right),$$
en posant $\Psi_0 = 0,\ \Psi_{-1} = -1,$ et tels que l'on ait :
$$\begin{aligned}
 \Psi_1&= 1,\\
 \Psi_2&= 2y,\\
 \Psi_3&= 3x^4 + 6Ax^2 + 12Bx - A^2, \\
 \forall n\geq 2,\ 2y\Psi_{2n} &= \Psi_n(\Psi_{n+2}\Psi_{n-1}^2 - \Psi_{n-2} \Psi_{n+1}^2), \\
\forall n\geq 2 ,\ \Psi_{2n+1} &= \Psi_{n+2}\Psi_n^3 - \Psi_{n+1}^3\Psi_{n-1}.
\end{aligned}$$
\end{lem}

Lorsque $n$ est impair, on montre aisément par récurrence que $\Psi_n$ peut être vu comme un polynôme en $x$ uniquement et de degré $\frac{n^2-1}{2}$. Si de plus $p\nmid n$, on sait que $\#E[n](\bar{k})=n^2$, donc $\Psi_n$ est un polynôme séparable dont les racines sont exactement les coordonnées $x$ des points de $n$-torsion de la courbe. 

\begin{proof}
Il s'agit d'une simple récurrence en utilisant la forme explicite de la loi de groupe sur la courbe. Rappelons les formules d'addition sur une courbe elliptique en forme de Weierstrass, données par le mécanisme \og corde et tangente \fg\ : si $P$, $Q$ sont des points affines de $E(\bar{\F_p})$ de coordonnées, $(x_P, y_p),\ (x_Q, y_Q)$ avec $x_P\neq x_Q$, on a
$$-P = (x_P, -y_P),\quad P+Q = (\lambda^2 - (x_P + x_Q), -\lambda(x_P + \mu))$$
où
$$\begin{cases}
\lambda = \frac{y_P - y_Q}{x_P - x_Q},\\
\mu = \frac{x_P y_Q - x_Q y_P}{x_P - x_Q}.
\end{cases}$$
Ainsi, la droite d'équation $y = \lambda x + \mu$ est la droite passant par $P$ et $Q$. D'autre part, on a
$$2P = \left(\frac{(3x_P^2 + A)^2 - 8x_P(x_P^3 + Ax_p + B)}{4(x_P^3+Ax_P+B)},\frac{x_P^3-Ax-2B-(3x_P^2+A)x_{2P}}{2y_P}\right).$$
En écrivant $2nP = nP + nP$ et $(2n+1)P = nP + (n+1) P$, on trouve les formules de récurrence annoncées (?).
\end{proof}

On utilise ensuite ces relations de récurrence pour écrire l'algorithme~{\sc PolynômeDeDivision} dans l'étape~\ref{alg:div:poldiv}.

Pour l'étape \ref{alg:div:fact}, {\sc Factorisation}, on utilise l'algorithme de Cantor--Zassenhaus \cite{vzGG}. La première étape est de séparer les facteurs irréductibles d'un polynôme $P$ par degré, ce que l'on fait en prenant le pcgd avec les polynômes $X^p - X$, $X^{p^2} - X$, etc. que l'on calcule directement modulo $P$ par mises au carré répétées. Ensuite on tente de factoriser le reste obtenu $R$ en prenant un polynôme au hasard $a$ et en calculant $\mathrm{pgcd}(a^{\frac{p^r - 1}{2}} + 1, R)$.

Dans l'étape \ref{alg:div:rec}, il s'agit de retrouver les deux polynômes à coefficients dans $k$ définissant des sous-groupes cycliques d'ordre $\ell$. On peut peut-être utiliser {\sc CompareDirection} sur chaque facteur irréductible, pour décider lesquels apparaissent dans $K_1$ ?

Dans l'étape \ref{alg:div:check}, on souhaite vérifier si le polynôme de noyau $K_1$ correspond bien à l'action de l'idéal $\frak l$ donné par $(\ell,v)$. $v$ étant défini par $\pi = v \mod \frak l$, l'endomorphisme $\pi - [v]$ de $E$ est un élément de $\frak l$, et doit donc s'annuler sur le sous-groupe défini par $K_1$. C'est cette vérification qui consistue l'algorithme {\sc CompareDirection}. Concrètement, on effectue ce calcul à l'aide du point tautologique $(X, Y)$ de l'anneau $k[X, Y]/(K_1,\ Y^2 - X^3 - AX - B).$ Le fait que l'égalité soit valide pour tous les points de $E(\bar{\F_p})$ annulés par $K_1$ implique que l'égalité ait lieu dans cet anneau (c'est le Nullstellensatz) : en effet, l'idéal $(K_1,\ Y^2 - X^3 - AX - B)$  est radical (?).


Enfin, on souhaite calculer une équation de la courbe image connaissant $E$ et le noyau $K$ de l'isogénie.

\begin{lem}[Formules de Vélu]
Soit $E$ une courbe elliptique donnée par une équation sous forme de Weierstrass
$$y^2 + a_1 x y + a_3 y = x^3 + a_2 x^2 + a_4 x + a_6,$$
et soit $K = \sum_{i=0}^n (-1)^{n-i} \sigma_i X^i$ un sous-groupe de $E$, cyclique d'ordre $2n+1$. Notons $b_2, b_4, b_6, b_8$ les $b$-invariants de l'équation de $E$, et 
$$ t = 6(\sigma_1^2 - 2 \sigma_2) + b_2 \sigma_1 + n b_4,$$
$$ w = 10 (\sigma_1^3 - 3 \sigma_1 \sigma_2 + 3 \sigma_3) + 2  b_2 (\sigma_1^2 - 2 \sigma_2) + 3 b_4 \sigma_1 + n b_6.$$
Alors, en notant $E'$ la courbe donnée par les invariants
$$ a_1' = a_1,\ a_2' = a_2,\ a_3' = a_3,\ a_4' = a_4 - 5 t,\ a_6' = a_6 - b_2 t - 7 w,$$
il existe une isogénie séparable $E\vers E'$ de noyau $K$.
\end{lem}

\begin{proof}
Soit $G$ le sous-groupe étale de $E$ défini par $K$. On définit deux fonctions rationnelles sur $E$:
$$\begin{aligned}
x_G(P) = x(P) + \sum_{Q\in G\backslash \{0\}} x(P+Q) - x(Q),\\
y_G(P) = y(P) + \sum_{Q\in G\backslash \{0\}} y(P+Q) - y(Q).
\end{aligned}$$
Ces fonctions rationnelles sont bien définies et invariantes par $G$ : elles définissent donc des fonctions sur la courbe $E/G$. On montre alors qu'elles satisfont l'égalité donnée par l'équation de Weierstrass ci-dessus, et qu'elles engendrent le corps des fonctions de la courbe quotient.
\end{proof}

\begin{rem}
On dispose depuis récemment de meilleures formules dans le cas de la forme de Montgomery : si $P$ est un point d'ordre $2n+1$ sur une courbe de Montgomery d'équation $By^2 = x^3 + Ax^2 + x$, alors il existe une isogénie normalisée dont le noyau est le sous-groupe engendré par $P$ vers la courbe d'équation
$$B'y^2 = x^3 + A'x^2 + x$$
avec $A' = (6\sigma + A)\pi^2,\ B' = B\pi^2$, où
$$\begin{aligned}
\pi &= \prod_{i = 1}^n x_{iP}, \\
\sigma &= \sum_{i = 1}^2 \left(\frac{1}{x_{iP}} - x_{iP}\right).
\end{aligned}$$
Comme ci-dessus, on peut réexprimer $\pi$ et $\sigma$ en termes des coefficients du polynôme $K$.

Sans utiliser ces formules, une autre méthode est disponible : étant donné une courbe sous forme de Montgomery, on peut calculer une équation de Weierstrass de cette courbe, et utiliser les formules de Vélu pour cette courbe. En se souvenant du point de 2-torsion, on peut ensuite retrouver une équation de Montgomery pour la courbe quotient sans extraction de racine carrée.
\end{rem}

D'un point de vue pratique, factoriser entièrement un polynôme de degré $\frac{\ell^2 - 1}{2}$ lorsque $p$ est grand et $\ell$ n'est pas très petit, est une opération trop coûteuse. Cela pousse à utiliser une autre méthode à l'aide d'une équation modulaire, ce que l'on décrit maintenant.


\subsection{Calcul à l'aide d'une équation modulaire}

Nous avons vu précédemment que la courbe modulaire $Y_0(\ell)(\F_p)$ paramétrise les isogénies cycliques de degré $\ell$ définies sur $\F_p$. Le polynôme modulaire de degré $\ell$ relie les $j$-invariants de ces courbes ; on a vu de plus que les points singuliers sur $\F_p$ de l'application
$$Y_0(\ell) \vers \A^1 \times \A^1,$$
dont l'image et définie par le polynôme modulaire, sont nécessairement des réductions modulo $p$ de points singuliers complexes, donc peu nombreux (et dont l'ordre a un petit discriminant, on reviendra sur ce point plus tard). En utilisant ces outils, on obtient l'algorithme suivant pour un calcul d'isogénie.

\begin{algorithm}
\caption{{\sc \'Equation modulaire} : un pas dans le graphe d'isogénies en utilisant une équation modulaire}
\label{alg:mod}
\KwIn{Une courbe elliptique $E\in \Ell_t(\O)$, et un idéal représenté par $(\ell, v)$}
\KwOut{Une courbe elliptique $E'$ telle que l'isogénie $E\vers E'$ corresponde à cet idéal}
$\Phi_\ell(X, Y) \gets \textsc{PolynômeModulaire}(\ell)$
\label{alg:mod:polmod}
\;
$(j_1, j_2) \gets \textsc{Racines}(\Phi_\ell(j(E), Y), \F_p)$
\label{alg:mod:roots}
\;
$K_1 \gets \textsc{Noyau}(E, \ell, j_1)$
\label{alg:mod:ker}
\;
\lIf{$\textsc{CompareDirection}(K_1,v)$}{\Return{$\textsc{CourbeElliptique}(j_1)$}}
\label{alg:mod:check}
\lElse{\Return{$\textsc{CourbeElliptique}(j_2)$}}

\end{algorithm}

De la même façon, on étudie les étapes de l'algorithme l'une après l'autre.

Pour le calcul des polynômes modulaires, on renvoie à \cite{Elkies}.

Pour le calcul des racines, on utilise à nouveau l'algorithme de Cantor--Zassenhaus.

L'algorithme {\sc CompareDirection} a déjà été discuté, et nous dirons un mot de l'étape {\sc CourbeElliptique} plus tard. Examinons maintenant le calcul du noyau.

\begin{defi}
Soit $\phi\de E\vers E'$ une isogénie. On a alors une application $k$-linéaire
$$\phi^*\de H^0(E', \Omega^1)\vers H^0(E, \Omega^1).$$
On sait que cette application est nulle si et seulement si $\phi$ est non séparable. Si l'on fixe une équation de Weierstrass pour $E$ et $E'$, on dispose de 1-formes privilégiées et donc d'un isomorphisme (non canonique, donc)
$$\Hom(H^0(E', \Omega^1), H^0(E, \Omega^1))\simeq k.$$
On dit alors que $\phi$ est \emph{normalisée} si $\phi^* = 1$ dans cette identification.
\end{defi}

\begin{prop}
Soit $\phi\de E\vers E'$ une isogénie normalisée de degré $\ell$ entre courbes sous forme de Weierstrass telles que $a_1 = a_3 = a_1' = a_3' = 0$ (ainsi les équations sont de la forme $y^2 = f(x)$). Notons $\phi_x(x,y),\ \phi_y(x,y)$ les applications rationnelles définissant $\phi$ sur l'ouvert affine ; alors $\phi_x$ est une fonction paire, donc peut s'écrire sous forme irréductible
$$\phi_x(x, y) = \frac{N(x)}{D(x)}$$
où $N$ et $D$ sont des polynômes de degré au plus $\ell$. Notons
$$G(X) = a_6 X^3 + a_4 X^2 + a_2 X + 1,\ H(X) =a_6' X^3 + a_4' X^2 + a_2' X + 1.$$
Alors la fraction rationnelle 
$$T(X) = \frac{D(1/X)}{N(1/X)}$$
est solution de l'équation différentielle
$$\frac{T}{X} H(T) - G(X) T'^2 = 0.$$
\end{prop}

\begin{proof}
Comme $\phi$ est normalisée, on sait que $\phi_y (x, y) = y \phi_x'(x).$ En écrivant 
$$\phi_x(x) = \frac{N(x)}{D(x)} = \frac{1}{T(1/x)},$$
on a donc
$$\phi_y(x, y) = \frac{y}{x^2} \frac{T'(1/x)}{T(1/x)^2}$$
et
$$ \left(\frac{y}{x^2} T'(1/x)\right)^2 = \left(\frac{1}{T^3(1/x)} + \frac{a_2'}{T^2(1/x)} + \frac{a_4'}{T(1/x)} + a_6'\right)T(1/x)^4$$
ce qui donne, vu l'équation de $E$ :
$$ \frac{1}{x} G(1/x) T'^2(1/x) =T(1/x) H(T(1/x))$$
et par changement de variable
$$ G(u) T'^2(u) = \frac{T(u)}{u} H(T(u)),$$
ce qui est l'équation annoncée.
\end{proof}

Remarquons que dans cette description, lorsque $\ell$ est impair, le polynôme $D$ est précisément le carré du polynôme de noyau $K_\phi$. Précisons maintenant comment se déroule le calcul du noyau.

\begin{algorithm}
\caption{{\sc Noyau} : Calcul du noyau de l'isogénie}
\label{alg:ker}
\KwIn{Une courbe elliptique $E\in \Ell_t(\O)$, un nombre premier $\ell\neq p$, un élément $j'\in k$ qui est le $j$-invariant d'une courbe reliée à $E$ par une isogénie cyclique de degré $\ell$}
\KwOut{Le noyau de cette isogénie}
$E' \gets \textsc{EquationNormalisée}(E, \ell, j')$
\label{alg:ker:eq}
\;
$T \gets \textsc{SolutionEquaDiff}(E, E')$
\label{alg:ker:newt}
\;
$D \gets \textsc{Dénominateur}(1/T(1/X))$
\label{alg:ker:bm}
\;
$K\gets \textsc{RacineCarrée}(D)$
\label{alg:ker:sqrt}
\;
\Return{$K$}

\end{algorithm}

A nouveau, prenons ces étapes dans l'ordre, à commencer par le calcul d'une équation normalisée.

\begin{lem}
Avec les notations de l'algorithme, soit $\Phi_ \ell(X, Y)$ le polynôme modulaire de degré $\ell$. On suppose que $E$ est donnée sous forme réduite
$$y^2 = x^3 + A x + B.$$
On définit
$$\lambda = \frac{-18}{\ell}\cdot\frac{B}{A}\cdot\frac{\frac{\partial \Phi_\ell}{\partial X} (j(E), j')}{\frac{\partial \Phi_\ell}{\partial Y} (j(E), j')} \cdot j(E) $$
Alors, si $E'$ désigne la courbe d'équation
$$y^2 = x^3 + A' x + B' $$
avec
$$A' = \frac{-\lambda^2}{48 \ell^4 j' (j' - 1728)},\ B' = \frac{-\lambda^3}{864 \ell^6 j'^2(j'-1728)},$$
on a $j(E') = j'$ et il existe une isogénie normalisée $E\vers E'$ de degré $\ell$.
\end{lem}

\begin{proof}
Dans le cas complexe, on a des égalités entre séries formelles :
$$ A = \frac{- E_4}{48}, \quad B = \frac{E_6}{864}$$
et on peut tout écrire explicitement en fonction de $q$ \cite{Schoof}. Dans le cas d'un corps fini, on peut utiliser le théorème de Deuring vu précédemment. L'isogénie de degré $\ell$ correspond à un idéal de norme $\ell$, qui est donc inversible lorsque $l$ est premier au discriminant : ainsi on sait que l'isogénie est la réduction d'une isogénie définie sur les complexes.
\end{proof}

Remarquons que l'on peut tout à fait utiliser ce calcul pour la fonction {\sc CourbeElliptique} précédente. On pourrait en fait choisir $\lambda$ arbitrairement, mais le coût de ce calcul est de toute façon négligeable.

Pour résoudre l'équation différentielle jusqu'à une précision donnée, on utilise une itération de Newton.

\begin{lem} Avec les notations de l'algorithme~\ref{alg:ker}, on définit
$$T_0 = 0, \quad \forall i\geq 0,\ T_{i+1} = T_i + T_i' \sqrt{G} \sqrt{x} \int \frac{k_i(x)}{2\sqrt{x}}.$$
Alors pour tout $i\geq 0$, $T_i$ est une solution de l'équation différentielle modulo $x^{2i+1}$.
Résultat d'unicité ?
\end{lem}

\begin{proof}

\end{proof}

 Pour récupérer le dénominateur, un simple algorithme d'Euclide étendu permet de récupérer les polynômes $N$ et $D$ lorsque l'on connaît les coefficients de la série formelle $\frac{N}{D}$ jusqu'au degré $2\ell + 1$. En effet, si l'on connaît un polynôme $U$ tel que $U = \frac{N}{D} \mod X^{2\ell + 1}$, alors il existe un polynôme $P$ tel que $D U = N + P X^{2\ell+1}$, ce que l'on peut réécrire
 $$D U - P X^{2\ell + 1} = N.$$
On lance alors l'algorithme d'Euclide étendu avec $U$ et $X^{2\ell + 1}$, que l'on arrête dès que les coefficients $N$ et $D$ sont tous deux de degré inférieur à $\ell$. Là encore, on a un résultat d'unicité.

Pour le calcul de racine carrée, on utilise la remarque suivante :

\begin{lem}
Si $D = K^2$ où $K$ est un polynôme séparable, alors $K$ et $\mathrm{pgcd}(D, D')$ sont associés.
\end{lem}

En effet, on a $D' = 2 K K'$ et pgcd$(K, K')$ = 1 puisque $K$ est séparable.

Le coût principal de l'algorithme~\ref{alg:mod} est partagé entre le calcul des racines et l'algorithme {\sc CompareDirection}, qui impliquent tous deux de calculer une puissance $p$-ième modulo un polynôme de degré environ $\ell$. C'est une amélioration importante d'un point de vue pratique par rapport au $\ell^2$ précédent.

\subsection{Calcul à l'aide de torsion rationnelle}

Pour certaines valeurs propres spéciales du Frobenius, on peut proposer une autre méthode qui évite complètement le calcul d'une puissance $p$-ième. Par exemple, si 1 est valeur propre modulo $\ell$, cela signifie qu'il existe un sous-groupe cyclique d'ordre $\ell$ constitué de points rationnels sur $k$ (et un seul, car les valeurs propres sont distinctes). On obtient alors l'algorithme suivant.

\begin{algorithm}
\caption{{\sc Torsion} : un pas dans le graphe d'isogénies à l'aide de points de torsion}
\label{alg:tors}
\KwIn{Une courbe elliptique $E\in \Ell_t(\O)$, et un idéal représenté par $(\ell, v)$ tel que $v = 1$}
\KwOut{Une courbe elliptique $E'$ telle que l'isogénie $E\vers E'$ corresponde à cet idéal}
$P \gets \textsc{PointDeTorsion}(E, \ell)$
\label{alg:tors:tors}
\;
$K \gets \textsc{SousGroupeEngendré}(E, P)$
\label{alg:tors:sg}
\;
$E' \gets \textsc{Quotient}(E, K)$
\label{alg:tors:quo}
\;
\Return{$E'$}

\end{algorithm}

Pour trouver un point de torsion, on procède de la manière suivante.

\begin{algorithm}
\caption{{\sc PointDeTorsion} : calcul d'un point de torsion rationnel}
\label{alg:ptors}
\KwIn{Une courbe elliptique $E\in \Ell_t(\O)$, et un nombre premier d'Elkies $\ell$ tel que $E$ admet un point de $\ell$-torsion rationnel}
\KwOut{Un point de $\ell$-torsion primitif}
$C \gets q + 1 - t$
\label{alg:ptors:card}
\;
\Repeat{$P \neq 0$}{
$Q \gets \textsc{PointAuHasard}(E,k)$
\;
$P \gets \left[\frac{C}{\ell}\right]Q$
\label{alg:ptors:scal}
    }
\Return{$P$}

\end{algorithm}

L'algorithme {\sc SousGroupeEngendré} utilise simplement $\frac{\ell-1}{2}$ additions sur la courbe.

Le coût principal de cet algorithme est l'étape~\ref{alg:ptors:scal}. Pour calculer ce type de multiplications scalaires, on utilise un mécanisme de \emph{double-and-add}, qui utilise environ $\log(C)$ (c'est à dire $\log(p)$) additions sur la courbe. Cela représente un gain important par rapport aux algorithmes~\ref{alg:div} et~\ref{alg:mod}. Lorsque $\ell$ devient grand, c'est l'étape {\sc SousGroupeEngendré} qui devient coûteuse.

Bien évidemment, l'algorithme~\ref{alg:tors} n'est utilisable que pour les nombres premiers $\ell$ pour lesquels 1 est valeur propre. Intuitivement, il y a environ une chance sur $\ell$ que cela survienne pour une courbe choisie \og au hasard \fg . Pour profiter pleinement de cette accélération, il faut donc choisir des courbes adéquates. En pratique, en utilisant des courbes sous forme de Montgomery, on peut se dispenser de calculer les coordonnées $y$ : la condition devient alors que $+1$ ou $-1$ soit valeur propre.

Remarquons que l'on peut donner une variante de l'algorithme~\ref{alg:tors} lorsque les points de $\ell$-torsion ne sont pas nécessairement $k$-rationnels, mais seulement définis sur une extension de $k$ de petit degré. Il faut alors multiplier un point choisi au hasard par $\frac{C'}{\ell}$, où $C'$ est le cardinal de la courbe sur cette extension (qui peut se calculer directement à partir de $C$ par la théorie de Hasse). Pour assurer qu'il n'y ait qu'un seul groupe cyclique d'ordre $\ell$ défini sur cette extension minimale, il faut alors demander la propriété suivante :

\begin{defi}
Soit $E$ une courbe elliptique sur $k$ et $\ell$ un nombre premier. On dit que $\ell$ est \emph{adapté} si $\ell$ est un nombre premier d'Elkies et si les ordres multiplicatifs des valeurs propres du Frobenius modulo $\ell$ sont distincts.
\end{defi}

De plus, utiliser l'algorithme~\ref{alg:tors} n'est intéressant que lorsque le degré de l'extension est suffisamment petit, c'est à dire lorsque le minimum de ces deux ordres est petit. Pour rechercher des courbes ayant beaucoup de nombres premiers adaptés, on utilise une variante de l'algorithme utilisé pour calculer le nombre de points d'une courbe elliptique sur un corps fini, que l'on peut présenter sous la forme suivante.


\begin{algorithm}
\caption{{\sc EstAdapté} : décide si un nombre premier est adapté à une courbe}
\label{alg:adapt}
\KwIn{Une courbe elliptique $E$, et un nombre premier $\ell$}
\KwOut{{\bf True} si $\ell$ est adapté, {\bf False} sinon}
$\Phi_\ell \gets \textsc{PolynômeModulaire}(\ell)$
\label{alg:adapt:polmod}
\;
$(j_1,j_2) \gets \textsc{Racines}(\Phi_\ell(j(E), Y), k)$
\label{alg:adapt:roots}
\;
$(K_1, K_2) \gets (\textsc{Noyau}(E,\ell,j_1), \textsc{Noyau}(E,\ell,j_2))$
\label{alg:adapt:ker}
\;
$(v_1, v_2) \gets (\textsc{ValeurPropre}(E, K_1), \textsc{ValeurPropre}(E, K_2))$
\label{alg:adapt:eigen}
\;
$(o_1, o_2) \gets (\textsc{Ordre}(v_1),\textsc{Ordre}(v_2))$
\label{alg:adapt:ord}
\;
\Return{$o_1\neq o_2$}
\end{algorithm}

L'étape~\ref{alg:adapt:eigen} est analogue à l'algorithme {\sc CompareDirection} précédent : on teste simplement chacun des éléments de $\F_\ell$ les uns après les autres jusqu'à trouver la valeur propre. Enfin, comme $\ell \ll p$ dans notre utilisation, on peut se contenter d'une recherche exhaustive pour la fonction {\sc Ordre}.

Procéder ainsi se révèle beaucoup moins coûteux que de calculer le nombre de points de $E$, puis décider si $\ell$ est adapté ou non en observant la factorisation de $X^2 - tX + q$. Cela permet de tester beaucoup de courbes, en les évacuant rapidement s'il n'y a pas assez de nombres premiers adaptés parmi les petites valeurs.

Une fois que l'on a trouvé une courbe qui semble intéressante, on calcule son cardinal entièrement à l'aide de l'algorithme de Schoof--Atkin--Elkies \cite{Schoof}. L'idée de cet algorithme polynomial est de calculer la trace du Frobenius modulo plusieurs petits nombres premiers grâce à un calcul comme ci-dessus, puis de récupérer la trace entière grâce au théorème chinois, connaissant les bornes de Hasse. Enfin, on regarde quels nombres premiers sont adaptés parmi les plus grandes valeurs en regardant la factorisation de $X^2 - tX + q$.

Pour connaître le temps nécessaire à cette recherche, on aimerait connaître le nombre de courbes elliptiques ayant $\ell$ points de $\ell$-torsion rationnels, pour un nombre premier $p\neq \ell$. Pour que $\ell$ soit adapté, il faut en particulier que l'on ait $p\neq 1$ mod $\ell$, car $p$ est le produit des valeurs propres du Frobenius : on se place donc dans ce cas.

\begin{thm}
Soit $p>3$ un nombre premier, et $\ell\neq p$ un nombre premier tel que $p\neq 1 \mod\ell$. Alors on a
$$ \#'\left\{E/\F_p \right\}/\mathrm{isom.} = p.$$
De plus, il existe une constante explicite $C$ telle que
$$ \left| \frac{p}{\ell - 1}  - \#' \left\{E/\F_p\ :\ \#E(\F_p) = 0 \mod\ell\right\}/\mathrm{isom.} \right|\leq C\ell\sqrt{p}.$$
\end{thm}

Dans ce théorème, $\#'$ désigne la cardinalité \emph{pondérée}, c'est à dire que la classe d'isomorphisme d'une courbe $E$ est comptée avec le poids $\frac{1}{\#\mathrm{Aut}(E)}$. Cette correction correspond aux seules courbes de $j$-invariant 0 ou 1728. Ainsi, la proportion de courbes ayant un point rationnel de $\ell$-torsion lorsque $p\neq 1 \mod \ell$ est $\frac{1}{\ell - 1} + O(\frac{\ell}{\sqrt{p}}).$

Bien sûr, le fait d'admettre de la $\ell_1$-torsion rationnelle et de la $\ell_2$-torsion rationnelle ne sont pas des événements indépendants, mais on peut disposer d'un résultat analogue lorsque $\ell$ n'est pas premier. Ainsi, on ne peut pas espérer trouver une courbe ayant beaucoup de points de torsion rationnels (ou même définis sur une extension de petit degré) : en pratique, 20 nombres premiers différents est le maximum que l'on puisse faire.

UNE REMARQUE pour accélérer encore ces calculs. L'étape coûteuse est le calcul d'un point de torsion, on peut donc précalculer des points de $\prod \ell_i$-torsion. On peut stocker un point pour chaque liste de premiers, répartis selon le degré où vivent les points de torsion.
Exemple :
\begin{itemize}
\item Un point de torsion pour les degrés 1,
\item Un point de torsion pour les degrés 3,
etc.
\end{itemize}

Mais on ne peut les utiliser qu'une seule fois...


\subsection{Recherche de courbes}

Une première stratégie, peu coûteuse en mémoire, consiste à tirer des courbes au hasard et tester si des nombres premiers sont adaptés, les uns après les autres. On jette la courbe si un de ces tests rate. Une autre façon de procéder est de choisir une courbe elliptique associée à un point d'une courbe modulaire. Si l'on connaît une forme explicite de la courbe modulaire $X_1(N)$ ainsi que du revêtement
$$j\de X_1(N)\vers X(1),$$
on peut prendre des courbes candidates dont la cardinalité sur $\F_p$ sera divisible par $N$.

On recherche une équation de $X_1(N)$ sous forme de courbe plane, valable au moins sur le corps des nombres complexes (on espère alors que cette équation est à coefficients entiers et donne bien une équation de $X_1(N)_{\F_p}$ une fois réduite modulo $p$). On recherche donc deux formes modulaires $f$, $g$ de niveau $\Gamma_1(N)$, et un polynôme $\Phi_N^{(f, g)}$ à coefficients entiers tel que
$$\Phi_N^{(f, g)}(f, g) = 0.$$

On suit ici l'article \cite{Baaziz}. On note $\wp(z, \Lambda_\tau)$ la fonction $\wp$ de Weierstrass associé au réseau
$$\Lambda_\tau = \Z\oplus\Z \tau.$$

\begin{thm}
On définit deux fonctions méromorphes de la variable complexe :
$$f(\tau) = \frac{(\wp(1/N, \Lambda_\tau) - \wp(1/N, \Lambda_\tau))^3}{\wp'(1/N, \Lambda_\tau)^2}, \quad
g(\tau) = \frac{\wp'(2/N, \Lambda_\tau)}{\wp'(1/N,\Lambda_\tau)}.$$
Alors $f$ et $g$ sont deux fonctions modulaires pour $\Gamma_1(N)$, et engendrent le corps des fonctions de la courbe $X_1(N)$. Elles vérifient l'équation
$$\psi_N(1+g, f, f) = 0$$
où $\psi_N(a_1, a_2, a_3)$ est le $N$-ième polynôme de division de la courbe d'équation
$$y^2 + a_1xy + a_3y = x^3 + a_2x^2$$
évalué au point $(x, y) = (0, 0)$.
\end{thm}

Remarquons que le calcul de ce polynôme de division évalué en $(0, 0)$ est facile, en utilisant des formules récursives comme précédemment (?). Rappelons que le polynôme de division s'annule exactement sur les points affines de $N$-torsion de la courbe.

\begin{lem}
Toute classe d'isomorphisme de courbes elliptiques $(E, P)$ où $P$ est un point d'ordre $N$ sur un corps $K$ algébriquement clos contient un représentant de la forme
$$E\de y^2 + ((1+g) x + f)y = x^3 + fx^2,\ P = (0,0).$$
avec $f\in K^\times,\ g\in K$. Ce représentant est unique à unique isomorphisme près.
De plus, si $K=\C$ et $(E, P) = (\C/\Lambda_\tau, 1/N)$, on a
$$f = \frac{(\wp(1/N, \Lambda_\tau) - \wp(1/N, \Lambda_\tau))^3}{\wp'(1/N, \Lambda_\tau)^2}, \quad  g = \frac{\wp'(2/N, \Lambda_\tau)}{\wp'(1/N,\Lambda_\tau)}.$$
\end{lem}
Ce lemme s'établit à l'aide de calculs sur les modèles de Weierstrass. L'ingrédient clé est que la courbe elliptique $\C/\Lambda_\tau$ est isomorphe à la courbe elliptique $y^2 = x^3 + g_2(\tau) x + g_3(\tau)$, l'isomorphisme étant donné par 
$z\mapsto (\wp(z), \wp'(z))$.

\begin{proof}[Démontration du théorème]. On vérifie que les fonctions méromorphes
$$\tau\mapsto \wp(1/N, \Lambda_\tau),\quad \tau\mapsto\wp'(1/N,\Lambda_\tau),\quad \tau\mapsto\wp''(1/N, \Lambda_\tau)$$
sont des formes modulaires pour $\Gamma_1(N)$ de poids respectifs 2, 3 et 4. On montre alors que $f$ et $g$ sont des fonctions modulaires (de poids 0) pour $\Gamma_1(N)$ à l'aide des formules de duplication, donc des fonctions sur $X_1(N)_\C = \overline{\Gamma_1(N)\backslash \H}$. Si $a$ est un point affine de la courbe modulaire, selon le lemme, $a$ admet un représentant $(E, P)$ qui s'écrit
$$E\de y^2 + ((1 + g(a))x + f(a))y = x^3 + f(a)x^2,\quad P = (0, 0).$$
$P$ étant un point de $N$-torsion, on a bien $\psi_N(1+g(a), f(a), f(a)) = 0$. \'Etant vraie sur un ouvert affine, cette égalité est vraie sur tout $X_1(N)$.

Il reste à montrer que $f$ et $g$ engendrent le corps des fonctions de $X_1(N)_\C$. Pour cela, il suffit de trouver un ouvert non vide $U$ de la surface de Riemann compacte $X_1(N)_\C$ tel que $(f, g)$ sépare les points sur $U$. Or, c'est le cas sur l'ouvert affine $Y_1(N)$, car le lemme donne un représentant d'un tel point en fonction uniquement de $f$ et $g$.
\end{proof}

\begin{rem}
Le fait que $f$ et $g$ engendrent le corps des fonctions montrent que l'application rationnelle
$$X_1(N)_\C \overset{(f,\ g)}{\vers} \left\{(u, v)\in \P^2(\C) \de \psi_N(1+v, u, u) = 0\right\}$$
est un isomorphisme birationnel sur son image : on a donc obtenu un modèle plan de la courbe modulaire $X_1(N)$. Bien évidemment, cela ne démontre pas directement que ce modèle est bien valable sur un corps fini, et il faudrait sans doute invoquer des arguments similaires à ce qui a déjà été fait sur les polynômes modulaires classiques et les courbes $X_0(N)$ pour justifier cela. Néanmoins, on admet ici que les équations obtenues restent valables sur $\F_p$ sans modification.
\end{rem}

Pour le problème qui nous intéresse, générer des points sur la courbe $X_1(N)_{\F_p}$, il est intéressant de chercher une équation minimisant le degré d'une des deux variables. On peut alors tirer la seconde au hasard, et on se retrouve à calculer les racines d'un polynôme de plus petit degré. Cela motive la définition suivante.

\begin{defi}
Soit $X$ une courbe algébrique sur un corps algébriquement clos $K$. On appelle \emph{gonalité} de $X$ le plus petit entier $k\geq 1$ tel qu'il existe une application rationnelle de degré $k$ de $X$ vers la droite projective. Autrement dit, c'est le plus petit entier $k$ tel qu'il existe une extension $K(X)/K(f)$ de degré $k$ pour un certain $f\in K(X)$.
\end{defi}

La gonalité mesure un défaut de rationalité : les courbes de gonalité 1 sont exactement les courbes birationnelles à $\P^1$.
Si l'on dispose d'une équation plane $\Phi(X, Y) = 0$ pour la courbe $\Cl$ (définie sur un corps), on peut majorer la gonalité de $\Cl$ par le degré de $\Phi$ en chacune de ses variables. En effet, soit $p_1 : \Cl\vers \P^1$ l'application donnée par la première variable. Il existe un ouvert $U$ de $\P^1$ tel que tout $x\in U$ admet exactement $d$ préimages distinctes par $p_1$, quitte à prendre une clôture algébrique, où $d$ est le degré de $\P^1$. Ces préimages sont de la forme $(x, y)$ où $y$ est une racine du polynôme $\Phi(x, Y)$ : ainsi on a $d \leq \mathrm{deg}_Y \Phi$.

En toute généralité, on ne sait pas donner de gonalité d'une courbe : ainsi, la gonalité de $X_1(N)$ est inconnue en dehors des plus petites valeurs de $N$. (Voir cependant un algorithme de calcul de la gonalité). Le genre intervient également, bien qu'il ne détermine pas la gonalité :

\begin{prop}
Soit $\Cl/K$ une courbe de genre $g$ et de gonalité $k$. Alors on a
$$k\leq \frac{g+2}{3}.$$
De plus, si $\Cl = X_1(N)$, on a également l'inégalité
$$\frac{21}{100}(g-1)\leq k.$$
\end{prop}

\begin{proof} Pour le premier point,\ldots
Pour le second, on renvoie à \cite{gon}.
\end{proof}


Pour avoir une intuition sur le degré des polynômes mis en jeu, on allie cette propriété au résultat suivant :

\begin{prop}
Soit $N\geq 1$ un entier. Alors le genre de $X_1(N)$ est 
$$  g(N) = 
\begin{cases}
0 &\text{si}\ 1\leq N\leq 4 \\
\displaystyle 1 + \frac{N^2}{24} \prod_{\substack{p|N \\ p\ \mathrm{premier}}} \left(1 - \frac{1}{p^2}\right) + \sum_{\substack{d|N\\ d\geq 0}} \varphi(d)\varphi\left(\frac{N}{d}\right)& \text{sinon,}
\end{cases}$$
où $\varphi$ désigne la fonction indicatrice d'Euler.
\end{prop}

\begin{proof}
\cite{Coreens}.
\end{proof}

On peut donc s'attendre à trouver des équations pour $X_1(N)$ de degré de l'ordre de $N^2$ en chaque variable. En pratique, on peut s'arranger pour réduire ce degré d'un facteur constant. Une première étape est de remarquer que $\psi_N(1+g, f, f)$ possède un facteur $f^{v_N}.$ On peut déterminer cette valuation à partir des formules de récurrence, et on obtient
$$ v_N =
\begin{cases}
3k^2 & \text{si}\ N = 3k, \\
3k^2 + 2k &\text{si}\ N = 3k+1, \\
3k^2 + 4k + 1 &\text{si}\ N = 3k+2.
\end{cases}$$
En pratique, on calcule directement la quantité $f^{-v_N}\psi_N$, pour laquelle on peut donner une formule récursive semblable. De plus, si $M$ divise $N$, alors $\psi_M(1+g, f, f)$ divise $\psi_N(1+g, f, f)$ (si $MP = 0$, on a aussi $NP = 0$). On définit donc des polynômes $\Phi^{(f, g)}$ vérifiant :
$$\forall N\geq 1,\ f^{-v_N} \psi_N(1+g, f, f) = \prod_{M | N} \Phi_M^{(f, g)}.$$

On a alors $\Phi_N^{(f, g)}(f, g) = 0$. Une fois cette équation calculée, on peut tenter  un changement de variables pour la simplifier. On pose ainsi successivement :
$$\begin{matrix} f = st(t-1), & g = s(t-1)\\
 s = q\frac{r+1}{r} - 1, & t = q(r+1) + 1.
 \end{matrix}$$
On peut aisément inverser ces changements de variables pour vérifier que $q$ et $r$ engendrent toujours le corps des fonctions de $X_1(N)$. L'équation obtenue reste à coefficients entiers, et est de plus petit degré que $\Phi^{(f, g)}$ ; expliquer cette constatation expérimentale reste une question ouverte (voir par exemple \cite{Galbraith} et \cite{Baaziz}). Une autre approche serait d'effectuer des changements de variables \og aléatoires \fg\ jusqu'à en trouver de plus intéressants : c'est par exemple l'approche de \cite{Sutheq}.

Vu la remarque précédente, il faut s'attendre à trouver des polynômes bivariés de degré $N^2$ en chaque variable, c'est à dire de l'ordre de $N^4$ coefficients. De plus, ces coefficients entiers deviennent plus grands avec $N$, ce qui induit un coût supplémentaire. En conclusion, cette construction d'une équation est un algorithme certes polynomial en $N$, mais qui devient rapidement impraticable quand $N$ augmente, même s'il reste plus intéressant que la recherche au hasard de courbes ayant des points de $N$-torsion. En pratique, on a calculé une équation de $X_1(N)$ pour $N = \dots$, qui est de degré $\dots$ en $q$ et $\dots$ en $r$, la valeur absolue du plus grand coefficient étant $\dots$.

\begin{rem}
Dans l'optique d'accélérer le calcul d'isogénies, les courbes ayant beaucoup de premiers d'Elkies sont intéressantes au même titre que les courbes ayant des points de torsion rationnels. Nous n'avons pas cherché à savoir quelle était la meilleure méthode de recherche, car les gains de temps découlant de la présence d'un nombre premier d'Elkies supplémentaire, par exemple, sont difficiles à estimer.

En plus de la recherche brute et de l'utilisation de la courbe modulaire $X_1(N)$, il est possible d'utiliser, par exemple, la courbe $X_0(30)$. Expliquer \dots
\end{rem}

\subsection{Catalogue de courbes}

Donnons maintenant les résultats obtenus par différentes méthodes. Une fois les candidats désignés, on calcule le cardinal entièrement à l'aide de l'algorithme SEA, implanté par exemple dans PARI \cite{PARI}. Le principe de ce dernier est de calculer la trace du Frobenius modulo $\ell$ en travaillant modulo les polynômes de division de la courbe, pour des nombres premiers dont le produit est supérieur à $4\sqrt{p}$, puis à retrouver la trace entière entre les bornes de Hasse par le théorème chinois. Lorsque $\ell$ est d'Elkies, on peut gagner du temps en calculant le polynôme de noyau d'une isogénie de degré $\ell$ et la valeur propre du Frobenius $v\in \F_\ell$ associée, puis à récupérer la trace $t = v + p/v \mod\ell.$

%%%%%%%%%%%%%%%%%%%%%%%%%%%%%

$$p = 2^{502} + 49$$

On donne une liste de courbes sous forme de Montgomery :

$$E\ :\ y^2 = x^3 + A x^2 + x$$

qui ont des bonnes propriétés.

\begin{enumerate}
\item Une courbe pour laquelle on a dix premiers $\leq 120$ pour lesquels on peut utiliser la torsion rationnelle :
$$\begin{aligned}
a_1 =\ & 729315552663468518500402605869119052890462281082522742 \\
& 859468783872284240397481195217026275077304977485894385 \\
& 4970570777116006830331605352557759228257209
\end{aligned}$$
\item Une courbe qui a 14 nombres premiers avec des degrés d'extension modérés :
$$\begin{aligned}
a_2 =\ & 133341838804869238383216931284940524254637049733151939 \\
& 187754962716865583333147735978587954228126379795456832 \\
& 5268646879531429895399551361424391894986266
\end{aligned}$$
\item Une courbe qui a 12 premiers pour lesquels on peut utiliser la torsion rationnelle : (non)
$$\begin{aligned}
a_3 =\ & 564832868630758045506735044707612343080602902691601047 \\
& 590112526158976116993799041098269970288625684141753788 \\
& 4242244982087915458779319438801478285816733
\end{aligned}$$
\item Une courbe avec 16 premiers avec des extensions de degrés modérés, dont 10 rationnels :
$$\begin{aligned}
a_4 =\ & 121130277156411694343217230283055288213350601021857395 \\
& 825089516084303047229525156440432661253747992379693413 \\
& 57415871805293887568309777771549544670721692
\end{aligned}$$
\item Une courbe avec 18 premiers avec des degrés modérés, dont 10 rationnels :
$$\begin{aligned}
a_5 =\ & 8377883523781066605019538779691857966316879702452531886 \\
& 815339413567302883201831127809595853562762084636532865 \\
& 243960105048527932637249276459758642731879
\end{aligned}$$
\item Une courbe avec 18 premiers de cette sorte mais des degrés d'extension plus grands :
$$a_6 = 1202$$

\end{enumerate}



\newpage

\section{Attaques}


\subsection{Preuve de sécurité}


\begin{thm}

\end{thm}


\subsection{Attaques classiques génériques}

Partant de l'hypothèse que l'ordre du groupe de classe de la courbe choisie est environ $\sqrt{p}$, l'objectif de cette section est de déterminer des paramètres qui permettent d'atteindre 128 bits de sécurité, et d'estimer le coût en temps d'un échange de clés.

Soit $E$ une bonne courbe elliptique, par exemple :

$$exemple$$

On note $L_1$ l'ensemble des nombres premiers $\ell$ tels que $E$ ou son twist admet un point rationnel primitif de $\ell$-torsion, et $L_2$ son analogue sur une extension de degré modéré ($3\leq d\leq 5$ par exemple). On note $L_3$ l'ensemble des autres nombres premiers d'Elkies de $E$ inférieurs à 1000.

Vu les valeurs précédentes, le coût d'un pas est alors d'environ $c_1 = 7$ ms dans $L_1$, $c_2 = 100$ ms dans $L_2$, et $c_3 = 10 s$ dans $L_3$. Pour les meilleures courbes, on a par exemple
$$\# L_1 = 10,\ \# L_2 = 10,\ \# L_3 = 100.$$

Mettons que l'on effectue le même nombre de pas à l'intérieur de chacune de ces familles : soient $n_1, n_2, n_3$ les nombres de pas correspondants. On souhaite alors minimiser
$$ 7\cdot 10 n_1 + 100\cdot 10 n_2 + 10000 \cdot 100 n_3$$
sous la contrainte
$$ (n_1 + 1)^{10}(n_2 + 1)^{10}(2 n_3 + 1)^{100} \geq 2^{256}.$$

Par exemple, en prenant $n_3 = 1$, $n_2 = 10$, $n_1 = $, ces conditions sont vérifiées et le coût total en temps d'un échange de clés est alors dominé par le troisième terme, ce qui représente environ 20 minutes de calcul.

\subsection{Choix des paramètres}

On donne les meilleurs résultats obtenus par ces programmes. Ces valeurs sont obtenues avec le nombre premier $ p = 2^{500} + 55$ et la courbe $E/\F_p$ donnée par
$$E\ :\ y^2 = x^3 + 5x^2 + x.$$

$$\begin{matrix}
\ell & 11 & 13 & 17 & 23 & 37 & 43 & 47 & 53 & 59 \\
\text{Temps\ Elkies\ (ms)} & 120 & 130 & 210 & 300 & 640 & 740 & 820 & 1000 & 1200 
\end{matrix}$$

$$\begin{matrix}
\ell & 67 & 71 & 79 & 83 & 89 & 97 &\qquad & &\\
\text{Temps\ Elkies\ (ms)} & 1500 & 1550 & 1750 & 1900 & 2100 & 2250 & & &
\end{matrix}$$

$$\begin{matrix}
d & 1 & 3 & 4 & 5 & 7 & 9 & 13 & 16 & 19\\
\text{Temps\ torsion\ (ms)} & 7 & 60 & 96 & 170 & 510 & 740 & 1620 & 2040 & 3600
\end{matrix}$$
plus le coût linéaire à ajouter en $d\ell$.
\newpage

\subsection{L'attaque quantique de Jao et Soukharev}

\section{Appendice}

\subsection{Notions de géométrie algébrique}



\newpage

\nocite{*}

\bibliographystyle{plain}

\bibliography{biblio}


\end{document}
